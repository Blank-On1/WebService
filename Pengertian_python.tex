\documentclass[12pt,a4paper]{article} 
\linespread{1.5}
\begin{document}
\title{Pengertian Python}
\maketitle

\begin{itemize}
\item
Nama Kelompok 1\\
Farid Ariyanto Saputra 1164034\\
Nurgivani Syarifatul Husna 1164050\\
Velariza Alvioletta 1164056\\
Yogi Aditya Saputra 1164060 \\
\end{itemize}

\section{REST}
\subsection{Pengertian Python}
Python merupakan salah satu Bahasa pemrograman yang bersifat open source yang tertafsir oleh typing yang dinamis dan kuat. Python juga memiliki banyak library, seperti struktur data, files, dan jaringan. Bahasa pemrograman python juga banyak digunakan untuk berbagai keperluan, contohnya komputasi ilmiah, system administrasi, dan pengembangan web. Selain itu pula, keuntungan Bahasa pemrograman python yakni memiliki alat simulasi python gratis.
\subsection{Pengertian Python}
Python adalah suatu bahasa pemrograman yang bisa dikatakan bahasa pemrograman jaman sekarang, karena usianya sangat muda namun sudah banyak digunakan oleh programmer. Phyton dapat mendukung dalam membangun aplikasi berbasis desktop, web, mobile maupun lainnya. Untuk membangun sebuah aplikasi, bahasa pemrograman ini juga bisa digunakan menggunakan framework maupun tanpa framework. Namun, apabila tidak menggunakan framework akan membutuhkan waktu yang lama dalam tahap membangun aplikasi, begitu juga sebaliknya apabila menggunakan framework pembangunan aplikasi akan menjadi lebih cepat dan terstruktur, biasanya framework yang digunakan adalah Django, dimana disana terlah tersedia komponen seperti models, templates, views, forms, dan admin interface.
\end{document}