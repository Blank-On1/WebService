\documentclass[12pt, a4paper]{article}
\begin{document}

\begin{itemize}
        \item
        Ahmad Syafrizal Huda (1164062) \\
        Annisa Fathoroni (1164067) \\
        Puad Hamdani (1164084) \\
        Rahmi Roza (1164084) \\
        Tasya Wiendhyera (1164086) \\
\end{itemize}
    Secara harfiah internet adalah kependekan dari “interconnected network” yang berarti rangkaian komputer yang terhubung satu sama lain. Hubungan melalui suatu sistem antar perangkat komputer untuk lalu lintas data itulah yang dinamakan network. Sebagai contoh yaitu LAN, MAN dan WAN. Misalnya LAN. LAN merupakan singkatan dari Local Area Network yang menghubungkan komputer atau jaringan dalam area tertentu seperti kantor, sekolah atau warnet. Jadi komputer yang terhubung melalui satu jaringan dan saling berkomunikasi dengan waktu dan wilayah yang tidak terbatas, disebut internet.

    Internet di dunia bisnis untuk pergantian informasi, pencatatan produk,media yang mempromosikani, surat elektronik, bulletin boards, kuesioner elektronik, dan mailing list. Biasanya digunakan untuk berkomunikasi,berdiskusi, dan dilibatkan secara proaktif dan interaktif dalam perancangan, pengembangan,pemasaran, dan penjualan produk. Pemasaran melalui internet terdapat 2 metode, yaitu push dan pull marketing. keutamaan dari perencanaan bisnis yang didapat di internet ialah komunikasi dunia dan interaktif diantaranya, serta menyediakan informasi penting dan pelayanan yang sesuai dengan kebutuhan konsumen, juga meningkatkan kerja sama.
    
    Internet of Things (IOT) merupakan perkembangan keilmuan yang sangat menjanjikan untuk mengoptimalkan kehidupan berdasarkan sensor cerdas dan peralatan pintar yang bekerjasama melalui jaringan internet (Keoh, Kumar, & Tschofenig, 2014).Menurut (Burange & Misalkar, 2015) Internet of Things (IOT) adalah struktur di mana objek. orang disediakan dengan identitas eksklusif dan kemampuan untuk pindah data melalui jaringan tanpa memerlukan 2 arah antara manusia ke manusia yaitu sumber ke tujuan atau interaksi manusia ke komputer.
\end{document}
