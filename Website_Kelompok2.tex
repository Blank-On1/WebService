\documentclass[12pt, a4paper]{article}
\linespread{1.5}

\begin{document}
\title{World Wide Web}
\maketitle

\begin{itemize}
\item
	Imron Sumadireja (1164076) \\
	Jesron Marudut (1164077) \\
	Lusia Violita Aprilian (1164080) \\
	Mhd. Zulfikar Akram Nst. (1164081) \\
\end{itemize}

\section{Pengertian Website}
World wide web (www atau web) merupakan halaman situs informasi yang dapat diakses secara cepat atau sarana antar muka informasi di internet. Web dapat menggabungkan teks, grafik, dan multimedia. Web memudahkan penggunanya untuk mengakses informasi melalui konsep hypertext sehingga memungkinkan  suatu text untuk menjadi acuan membuka dokumen laindo. Informasi dapat mudah disebar dan diakses.

Sementara itu World wide web (www) dikembangkan pertama kali oleh Tim Berners-Lee pada tahun 1989. Pada awalnya, Tim mengusulkan WWW sebagai suatu cara berbagai dokumen diantara para peneliti. Dokumen online dapat diakses melalui alamat unik yang disebut Universal Resource Locator atau URL. Selain itu WWW tidak hanya dikembangkan untuk keperluan para peneliti, namun juga dikembangkan untuk kalangan pendidikan, bisnis dan perorangan. Berdasarkan penjelasan singkat diatas dapat disimpulkan bahwa antara web dan internet memiliki hubungan yang sangat erat walaupun keduangnay tidak bisa dikatakan sama. Web merupakan bagian dari layanan yang dapat berjala di atas teknologi internet.

Website dikelompokan dalam beberapa jenis-jenis Website agar dapat memudahkan dalam menentukan jenis website yang akan ditentukan. Dan berikut jenis-jenis website yang dikelompokan atas beberapa dasar:
1. Jenis Website berdasarkan sifat;
	a. Website Statis, merupakan web yang kontenya hampir jarang diubah
	b. Website Dinamis, Web yang konten atau isinya dapat berubah-ubah setiap saat
2. Jenis Website yang dikelompokkan berdasarkan Bahasa Pemrogramannya;
	a. Server side, Website yang memakai bahasa pemrograman yang tergantung
		dengan servernya
	b. Client side, adalah web yang tidak perluu server untuk menjalankannya.
	   Cukup diakses dengan browser.
3. Jenis-jenis Web menurut tujuannya;
	a. Web personal, biasanya web ini merupakan web yang berisi informasi
	   seorang
	b. Corporate Web, website yang dimiliki sebuah institusi atau
	   perusahaan.
	c. Web Portal, Web ini berisi banyak layanan, seperti berita, email dan jasa
	d. Web Forum, sebuah web yang dibuat sebagai sarana diskusi.
	
Keuntungan penggunaan web diantaranya yaitu :
a. Informasi dapat diberikan segera(tepat waktu) dan diperbarui secara berkala.
b. Presentasi fleksible dan visibilitas dapat menyediakan ragam isyarat untuk diseminasi informasi.
c. Informasi dapat diorganisir melalui tautan dan menu, berbagai tingkatan informasi dapat disediakan format file yang berbeda dapat digunakan untukj informasi yang dapat diunduh. Integrasi informasi dapat dilakukann melalui tautan dan seksi lain, halaman lain, atau web lain.
d. Tauta  dan menu dapat menyediakan informasi bagi pemangku kepentingan yang berbeda, informasi dapat pula diberikan melalui daftar email kepada pemangku kepentingan.
e. Setiap orang yang dapat mengakses web dapat memperoleh informasi karena keterjangkauan global dan potensi komunikasi masl dari web.
	   
	   
\end{document}