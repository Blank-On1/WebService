\documentclass[12pt, a4paper]{article}
\linespread{1.5}

\begin{document}
\title{World Wide Web}
\maketitle

\begin{itemize}
\item
	Imron Sumadireja (1164076) \\
	Jesron Marudut (1164077) \\
	Lusia Violita Aprilian (1164080) \\
	Mhd. Zulfikar Akram Nst. (1164081) \\
\end{itemize}

\section{Pengertian Website}
World wide web (www atau web) merupakan halaman situs informasi yang dapat diakses secara cepat atau sarana
antar muka informasi di internet. Web dapat menggabungkan teks, grafik, dan multimedia. Web memudahkan
penggunanya untuk mengakses informasi melalui konsep hypertext sehingga memungkinkan  suatu text untuk
menjadi acuan membuka dokumen laindo. Informasi dapat mudah disebar dan diakses.

\section{Sejarah Website}
Sementara itu World wide web (www) dikembangkan pertama kali oleh Tim Berners-Lee pada tahun 1989. Pada
awalnya, Tim mengusulkan WWW sebagai suatu cara berbagai dokumen diantara para peneliti. Dokumen online dapat
diakses melalui alamat unik yang disebut Universal Resource Locator atau URL. Selain itu WWW tidak hanya
dikembangkan untuk keperluan para peneliti, namun juga dikembangkan untuk kalangan pendidikan, bisnis dan
perorangan. Berdasarkan penjelasan singkat diatas dapat disimpulkan bahwa antara web dan internet memiliki
hubungan yang sangat erat walaupun keduangnay tidak bisa dikatakan sama. Web merupakan bagian dari layanan
yang dapat berjala di atas teknologi internet.

\section{Jenis-jenis website}
Website dikelompokan dalam beberapa jenis-jenis Website agar dapat memudahkan dalam menentukan jenis website
yang akan ditentukan. Dan berikut jenis-jenis website yang dikelompokan atas beberapa dasar:
1. Jenis Website berdasarkan sifat;
	a. Website Statis, merupakan web yang kontenya hampir jarang diubah
	b. Website Dinamis, Web yang konten atau isinya dapat berubah-ubah setiap saat
2. Jenis Website yang dikelompokkan berdasarkan Bahasa Pemrogramannya;
	a. Server side, Website yang memakai bahasa pemrograman yang tergantung
		dengan servernya
	b. Client side, adalah web yang tidak perluu server untuk menjalankannya.
	   Cukup diakses dengan browser.
3. Jenis-jenis Web menurut tujuannya;
	a. Web personal, biasanya web ini merupakan web yang berisi informasi
	   seorang
	b. Corporate Web, website yang dimiliki sebuah institusi atau
	   perusahaan.
	c. Web Portal, Web ini berisi banyak layanan, seperti berita, email dan jasa
	d. Web Forum, sebuah web yang dibuat sebagai sarana diskusi.
	
\section{Keuntungan Web}
Keuntungan penggunaan web diantaranya yaitu :
a. Informasi dapat diberikan segera(tepat waktu) dan diperbarui secara berkala.
b. Presentasi fleksible dan visibilitas dapat menyediakan ragam isyarat untuk diseminasi informasi.
c. Informasi dapat diorganisir melalui tautan dan menu, berbagai tingkatan informasi dapat disediakan format
file yang berbeda dapat digunakan untukj informasi yang dapat diunduh. Integrasi informasi dapat dilakukann
melalui tautan dan seksi lain, halaman lain, atau web lain.
d. Tauta  dan menu dapat menyediakan informasi bagi pemangku kepentingan yang berbeda, informasi dapat pula
diberikan melalui daftar email kepada pemangku kepentingan.
e. Setiap orang yang dapat mengakses web dapat memperoleh informasi karena keterjangkauan global dan potensi
komunikasi masl dari web.
	   
\section{tentang web scraping}
web scraping atau scraping web (dapat disebut juga panen web atau web ekstraksi data) merupakan sebuah
perangkat lunak komputer teknik penggalian informasi dari situs web seperti mengambil mengambil data
berbentuk teks yang umumnya bertipe HTML atau XHTML. contohnya seperti Internet Explorer (IE) dan Mozilla Web
Browser. web scraping berkaitan erat dengan pengindekan web.

\section{manfaat dari web scraping}
Web scraping sering dikenal dengan screen scraping. Web scraping tidak dapat dimasukkan kedalam bidang data
mining karena dalam data mining menyiratkan upaya untuk memahami pola semantik dari sejumlah data besar yang
telah diperoleh. Aplikasi Web scraping hanya fokus pada cara memperoleh data melalui pengambilan dan ekstrasi
dengan ukuran data yang bervariasi. Manfaat dari web scraping adalah agar informasi yang diambil lebih
terfokus sehingga dapat memudahkan dalam melakukan pencarian sesuatu, adapun cara untuk mengembangkan teknik
web scraping yaitu dengan cara sebagai berikut:
	a. Pengembang/pembuat program mempelajari dokumen HTML dari website yang akan diambil informasinya untuk
	   di tag HTML tujuannya yakni untuk mengapit informasi yang akan diambil (Create Scraping Template)
	b. Pengembang/pembuat program mempelajari teknik navigasi pada website yang akan diambil informasinya
	   untuk ditiru pada aplikasi web scraping yang akan dibuat (Explore Site Navigation)
	c. Selanjutnya aplikasi web scraping akan mengotomisasi informasi yang didapatkan dari website yang telah
	   ditentukan (Automate Navigation and Extraction), informasi yang didapat tersebut akan disimpan dalam 
	   tabel basis data (Extracted Data and Package History).

\section{Perbandingan Metode Web Scraping}
Berikut perbandingan antara metode Web Scraping menggunakan CSS Selector dan Xpath Selector
	1.	Penggunaan metode XPATH Selector untuk web scraping menghasilkan artikel yang lebih lengkap
	    dibandingkan dengan
		menggunakan metode CSS Selector, Ditunjukkan dengan jumlah item dan ukuran file yang didapatkan lebih
		besar.
		Namun juga menyisakan proses lain untuk menghilangkan kode HTML yang tidak diinginkan dari artikel
		yang dihasilkan menggunakan metode XPATH Selector.
	2.	Dalam penggunaan memori baik metode XPATH Selector dan CSS Selector tidak memiliki perbedaan yang
	    signifikan
		(cenderung sama). Disebabkan karena engine scrapy yang baik dalam penggunaan resource-nya. 
	3.	Metode XPATH Selector memiliki waktu proses yang lebih cepat daripada menggunakan metode CSS
	    Selector.
	4.	Pada metode XPATH, selector cukup mengikuti node pada halaman web, sehingga waktu yang dibutuhkan
		relatif lebih singkat.

\section{Tentang Web Hosting}
Web hosting merupakan jasa penyewaaan tempat penyimpanan data di internet atau biasa disebut dengan cloud
yang diperlukan oleh sebuah website. Web hosting ialah salah satu syarat agar website bisa diakses secara 
online dan dapat diakses dari seluruh dunia. Ukuran yang digunakan dalam suatu web hosting adalah kapasitas
dan bandwidth. 

Kapasistas merupakan ukuran besarnya kemampuan sebuah web hosting untuk menyimpan data-data di internet.

Bandwidth merupakan ukuran maksimal dari jumlah volume data yang diperbolehkan untuk diakses dari web hosting
setiap bulannya. Sebagai contoh, sebuah halaman website yang mempunyai ukuran 2 MB dan bandwidth web hosting
2000 MB, maka setiap bulannya website tersebut dapat diakses sebanyak 2000 kali.

\section{Tentang Domain}
Domain merupakan sebuah alamat di dunia internet atau sebuah identitas dari sebuah website. Domain digunakan untuk mempermudah dalam mengakses situs yang ada di internet. Domain terbagi menjadi 2 jenis domain yang dibagi berdasarkan pemisahaan titiknya, yaitu; Top Level Domain (TLD) dan Second Level Domain (SLD). Top Level Domain merupakan bagian terakhir dalam sebuah domain website. Contohnya "facebook.com" dan disitu yang jadi domainnya adalah ".com". Selanjutnya Second Level Domain atau SLD merupakan bagian dari domain yang terdapat sebelum Top Level Domain. Contohnya "Facebook.com" yang menjadi SLDnya adalah Facebook. Jadi SLD adalah unsur domain yang didaftarkan terdahulu pada jasa Web Hosting. Dan ada juga yang disebut Country Code Second Level Domain (ccSLD). Berguna sebagai penunjuk organisasi apa yang mendaftar pada suatu domain. Setiap negara juga mempunyai ccSLD yang berbeda-beda tiap negaranya.

\section{Tentang Hubungan Domain dan Web Hosting}
Hubungan Domain dan Web Hosting merupakan satu kesatuan yang saling membutuhkan.
Pada sebuah Website, domain dan web hosting saling ketergantungan. Apabila yang tersedia hanya web hosting, maka website tidak akan dapat diakses. Begitu juga dengan domain, apabila yang tersedia hanya domain, maka tidak akan ada website yang akan ditampilkan, karena halaman website tersimpan didalam web hosting.

\section{Macam-macam Web Hosting}
Saat ini banyak jasa penyedia hosting dengan harga relatif murah bahkan gratis.
Berikut adalah macam-macam web hosting :
	1.	Free Hosting / Web Hosting Gratis
		Dengan free hosting, kita dengan mudah mencari layanan web hosting dan domain gratis di internet dengan
		menggunakan fasilitas search engine seperti google atau yang lainnya. Biasanya penyedia web hosting tidak
		mengenakan biaya. Namun, memiliki banyak keterbatasan beberapa fitur.
	2.	Web Hosting Berbagi atau Shared Hosting
		Jenis hosting ini paling sering digunakan karena bukan hanya murah, namun juga memiliki layanan yang dapat
		mencukupi segala kebutuhan. Biasanya, untuk menggunakan layanan web hosting ini anda hanya perlu untuk mengeluarkan
		biaya sebesar 100 hingga 200 ribu untuk mendapatkan ruang sebesar 2 GB – 7,5 GB dengan bandwith unlimited.
	3.	VPS Web Hosting
		VPS merupakan singkatan dari Virtual Private Server. Jadi disini anda dapat seperti memiliki server sendiri
		untuk situs anda. dengan server ini, anda akan memiliki control yang lebih dalam seperti Dedicated Server.
	4.	Dedicated Web Hosting
		Dedicated Web Hosting merupakan sebuah layanan hosting dengan server yang memiliki kemampuan untuk melakukan
		handle terhadap traffic dengan jumlah sangat banyak. Serta memiliki banyak fitur premium di dalamnya. Selain itu,
		juga memiliki control penuh terhadap server walaupun  hanya menyewanya.
	5.	Managed Web Hosting
		Managed Hosting / Web Hosting Terkelola ini adalah web hosting yang dikhususkan untuk situs dengan
		Platform yang sama. Managed web hosting biasanya lebih aman dan kinerjanya lebih optimal. Selain itu,
		memudahkan untuk melakukan beragam pengaturan, mulai dari installasi, sampai setting macam-macamnya.

\section{Cara Mendapatkan Web Hosting dan Domain}
	Web hosting dan domain lebih sering ditemukan oleh perusahaan-perusahaan penyedia web hosting atau domain. Untuk menemukannya, cukup cari di google. Maka akan banyak perusahaan yang menyediakan jasa web hosting. Kinerja web hosting berbeda-beda dari setiap perusahaan Web hosting. Karna apabila web hosting yang dibuat oleh jasa tersebut buruk maka website tersebut akan mudah bermasalah. Oleh karena itu dalam pembelian jasa domain atau web hosting perlu diperhatikan hal-hal seperti profil dari penjual web hosting, fitur untuk website dan harga dari web hosting tersebut. Dalam memilih penyedia web hosting, pastikan penyedia mempunyai reputasi yang bagus dan terpercaya. Dan sebaiknya memilih perusahaan web hosting yang sudah dalam bentuk perusahaan CV atau PT agar pertanggung jawabannya jelas ketika terjadi gangguan web hosting.

\end{document}