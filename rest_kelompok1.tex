\documentclass[12pt,a4paper]{article} 
\linespread{1.5}
\begin{document}
\title{REST WEB SERVICE}
\maketitle

\begin{itemize}
\item
Nama Kelompok 1\\
Farid Ariyanto Saputra 1164034\\
Nurgivani Syarifatul Husna 1164050\\
Velariza Alvioletta 1164056\\
Yogi Aditya Saputra 1164060 \\
\end{itemize}

\section{REST}
\subsection{Pengertian REST}
REST atau singkatan dari Representational State Transfer adalah salah satu model arsitektur web yang memiliki aturan berupa interface yang seragam sehingga jika diterapkan dalam web service akan meningkatkan dan memaksimalkan kinerja dalam web sevice, terutama dalam performa dan kemudahan dalam memodifikasi. Dalam arsitektur REST, data-data serta fungsinya dianggap sebagai sumber daya yang biasa di akses melalui URI, yang merupakan singkatan dari Uniform Resource Identifier yang biasanya berupa link pada web.
\subsection{Konsep Kerja REST}
REST merupakan salah satu jenis web service yang menerapkan konsep perpindahan antar state. Jika digambarkan state bisa dikatakan seperti, saat browser meminta suatu halaman web, lalu server akan mengirimkan state halaman web yang diminta kepada browser. (Tidwell, D., 2001) Begitu juga REST mempunyai konsep kerja, dengan bernavigasi dengan link-link HTTP untuk melakukan suatu aktivitas tertentu, seakan telah terjadi perpindahan state, dari state satu ke state lainnya. Perintah  HTTP yang biasanya digunakan adalah fungsi GET, POST, PUT dan DELETE. Respon yang akan dikirim berupa hasil dalam bentuk XML yang sederhana tanpa ada protokol pembagian paket data supaya informasi yang diterima lebih mudah dibaca dan tidak ada pemecahan data pada pengguna atau client.
\subsection{Arsitektur REST}
Arsitektur web service
Web Service memiliki tiga entitas dalam arsitekturnya yaitu, Service requester,service provider dan service registry. Masing-masing arsitektur memiliki fungsi yang berbeda-beda sebagai berikut :
a.	Service provider
Memiliki fungsi untuk menyediakan layanan/service dan mengolah registry agar dapat tersedianya layanan.
b.	Service Registry 
Memiliki fungsi sebagai lokasi central yang mendeskripsikan layanan yang telah di register.
c.	Service Requestor 
yang membutuhkan layanan mencari dan menemukan layanan yang dibutuhkan dan menggunakan layanan tersebut.

\subsection{Perintah Dalam REST}

\end{document}