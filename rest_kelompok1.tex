\documentclass[12pt,a4paper]{article} 
\linespread{1.5}
\begin{document}
\title{REST WEB SERVICE}
\maketitle

\begin{itemize}
\item
Nama Kelompok 1\\
Farid Ariyanto Saputra 1164034\\
Nurgivani Syarifatul Husna 1164050\\
Velariza Alvioletta 1164056\\
Yogi Aditya Saputra 1164060 \\
\end{itemize}

\section{REST}
\subsection{Pengertian REST}
REST atau singkatan dari Representational State Transfer adalah salah satu model arsitektur web yang memiliki aturan berupa interface yang seragam sehingga jika diterapkan dalam web service akan meningkatkan dan memaksimalkan kinerja dalam web sevice, terutama dalam performa dan kemudahan dalam memodifikasi. Dalam arsitektur REST, data-data serta fungsinya dianggap sebagai sumber daya yang biasa di akses melalui URI, yang merupakan singkatan dari Uniform Resource Identifier yang biasanya berupa link pada web.
\subsection{Konsep Kerja REST}

\subsection{Arsitektur REST}

\subsection{Perintah Dalam REST}

\end{document}