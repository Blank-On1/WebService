%KELOMPOK 4 Blank-On1
%\begin{enumerate}
%\item Andri Fajar Sunandhar
%\item Cokro Edi Prawiro
%\item Fadila
%\item Sandro Samuel Sinaga
%\end{enumerate}

\section{Common Gateway Interface}
CGI merupakan metode yang dipakai untuk mempertukarkan data di antara server dan klien (browser). CGI merupakan sebuah standar dimana program atau script bisa mengirim data kembali ke web server dimana ia diproses, yaitu dengan menggunakan tag HTML standar untuk mendapatkan data dari seseorang, kemudian meneruskannya ke CGI. Selanjutnya CGI melakukan serangkaian aksi terkait data tersebut\cite{prihatmoko2013pengembangan}.

\section{Common Gateway Interface}
Salah satu kekuatan utama yang memungkinkan developer web membangun aplikasi-aplikasi web yang dinamis adalah kemampuan server web untuk mengakses sistem database. Untuk keperluan pengembangan aplikasi web yang dinamis, pertama kali diperkenalkan Common Gateway Interface (CGI). CGI adalah bagian dari server web yang dapat berkomunikasi dengan program lain di luar server web. CGI memungkinkan server web memanggil suatu program, lalu mengirimkan data-data spesifik dari pengguna ke program tersebut. Hasil proses tadi diterima oleh CGI yang selanjutnya menyerahkannya kepada server web untuk kemudian, yang pada gilirannya akan mengirimkan informasi tersebut kembali dalam bentuk HTML ke browser web pengguna.


