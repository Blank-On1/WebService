%KELOMPOK 4 Blank-On1
%\begin{enumerate}
%\item Andri Fajar Sunandhar
%\item Cokro Edi Prawiro
%\item Fadila
%\item Sandro Samuel Sinaga
%\end{enumerate}

\section{Common Gateway Interface}
CGI merupakan metode yang dipakai untuk mempertukarkan data di antara server dan klien (browser). CGI merupakan sebuah standar dimana program atau script bisa mengirim data kembali ke web server dimana ia diproses, yaitu dengan menggunakan tag HTML standar untuk mendapatkan data dari seseorang, kemudian meneruskannya ke CGI. Selanjutnya CGI melakukan serangkaian aksi terkait data tersebut\cite{prihatmoko2013pengembangan}.


\par Adapun pengertian lain dari Common Gateway Interface yaitu sekumpulan aturan untuk mengarahkan sebuah server web berkomunikasi dengan software dalam mesin yang sama begitu pula sebaliknya antara software CGI programs dengan web server. Setiap perangkat lunak dapat menjadi perogram CGI dengan syarat software tersebut dapat melakukan input dan output sesuai setandar CGI. CGI menjadi setandar menghubungkan untuk menghubungkan data informasi yang terjadi antara server dan aplikasi, seperti HTTP. Script CGI dapat mengirtimkan data kembali ke web server  dimana CGI diperoses. CGI merupakan interface antara halaman website dengan web server yang menjalankan perogram\cite{aditya2015analisis}.


\par Salah satu kekuatan utama yang memungkinkan developer web membangun aplikasi-aplikasi web yang dinamis adalah kemampuan server web untuk mengakses sistem database. Untuk keperluan pengembangan aplikasi web yang dinamis, pertama kali diperkenalkan Common Gateway Interface (CGI). CGI adalah bagian dari server web yang dapat berkomunikasi dengan program lain di luar server web. CGI memungkinkan server web memanggil suatu program, lalu mengirimkan data-data spesifik dari pengguna ke program tersebut. Hasil proses tadi diterima oleh CGI yang selanjutnya menyerahkannya kepada server web untuk kemudian, yang pada gilirannya akan mengirimkan informasi tersebut kembali dalam bentuk HTML ke browser web pengguna\cite{ibrahim2011sistem}.

\section{PHP and Common Gateway Interface interconnections }
Common Gateway Interface is a standard that is used to connect various application programs to web pages. One example of the programming language is PHP. PHP is a software that is open source and can pass across the various platform. Php can be run in 2 ways ie as apache module in web server and also as binary in Common Gateway Interface.This language was created in 1994 by Ramus Lerdoff.  Initially, PHP is a CGI program that is devoted to receiving input through forms displayed in web pages or browser. The PHP code is usually processed by a PHP interpreter which is usually executed as a native web server module or Common Gateway Interface\cite{nahado2015bumbu}.


