\begin{itemize}
\item Ahmad Syafrizal Huda (1164062)
\item Annisa Fathoroni (1164067)
\item Puad Hamdani (1164084)
\item Rahmi Roza (1164085)
\item Tasya Wiendhyra (1164086)
\end{itemize}

\section{Definisi RESTful Web Service}
REST merupakan salah satu macam web service yang memasukkan konsep perpindahan antar state. State disini bisa dibayangkan seperti jika browser meminta suatu halaman web, maka server akan mengirimkan state halaman web yang sekarang ke browser. Menurut salah satu perkembangan Tidwell, D., 2001 bernavigasi melalui link-link yang disediakan sama halnya dengan mengganti state dari halaman web. Begitu pula REST bekerja, dengan bernavigasi melalui link-link HTTP untuk melakukan aktivitas tertentu, seakan-akan terjadi perpindahan state satu sama lain \cite{indrawan2017implementasi}.

\section{Prinsip Pada RESTful}
RESTful adalah salah satu teknologi web service untuk membuat suatu sistem yang terdistribusi dimana cara kerjanya berdasarkan resource. RESTful sendiri merupakan software yang didesain untuk penekanan pada skalabilitas,kesederhanaan dan kegunaan. Metode dalam REST terdiri dari empat prinsip utama teknologi, yaitu \cite{aji2016penerapan}:
\begin{enumerate}
\item Resource identifier melalui Uniform Resource Identifier (URI), REST Web service mencari sekumpulan sumber daya yang mengidentifikasi interaksi antar klien.
\item Uniform interface, sumber daya yang dimanipulasi CRUD (Create, Read, Update, Delete) menggunakan operasi PUT, GET, POST, dan DELETE.
\item Self-descriptive messages, sumberdaya informasi tidak terikat, sehingga dapat mengakses berbagai format konten (HTML, XML, PDF, JPEG, Plain Text dan lainnya). Metadata pun dapat digunakan.
\item Stateful interactions melalui hyperlinks, setiap interaksi dengan suatu sumber daya bersifat stateless, yaitu request messages tergantung jenis kontennya.
\end{enumerate}

\section{Contoh RESTful}
Implementasi RESTful Web Service untuk Sistem Penghitungan Suara Secara Cepat pada Pilkada.
Metode ini yang digunakan oleh penyelenggara pemilihan umum untuk menentukan hasil pilkada. Dengan memanfaatkan teknologi yang ada, proses pengumpulan data hasil perolehan suara bisa dilakukan dengan lebih cepat. Salah satu metode baru yang bisa digunakan untuk melakukan proses tersebut adalah metode perhitungan cepat riil. Metode ini memanfaatkan teknologi informasi dan komunikasi untuk melakukan proses penghitungan suara. Real-quick count mengambil hasil perhitungan dari semua tempat pemungutan suara (TPS). Tetapi hasil tersebut dikirim langsung dari TPS ke lembaga penyedia informasi hasil perhitungan cepat, tidak melalui prosedur seperti pada real count yang mengharuskan pengumpulan data berjenjang, oleh karena itu waktu yang dibutuhkan untuk memperoleh semua hasil suara bisa dioptimalkan. Pada jurnal ini dilakukan perbandingan antara SOAP dan REST pada aplikasi mobile dan multimedia conference. Hasil penelitian yang dilakukan pada aplikasi mobile computing menunjukkan bahwa ukuran pesan pada RESTful web service mencapai 9 sampai 10 kali lebih kecil dibandingkan ukuran pesan dari web service berbasis SOAP \cite{rofiq2017implementasi}.

\section{Contoh Implementasi RESTful untuk Sales Order dan Sales Tracking Berbasis Mobile}
Bagian penjualan merupakan bagian yang paling penting dalam penjualan produk. Perusahaan membutuhkan sistem yang dapat membantu aktivitas dan pemesanan produk. dengan membuat sebuah Controller terlebih dahulu,yang berperan untuk menentukan informasi apa yang akan disampaikan pada saat client mengakses web service. Dibuat dengan arsitektur REST dengan menggunakan method yang di dukung protokol HTTP seperi method DELETE, UPDATE, CREATE,dll. Aplikasi mobile ini akan menggunakan data dari GPS untuk memastikan lokasi penjual juga dilengkapi barcode untuk mempercepat input data barang.
\cite{kurniawan2015implementasi} .

\section{IMPLEMENTASI REST WEB SERVICE PADA APLIKASI PENGOLAH PESAN YAHOO MESSENGER PADA CV. MELIANA PRATAMA}
Mengimplementasikan REST Web Service pada aplikasi pengolah pesan Yahoo Messenger (YM). Aplikasi REST Web Service dapat dijadikan sebagai miidleware antara aplikasi pengolahan pesan Yahoo Messenger (YM) dengan database, sehingga proses transaksi ke database menjadil lebih efisien. Hal ini dikarenakan aplikasi client tidak perlu mengetahui database apa yang digunakan oleh server
\cite{ikrom2015implementasi}