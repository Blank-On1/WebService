\documentclass[12pt,a4paper]{article} 
\linespread{1.5}
\begin{document}
\title{Instalasi PIP dan Contoh Penggunaan}
\maketitle

\begin{itemize}
\item
Nama Kelompok 1\\
Farid Ariyanto Saputra 1164036\\
Nurgivani Syarifatul Husna 1164050\\
Velariza Alvioletta 1164056\\
Yogi Aditya Saputra 1164060 \\
\end{itemize}

\section{Python}
\subsection{Pengertian Python}
Python merupakan salah satu Bahasa pemrograman yang bersifat open source yang tertafsir oleh typing yang dinamis dan kuat. Python juga memiliki banyak library, seperti struktur data, files, dan jaringan. Bahasa pemrograman python juga banyak digunakan untuk berbagai keperluan, contohnya komputasi ilmiah, system administrasi, dan pengembangan web. Selain itu pula, keuntungan Bahasa pemrograman python yakni memiliki alat simulasi python gratis.
\subsection{Pengertian Python}
Python adalah suatu bahasa pemrograman yang bisa dikatakan bahasa pemrograman jaman sekarang, karena usianya sangat muda namun sudah banyak digunakan oleh programmer. Phyton dapat mendukung dalam membangun aplikasi berbasis desktop, web, mobile maupun lainnya. Untuk membangun sebuah aplikasi, bahasa pemrograman ini juga bisa digunakan menggunakan framework maupun tanpa framework. Namun, apabila tidak menggunakan framework akan membutuhkan waktu yang lama dalam tahap membangun aplikasi, begitu juga sebaliknya apabila menggunakan framework pembangunan aplikasi akan menjadi lebih cepat dan terstruktur, biasanya framework yang digunakan adalah Django, dimana disana terlah tersedia komponen seperti models, templates, views, forms, dan admin interface.
\subsection{Pengertian Python}
Python merupakan sebuah bahasa dalam pemrograman yang dibuat oleh Guido Van Rossum dan populer sebagai sebuah bahasa pemrograman berbasis Web. Python dikenal sebagai sebuah bahasa yang menggabungkan kapabilitas, kemahiran, dengan sintaksis kode yang jelas. Mengambil dari pengertian wikipedia, Python merupakan sebuah bahasa pemrograman interpretatif yang bisa digunakan dalam berbagai macam program web dengan filosofi perancangan yang berfokus ada tingkat keterbacaan kode.
\subsection{Pengertian Python}
Phyton merupakan salah satu Bahasa pemrograman kelas atas serta memiliki sifat intrepeter, object oriented, serta interaktif serta dapat berjalan pada sistem operasi seperti UNIX, MAC, Windows maupun platfrom lain. Karena Python merupakan bahasa pemrograman kelas tinggi, python dapat di kombinasikan dalam penggunaan tata kalimat dengan modul-modul yang telah siap pakai serta struktur data yang lebih efisien.

\section{PIP}
\subsection{Pengertian PIP}
PIP yang memiliki kepanjangan dari Pyhton Index Packaging. PIP itu sendiri adalah sebuah app store atau biasa disebut package manager yang biasa digunakan untuk mencari, mengunduh, menginstal serta mengelola package atau modules yang biasa ditemukan di PyPI ( Pyhton Package Index ). Dimana PyPI adalah sebuah library perangkat lunak untuk Bahasa pemrograman Pyhton.
\subsection{Pengertian PIP}
PIP merupakan singkatan dari python index packaging. PIP adalah sebuah aplikasi manajemen package yang biasa digunakan untuk menginstall dan mengelola package yang telah ditulis oleh python. Untuk menemukan packagenya, kita bisa mencari di situs Python Package Index (PyPI). Ada kurang lebih 134443 package dalam python yang bisa diinstall melalui PyPI.
\subsection{Pengertian PIP}
PIP (python index packaging) merupakan Package Management System yang biasanya digunakan untuk mengunduh dan mengelola package Python. Banyak sekali package yang bisa di temukan di PyPI. 
PIP bisa langsung digunakan di Python versi 2.7.9 dan versi 3 namun apabila menggunakan versi dibawahnya harus melakukan instalasi terlebih dahulu. 
Pip juga sebuah sistem untuk memeriksa perilaku sistem terdistribusi secara otomatis terhadap harapan programmer tentang sistem. Pip mengklasifikasikan perilaku sistem valid atau tidak valid, mengelompokkan perilaku ke dalam set yang dapat dipikirkan, dan menyajikan perilaku keseluruhan dalam beberapa bentuk yang sesuai untuk menemukan atau memverifikasi kebenaran perilaku sistem.
\subsection{Pengertian PIP}
PIP merupakan singkatan dari python index packaging yang merupakan package management sistem yang sering kali di gunakan untuk mengelola package python. Packages python dipasang dengan manajer paket pip, yang termasuk dalam semua lingkungan virtual. seperti sesi prompt perintah python akan memanggil versi alat ini daripada milik virtual enviroment yang diaktifkan.

\section{Cara Instalasi PIP}
\subsection{Instalasi di Windows}
Berikut adalah cara menginstall PIP atau Python Index Packaging. \\
1.	Kunjungi web resmi https://pip.pypa.io/en/stable/installing/ untuk mendownload dan melihat cara instalasi di windows.\\
2.	Download get-pip.py dari web tersebut. \\
3.	Buka Python.exe atau buka CMD dan ketikan perintah python get-pip.py di folder yang ada file get-pip.py yang sudah di 	download tadi.\\
4.	Kemudian set PATH dalam environmental variable ke tempat pip.exe. \\
5.	Untuk mengecek instalasi pip, ketikan perintah pip id di CMD. \\
\subsection{Instalasi PIP di Mac}
Disini saya akan memberitahu cara install PIP untuk mac x, caranya adalah : \\
1.	Download python terlebih dahulu pada website resminya \\
2.	Lalu akan mendapatkan file berbentuk .pkg  dari proses download yang telah dilakukan \\
3.	Selanjutnya lakukan doubleclik file tersebut dan otomatis akan memandu untuk melakukan installasi, ikuti langkah tersebut dengan menekan next pada setiap langkah-langkahnya \\
4.	Buka terminal dan tulis “python3 –version” \\
5.	Lalu, “sudo easy install pip” \\
6.	Dan, installasi selesai. \\
\subsection{Instalasi PIP di Linux}
Pada Sistem Operasi Linux, proses instalasi Pyhton tidak bermain dengan skrip getpip.py seperti di windows dan mac. Berikut adalah tahapan instalasinya.
\begin{itemize}
\item Melakukan memperbaharui daftar paket dan perangkat lunak sistem.
\item Proses Instalasi PIP di linux \\
Proses instalasi PIP di linux sangat sederhana karena hanya melakukan command satu perintah di Terminal.
\item Verifikasi PIP \\
Proses ini untuk menverifikasi apakah pip dan semua dependensi sudah terinstal atau belum agar dapat berjalan dengan optimal.
\subsection{Instalasi PIP di Linux}
Pada umumnya perangkat python merupakan perangkat lunak yang termasuk di dalam disribusi Linux. Untuk linux distribusi slackware digunakan Python versi 2.4, yang terdapat pada CD I direktori/slackware/d. Menggunakan Toolkit untuk melakukan installasi paket di slackware adalah installpkg, berikut langkah instalasinya.\\
\item  mount/mnt/cdrom\\
\item  cd/mnt/cdrom/slackware/d\\
\item  installpkg python-2.4.1-i486-1tgz\\
\\
Python adalah menyediakan modus interaktif yang sangat berguna dalam melakukan latihan dan tes kode. Untuk menulis kode dalam modus interaktif dilakukan dengan memanggil toolkit python pada shell Linux. \\
\item  python\\
\item Python 2.4.1 (1, Apr 10 2005, 22:30:36) \\
\item GCC 3.3.5 on linux2 \\
\item Type "help", "copyright", "credits" or "license" \\
\item for more information.\\
\item  >>>\\
 Tanda “>>>” merupakan suatu prompt dalam modus interaktif Python, selanjutnya Python siap menerima input kode yang dimasukkan.\\

\end{itemize}
\end{document}