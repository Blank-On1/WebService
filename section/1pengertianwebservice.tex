%Resume tentang Pengertian Web Service

%Kelompok 2 D4 TI / 2B

%Alwan Suryansah				1164033 
%Dinda Ayu Pratiwi				1164034
%Kurnia Sandi					1164042
%Teduh Sanubari					1164054
%Wildan Khaustara Wijaksana		1164058


\section{Definisi}

\subsection{Hartati Deviana}

	Web service adalah suatu komponen perangkat lunak self-containing dan aplikasi modular self-describing yang dapat disiarkan, dialokasikan, dan dijalankan di dalam web. Web service adalah teknologi yang mentransformasikan kemampuan internet dengan cara menambahkan beberapa kemampuan seperti kemampuan transactional web. Apa itu Transactional Web? Transactional Web yaitu kemampuan web dalam hal saling berinteraksi dengan pola program-to-program (P2P). Fokus web selama ini didominasi oleh komunikasi program-to-user dengan interaksi business-to-consumer (B2C), sedangkan transactional web akan didominasi oleh P2P dengan interaksi business-to-business (B2B). \cite{deviana2013penerapan}.

\subsection{Richards Robert}

	Web service merupakan salah satu implementasi dari teknologi XML (Extensible Markup Language) pada proses pertukaran antara (data exchange) platform yang berbeda sercara berbeda.

\textit{"A Web service is a software system designed to support interoperable machine-to-machine interaction over a network. It has an interface described in a machine-processable format(specifically WSDL).Other systems interact with the Web service in a manner prescribed by its description using SOAP messages, typically conveyed using HTTP with an XML seriali zation in conjunction with other Web-related standards"}.

Menurut Richards, web service dapat digunakan untuk berkomunikasi antara mesin satu dengan mesin yang lain melalui interface perantara yang umumnya berupa WSDL(Web Service Definition Language), layanan ini biasa bekerja pada protokol HTTP dengan bentuk response dan request berupa SOAP messange. SOAP (Simple Object Access Protocol) adalah standar untuk bertukar pesan-pesan berbasis XML melalui jaringan komputer atau sebuah jalan untuk program yang berjalan pada suatu sistem operasi (OS) untuk berkomunikasi dengan program pada OS yang sama maupun berbeda dengan menggunakan HTTP dan XML sebagai mekanisme untuk pertukaran data. Format SOAP message adalah mengikuti frame XML yang terstandarisasi \cite{ihya2011pembuatan}. 

\subsection{Chen, Xi dan Zheng, Zibin dan Yu, Qi dan Lyu, Michael R}

	Web Service adalah komponen perangkat lunak yang terintegrasi untuk mendukung interaksi antar mesin dengan mesin yang lainya ( komputer ) antar jaringan , layanan web service telah banyak digunakan untuk membangun suatu aplikasi yang berorientasi dengan layanan industri dan akademisi dalam beberapa tahun trakhir , jumlah layanan web yang tersedia untuk umum terus meningkat di internet , Namun  ini menyulitkan pengguna untuk memilih layanan yang tepat di antara banyaknya layanan web services\cite{chen2014web}.

\subsection{Witono, Timotius and Susanto, Raphael}

	Pengertian sederhana web service adalah aplikasi yang dibuat agar dapat dipanggil atau diakses oleh aplikasi lain melalui internet atau intranet dengan menggunakan XML sebagai format pengiriman pesan. Web service digunakan saat pengguna akan mentransformasi sebuah logik atau sebuah class dan objek yang terpisah dalam satu ruang lingkup yang menjadi satu, sehingga tingkat keamanan dapat ditangani dengan baik\cite{witono201511}.

\subsection{Kurniawan, Erick}

	Web Service adalah layanan yang tersedia di Internet. Web Service menggunakan format standar XML untuk pengiriman pesannya. Web Services juga tidak terikat kepada bahasa pemrograman atau sistem operasi tertentu (Ethan Cerami, 2002). Web Services adalah antar muka yang mendeskripsikan koleksi yang dapat diakses dalam jaringan menggunakan format standar XML untuk pertukaran pesan. Web Services mengerjakan tugas yang spesifik. Web Services dideskripsikan menggunakan format standar notasi XML yang disebut services description (Gottschalk, 2002)\cite{chen2014web}.

\subsection{Sarbini, Riska Nurtantyo}

	Web service merupakan satuan diskrit dari fungsionalitas programatis yang diekspos 
kepada client via protokol komunikasi, dan format data standar bernama HTTP dan 
XML. Protokol ini mengatasi masalah komunikasi lintas internet dan lintas 
firewall tanpa beralih ke solusi superior yang memerlukan port-port komunikasi 
tambahan yang harus dibuka untuk akses eksternal. Dikarenakan web service mamiliki fungsi untuk menformat dan menguraikan pesan XML\cite{sarbini2015pengembangan}. 

\subsection{M. Shalahuddin dan Rosa A.S.}

	Web Service merupakan suatu sistem yang menyediakan pelayanan yang dibutuhkan oleh klien. Klien dari web service tidak hanya berupa aplikasi web, tetapi juga bisa sebuah aplikasi enterprise. Jadi web service tidak sama dengan web server, bahkan sebuah aplikasi web pada web server dapat menjadi klien dari web service\cite{inayah2014aplikasi}.

\subsection{Gottschalk (2002)}

	Web Service adalah teknologi yang mengubah kemampuan internet dengan menambahkan kemampuan transactional web, yaitu kemampuan web untuk saling komunikasi dengan pola program to program (P2P). Fokus web selama ini didominasi oleh komunikasi program to user dengan interaksi business to costumer (B2C), sedangkan stransactional web akan didominasi oleh P2P dengan interaksi business to business\cite{fauziah2014aplikasi}.


\subsection{Slameto, Andika Agus}

	Web service adalah suatu sistem perangkat lunak yang dirancang untuk mendukung interoperabilitas dan interaksi antar sistem pada suatu jaringan. Web service digunakan sebagai suatu fasilitas yang disediakan oleh suatu web site untuk menyediakan layanan (dalam bentuk informasi) kepada sistem lain, sehingga sistem lain dapat berinteraksi dengan sistem tersebut melalui layanan-layanan (service)yang disediakan oleh suatu sistem yang menyediakan web service. Web service menyimpan data informasi dalam format XML, sehingga data ini dapat diakses oleh sistem lain walaupun berbeda platform, sistem operasi, maupun bahasa compiler\cite{slameto2015penerapan}.

\subsection{Jurnal Masyarakat Informatika}

	Web service adalah antarmuka yang mendeskripsikan sekumpulan operasi yang dapat diakses dalam sebuah jaringan melalui pesan XML yang telah distandartkan.xml iyalah bahasa markup yang sudah terintregrasi dengan web service. W3C mendefinisikan web service sebagai sebuah sistem perangkat lunak yang dirancang untuk mendukung inter operasi mesin ke mesin di sebuah jaringan.  Web service merupakan komponen perangkat lunak loosely coupled, dapat diguna ulang, membungkus fungsionalitas diskret, didistribusikan, dan diakses secara programatik melalui protokol internet standart . dan sangat di di perhatikan di bidang informatika \cite{saputra2integrasi}.

\subsection{Jurnal Sistem dan Teknologi Informasi}

	Web service menurut World Wide Web Consortium (W3C) (2004), organisasi yang mengembangkan standar-standar dalam dunia web, mendefinisikan web service sebagai "\textit{“a software system designed to support interoperable machine-to-machine interaction over a network. It has an interface described in a machine-processable format (specifically WSDL). Other systems interact with the Web service in a manner prescribed by its description using SOAP messages, typically conveyed using HTTP with an XML serialization in conjunction with other Web-related standards.” }"(Lucky,2008).

Berdasarkan definisi dari W3C dapat disimpulkan bahwa web service merupakan aplikasi yang dibuat agar dapat dipanggil atau diakses oleh aplikasi lain melalui internet maupun intranet dengan menggunakan XML sebagai format pengiriman pesan\cite{prasetya2013perancangan}.

\subsection{Pengertian Web Service menurut Hartono, Fajar Fani and Hendry, H and Somya, Ramos}

	Web Service dapat diartikan sebuah antar muka atau dalam bahas inggris yaitu interface  yang berarti menggambarkan sebuah sekumpulan operasi-operasi yang kemudian dapat diakses melalui jaringan, misalnya internet dalam bentuk pesan “Extensible Markup Language (XML)”. Web Service juga menyediakan standar komunikasi dalam berbagai software yang berbeda-beda, dan dapat berjalan di berbagai platform maupun framework\cite{hartono2013aplikasi}.

\subsection{Pengertian Web Service Menurut Kasaedja, Bramwell A and Sengkey, Rizal and Lantang, Oktavian A}

	O’Reilly menerbitkan sebuah buku, David A Chappel dan Tyler Jewell sebagai penulis mengartikan bahwa web service adalah suatu kumpulan logika bisnis dalam internet yang dapat di akses melalui protocol internet. Dalam buku tersebut juga dijelaskan bahwa terdapat tiga komponen teknologi dalam Web service yaitu, Simple Object Acces Protocol (SOAP), Web Service Description Language (WSDL), dan Universal Description, Discoveri, Integration (UDDI)\cite{kasaedja2014rancang}.

\subsection{Hamdani, Hamdani and Haviluddin, Haviluddin and Darmawangsa, Ngurah Satria}

	Web service diartikan sebagai sebuah antar muka (interface) yang menggambarkan sekumpulan operasi-operasi yang dapat diakses melalui jaringan, misalnya internet, dalam bentuk pesan XML. Web service diartikan sebagai sepotong atau sebagian informasi atau proses yang dapat diakses oleh siapa saja, kapan saja dengan menggunakan piranti apa saja, tidak terikat dengan sistem operasi atau bahasa pemrograman yang digunakan.

\subsection{Novi Nuari}

	Webservice ialah suatu sistem perangkat lunak yang dibangun guna mendukung interaksi antar mesin dalam suatu jaringan. Webservice digunakan sebagai suatu fasilitas yang disediakan oleh suatu website untuk menyediakan layanan (dalam bentuk informasi) kepada mesin lain, sehingga mesin lain dapat berinteraksi dengan mesin tersebut melalui layanan-layanan (service) yang disediakan oleh provider\cite{nuari2014perancangan}.

\subsection{Wellem, Theophilus}

	Web service merupakan suatu software sistem yang mendukung interaksi yang interoperable dari machine to machine melalui jaringan (World World Wide Consortium).  (Stencil Group). Dengan suksesnya Web service sebagai suatu standar teknologi software, memberikan peluang yang besar untuk pengembangan aplikasi terdistribusi melalui Internet.
Web service sebagai suatu standar teknologi software, memberikan peluang yang besar untuk pengembangan aplikasi terdistribusi melalui Internet. Saat ini Web service tidak hanya dapat diakses melalui komputer saja, tetapi juga dapat diakses melalui mobile device, seperti telepon seluler dan PDA, sehingga memungkinkan diciptakannya layanan mobile menggunakan Web service dan aplikasi mobile yang menggunakan Web service ini\cite{wellem2015perancangan}.

\subsection{Jurnal Informatika Kenali, Eko Win }

	Menurut Gerami (2002) web services adalah suatu layanan-layanan yang disediakan oleh internet, dengan menggunakan pengiriman pesan format Extensible Markup Language (XML), dan tidak saling bergantung pada satu sistem operasi atau Bahasa pemrograman. Komponen dalam web service memiliki 3 arsitektur, dan masing-masing komponen tersebut adalah Service provider, Service requestor, dan Service registry\cite{kenali2015desain}. 

\subsection{Sigit, Haris Triono and Sulistiyono, Sulistiyono}

	Web Service adalah bagian dari perangkat lunak yang membuat dirinya tersedia melalui internet dan menggunakan sistem pesan XML standar. XML digunakan untuk mengkodekan semua komunikasi ke Web Service. Misalnya, klien memanggil Web Service dengan mengirim pesan XML, kemudian menunggu tanggapan XML yang sesuai. Karena semua komunikasi ada dalam XML, Web Service tidak terkait dengan sistem operasi atau bahasa pemrograman manapun. Web Service adalah kumpulan protokol dan standar terbuka yang digunakan untuk pertukaran data antara aplikasi atau sistem\cite{sigit2017desain}.  

\section{Manfaat}

Layanan web memungkinkan penyedia layanan dan vendor untuk menjual layanan mereka dengan memublikasikannya
Yang di akses melalui World Wide Web.
Manfaat dari layanan web kita dapat berbagi data walaupun memiliki jarak yang jauh dan dapat mempermudah membagi suatu data dalam sebuah pekerjaan
interoperabilitas. Manfaat ini berasal dari antarmuka XML standar dan deskripsi akses
diberikan oleh WSDL (Web Services Description Language). Deskripsi WSDL sangat membantu dalam perusahaan
integrasi aplikasi, integrasi B2B (menyelesaikan tantangan antara bisnis dan bisnis partner, seperti customer, supplier, bank, dan jasa transportasi ) \cite{ferris2003web}.

\section{Arsitektur Web service}

\subsection{\textit{Service Oriented Architecure (SOA)} }

	Konsep arsitektur yang mendasari teknologi Web service adalah Service Oriented Architecure (SOA), SOA mendefinisikan 3 peran berbeda yang menunjukkan peran dari masing-masing komponen dalam system, yaitu (W3C, 2004) :
\begin{itemize}
\item \textit{Service provider}, yaitu suatu entitas yang menyediakan interface terhadap sistem yang menjalankan suatu sekumpulan tugas tertentu.
\item \textit{Service requestor}, yaitu suatu entitas yang meminta/memperoleh (dan menemukan) \textit{software service} dalam rangka meyelesai kan suatu tugas tertentu atau menyediakan solusi bisnis tertentu.
\item \textit{Service registry}, yaitu entitas yang bertindak sebagai penyimpan (\textit{repository}) suatu \textit{software service} yang dipublikasikan oleh \textit{service provider}\cite{hidayat2014penerapan}.
\end{itemize}

\subsection{Jurnal Masyarakat Informatika}

	Web service dibangun dari tiga komponen unsur utama, yaitu service provider, service registry, dan service requestor. Komponen-komponen tersebut saling berinteraksi melalui komponen web service itu sendiri, yang berupa deskripsi dan implementasi layanan dan prasarana. Dan juga terdapat tiga macam operasi yang memungkinkan komponen komponen tersebut untuk dapat saling berinteraksi, yaitu publish, find, dan bind. Keterkaitan antara peran, operasi, dan komponen web service \cite{saputra2integrasi}.

\subsection{Arsitektur RESTful Web services}

	Berikut merupakan langkah-langkah yang dilakukan dalam model dasar RESTful Web services (HostBridge, 2009):
\begin{enumerate}
\item Query Request Provider melalui HTTP dengan menggunakan URI (Uniform Resource Identifier). Request menggunakan methods (metode) HTTP untuk menentukan apakah request tersebut dimaksudkan untuk Create (menciptakan), Read (membaca), Update (memperbarui), atau Delete (menghapus) data.
\item HostBridge mengembalikan sebuah dokumen dalam bentuk XML untuk Requester (pemohon) dengan CICS data enclosed\cite{arsana2014rancang}.
\end{enumerate}



\section{Kesimpulan}

	Dari berbagai definisi tersebut dapat disimpulkan bahwa web service merupakan middleware sebuah internet yang memungkinkan berbagai sistem untuk saling berkomunikasi tanpa terpengaruh pada platform. Web service membungkus operasi-operasi ke dalam sebuah antarmuka yang ditulis dalam notasi XML. Antarmuka ini menyembunyikan detil implementasi dari layanan. Pertukaran informasi yang terjadi dalam web service juga menggunakan pesan dalam format XML \cite{saputra2integrasi}.


