%Resume Hello word python dan identation

%Kelompok 2 D4 TI / 2B

%Alwan Suryansah				1164033 
%Dinda Ayu Pratiwi				1164034
%Kurnia Sandi					1164042
%Teduh Sanubari					1164054
%Wildan Khaustara Wijaksana		1164058

\documentclass[12pt]{article}

\begin{document}
\section{Apa itu Pyton?}

\subsection{Journal JCONES}
	Tulisan ini mendiskusikan pengertian python , python adalah sebuah bahasa pemrograman model skripsi atau ( scripting language) terorientasi objek. atau bisa juga di artikan sebagai bahasa pemrograman yang freeware atau perangkat bebas , tidak ada batasan dalam penyalinan atau mendistribusikan . yang di dalamnya terdapat source code , debugger dan profiler \cite{perkasa2014rancang}.
	
\subsection{Dierbach, Charles}
Bahasa pemrograman Python telah dengan cepat mendapatkan popularitas selama beberapa tahun terakhir sebagai bahasa pilihan untuk kursus CS1. Beberapa perkiraan menyebutkan bahwa penggunaannya meningkat empat puluh persen per tahun. Sejauh 2006 ada laporan peningkatan yang signifikan dalam kepuasan siswa dan instruktur dengan mendesain ulang kursus pengantar untuk menggunakan kesederhanaan Python daripada kompleksitas bahasa seperti Java\cite{dierbach2014python}.

\subsection{Computing, Jurnal ULTIMA}
Python merupakan bahasa pemrograman open source yang banyak digunakan untuk menangani beberapa jenis masalah dalam pemrograman. Python banyak digunakan untuk meningkatkan kualitas perangkat lunak, produktivitas pengembang, portabilitas program, dan integrasi komponen. Python digunakan oleh setidaknya ratusan ribu pengembang di seluruh dunia dalam bidang-bidang seperti scripting internet, pemrograman sistem, user interface, kustomisasi produk, pemrograman numerik, dan banyak lagi\cite{computingaplikasi}.

\section{Membuat "Hello World" dengan Pyton}


\section{Pengertian Identation pada Pyton}


\section{Error yang Muncul serta Solusinya}


\section{Kesimpulan}

\end{document}


