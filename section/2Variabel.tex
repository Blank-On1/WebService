
%Resume Protokol Kelompok 3 D4TI2B
%\begin{enumerate}
%\Fikri aldi nugraha                  1164038
%\Nur Arkhamia Batubara               1164049 
%\Miftahul Hasanah                    1164046 
%\Si Made Angga Dwitya P              1164053 
%\Widary Anggraini Mindo V Siahaan    1164057
%\end{enumerate

\section{Pengertian Variabel}
Variable adalah suatu objek penelitian yang bervariasi dan memiliki gejala yang bervariasi, atau sebagai suatu pusat penelitian yang dapat 
diukur. Variable juga sebagai konsep yang mempunyai  lebih dari satu nilai , suatu keadaan dan suatu kondisi. Pembahasan tentang variable 
sangat  penting  untuk suatu keperluan pada penetapan system penelitian, menstrukturkannya ke dalam teori penelitian sebagai landasan 
pengembangan hipotesis .  

\section{Jenis – Jenis Pengukuran Variabel}
Di dalam variable juga terdapat alat alat yang digunakan dalam pengukuran. Sehingga varibel dikelompokan menjadi empat menurut bentuk   
pengukuran, itu dilakukan supaya banyak cara yang dilakukan dalam menentukan variable yang diinginkan. Variabel Nominal, yaitu variabel
yang dapat ditetapkan berdasarkan penggolongan. Variabel ini juga bersifat diskrit (bijaksana) dan saling pilih memillih antara satu 
kategori dan kategori lain.

\section{Hubungan variable pada hipotesis dapat kita bagi 3 yaitu}
1.	Model kontingensi dapat kita nyatakan dalam bentuk table silang. Contohnya saja yaitu hubungan antara variable antar agama dan antara 
variable parpol pada pemilu pada tahun 1997. 
2.	Model asosiatif yang mana model ini terdapat diantara 2 buah variable yang mana keduanya sama sama ordinal atau keduanya sama sama 
interval.variabel ini mempunyai pola monoton linier.
3.	Hubungan fungsional yaitu merupakan antara suatu variable yang berfungsi di dalam sebuah variable lain.


