
%Resume Variabel Kelompok 3 D4TI2B
%\begin{enumerate}
%\Fikri aldi nugraha                  1164038
%\Nur Arkhamia Batubara               1164049 
%\Miftahul Hasanah                    1164046 
%\Si Made Angga Dwitya P              1164053 
%\Widary Anggraini Mindo V Siahaan    1164057
%\end{enumerate}

\section{Pengenalan Variabel}
Variabel memberikan konsep yang telah diberi ukuran tertentu. Ukuran inilah yang membedakan variabel dengan yang bukan variabel. 
Contoh jenis kelamin merupakan variabel dan dibedakan menjadi 2, yaitu perempuan dan laki-laki. Pendapatan seseorang atau 
kelompok masyarakat merupakan variabel yang dapat dibedakan tingkatannya menjadi kategori setuju, tidak setuju, dan tidak tahu. 
Umur seseorang dapat disebut variabel dan dibedakan (dikategorikan) menjadi empat, yaitu kanak-kanak, remaja, dewasa, dan 
manula/tua. Kekompakan kelompok disebut variabel dan setelah diukur dengan kriteria tertentu dapat dibedakan/dikelompokkan 
menjadi tinggi (kompak sekali), sedang (cukup kompak), rendah (kompak saja), dan sangat rendah (tidak kompak). Kemajuan negara 
dapat disebut variabel dan ditandai dengan ukuran (indikator) berupa tingkat pendidikan atau melek huruf (literasi), pendapatan 
nasional (product domestic brutto), dan tingkat ekspor produksi barang dan jasa. Jadi, variabel adalah konsep (keadaan, 
kegiatan) yang telah diberi ukuran tertentu dan dapat dijadikan objek atau unsur dalam penelitian ilmiah.

Data yang diperoleh dari responden dengan acuan hubungan antarvariabel yang umumnya bersifat dugaan (hipotetis). Jenis atau 
macam penelitian kausal adalah dua variabel (bivariat) atau lebih dari dua variabel (multivariate). Hal yang membedakan unsur 
kausal dengan penelitian exploratory dan descriptive adalah ada masalah/permasalahan, ada hubungan antarvariabel, ada hipotesis, 
ada metode pengumpulan data, ada sumber/narasumber/responden yang diteliti, dan metode analisis data yang jelas dan terstruktur, 
termasuk analisis kuantitatif yang umumnya menggunakan alat bantu statistik inferensial yang meneliti dan menghitung besarnya 
hubungan (korelasi) antarvariabel.

\section{Pengertian Variabel}
Variable adalah suatu objek penelitian yang bervariasi dan memiliki gejala yang bervariasi, atau sebagai suatu pusat penelitian yang 
dapat diukur. Variable juga sebagai konsep yang mempunyai  lebih dari satu nilai , suatu keadaan dan suatu kondisi. Pembahasan tentang 
variable sangat  penting  untuk suatu keperluan pada penetapan system penelitian, menstrukturkannya ke dalam teori penelitian sebagai 
landasan pengembangan hipotesis.

Variabel sebagai suatu tanda pengenal/identifier yang digunakan untuk mewakili suatu nilai tertentu di dalam satu proses program. 
Sebuah nilai dari suatu variable bisa diubah sesuai kebutuha, Nama dari suatu variable  yang dapat ditentukan sendiri oleh program. 
Sebuah operator dalam Bahasa pemograman merupakan sebuah symbol yang dapat dikenakan atau dapat mempengaruhi nilai dari satu atau 
beberapa variable .

\section{Aturan Penamaan Variabel}
Dalam bahasa pemrograman, penamaan variabel menjadi salahsatu hal yang sangat penting dan perlu diperhatikan. 
Ada beberapa bahasa pemrograman yang memiliki sifat case-sensitive sehingga penggunaan huruf besar dan kecil dibedakan seperti pada saat penamaan variabel. Variabel dengan nama PERUSAHAAN akan berbeda dengan perusahaan. 
Beberapa aturan yang  digunakan dalam penamaan variabel adalah sebagai berikut:
\begin{enumerate}
\item Harus unik, tidak boleh ada variabel dengan nama yang sama pada satu ruang lingkup yang sama.
\item Harus dimulai dengan huruf(alfabet).
\item Maksimum 255 karakter, tetapi hanya 40 karakter pertama yang dianggap sebagai nama variabel. Karakter sisanya akan diabaikan.
\item Tidak boleh ada spasi. Sebagai ganti spasi dapat menggunakan karakter underscore “_”. Misalnya nama_dosen.
\item Tidak boleh menggunakan karakter-karakter khusus yang digunakan unutk operator, seperti +,-,*/,<,>,;,=,#,koma, dan lain-lain.
\end{enumerate}

\section{Jenis – Jenis Pengukuran Variabel}
Di dalam variable juga terdapat alat alat yang digunakan dalam pengukuran. Sehingga varibel dikelompokan menjadi empat menurut bentuk   
pengukuran, itu dilakukan supaya banyak cara yang dilakukan dalam menentukan variable yang diinginkan. Variabel Nominal, yaitu variabel
yang dapat ditetapkan berdasarkan penggolongan. Variabel ini juga bersifat diskrit (bijaksana) dan saling pilih memillih antara satu 
kategori dan kategori lain. Beberapa alat pengukuran:

\subsection{Variabel Ordinal}
Variabel Ordinal adalah variabel yang dapat dibangun dan dibentuk berdasarkan jenjang atau tingakatan atribut tertentu. Jenjang 
tertinggi dan terendah juga sesungguhnya dan terendah sesungguhnya ditetapkan menurut kesepakatan. Dan oleh karena itu, angka 1 atau 
juga kelompok 10 dapat berada pada tingkatan jenjang yang tinggi sampai paling rendah. Variabel ini sangat banyak digunakan oleh 
beberapa pihak karena kemudahan dan ke fleksibelannya.

\section{Hubungan variable pada hipotesis dapat kita bagi 3 yaitu}
1.Model kontingensi dapat kita nyatakan dalam bentuk table silang. Contohnya saja yaitu hubungan antara variable antar agama dan antara 
variable parpol pada pemilu pada tahun 1997. 
2.Model asosiatif yang mana model ini terdapat diantara 2 buah variable yang mana keduanya sama sama ordinal atau keduanya sama sama 
interval.variabel ini mempunyai pola monoton linier.
3.Hubungan fungsional yaitu merupakan antara suatu variable yang berfungsi di dalam sebuah variable lain.

\section{Macam macam variable}
Berdasarkan cara pengukuran maka variable(Ferdinand,2006:12) dapat dibagi atas:
1.Yang pertama variable laten , merupakan suatu variabel bentukan dari hasil bentukan melalui indikator-indikator yang diamati dalam 
kehidupan duni nyata. Nama lain dari variabel ini yaitu factor atau konstruk maupun unobserved variable
2.Yang kedua yaitu variabel terukut merupakan sebuah variabel yang mana datanya harus kita cari melalui penelitian langsung kelapangan 
contohnya saja yaitu seperti survei secara langsung. Nama lain dari variabel ini diantaranya observed variable, maupun indicator 
variable.

\section{Array}
Array adalah variabel yang mampu menampung koleksi data terurut dengan tipe sama. 
Nilai-nilai data dalam array disebut dengan elemen-elemen array. Letak urutan elemen array ditunjukkan oleh indeks elemen array. 
Ada dua jenis variabel array, yaitu array yang ukurannya tetap (Fixed Array) dan array yang ukurannya dapat berubah (Dynamic Array).

