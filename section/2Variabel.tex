
%Resume Variabel Kelompok 3 D4TI2B
%\begin{enumerate}
%\Fikri aldi nugraha                  1164038
%\Nur Arkhamia Batubara               1164049 
%\Miftahul Hasanah                    1164046 
%\Si Made Angga Dwitya P              1164053 
%\Widary Anggraini Mindo V Siahaan    1164057
%\end{enumerate}

\section{Pengenalan Variabel}
variabel merupakan konsep yang telah ada ukurannya atau telah diberi nilai. Variabel umur seseorang dapat diteliti dengan 
menanyakan kepada yang bersangkutan, kemudian hasilnya dicatat. Variabel pengetahuan orang tentang media televisi dapat diukur,  yaitu dengan memberi pertanyaan tentang jenis-jenis acara yang sering disaksikan, ketepatan jam tayang acara televisi tersebut, 
dan nama stasiun televisi yang menyiarkan acara. Variabel kedisiplinan pegawai bekerja dapat dilihat dari ketepatan waktu masuk  dan pulang kerja. Sesuai namanya variabel maka di dalamnya ada variasi (ukuran) atau nilai, misalnya tinggi-sedang-rendah, 
sering-jarang, disiplin-tidak disiplin, dan sangat memahami- memahami-tidak memahami.

Variabel memberikan konsep yang telah diberi ukuran tertentu. Ukuran inilah yang membedakan variabel dengan yang bukan variabel. 
Contoh jenis kelamin merupakan variabel dan dibedakan menjadi 2, yaitu perempuan dan laki-laki. Pendapatan seseorang atau 
kelompok masyarakat merupakan variabel yang dapat dibedakan tingkatannya menjadi kategori setuju, tidak setuju, dan tidak tahu. 
Umur seseorang dapat disebut variabel dan dibedakan (dikategorikan) menjadi empat, yaitu kanak-kanak, remaja, dewasa, dan 
manula/tua. Kekompakan kelompok disebut variabel dan setelah diukur dengan kriteria tertentu dapat dibedakan/dikelompokkan 
menjadi tinggi (kompak sekali), sedang (cukup kompak), rendah (kompak saja), dan sangat rendah (tidak kompak). Kemajuan negara 
dapat disebut variabel dan ditandai dengan ukuran (indikator) berupa tingkat pendidikan atau melek huruf (literasi), pendapatan 
nasional (product domestic brutto), dan tingkat ekspor produksi barang dan jasa. Jadi, variabel adalah konsep (keadaan, 
kegiatan) yang telah diberi ukuran tertentu dan dapat dijadikan objek atau unsur dalam penelitian ilmiah.

Data yang diperoleh dari responden dengan acuan hubungan antarvariabel yang umumnya bersifat dugaan (hipotetis). Jenis atau 
macam penelitian kausal adalah dua variabel (bivariat) atau lebih dari dua variabel (multivariate). Hal yang membedakan unsur 
kausal dengan penelitian exploratory dan descriptive adalah ada masalah/permasalahan, ada hubungan antarvariabel, ada hipotesis, 
ada metode pengumpulan data, ada sumber/narasumber/responden yang diteliti, dan metode analisis data yang jelas dan terstruktur, 
termasuk analisis kuantitatif yang umumnya menggunakan alat bantu statistik inferensial yang meneliti dan menghitung besarnya 
hubungan (korelasi) antarvariabel.

Dua variabel dapat dihubungkan bersama dalam satu penelitian komunikasi berjudul Hubungan antara Pengetahuan tentang Media 
dengan Terpaan terhadap Media. Dapat diduga (dirumuskan hipotesis), melalui penelitian ini bahwa ada sekelompok orang yang 
pengetahuan tentang media tinggi (variabel pendahulu/sebab) dan terpaan terhadap media juga tinggi (variabel pengikut/akibat). 
Ada pula pengetahuannya tentang media sedang, tetapi terpaan terhadap media tinggi, dan seterusnya. 

\section{Pengertian Variabel}
Variable adalah suatu objek penelitian yang bervariasi dan memiliki gejala yang bervariasi, atau sebagai suatu pusat penelitian yang 
dapat diukur. Variable juga sebagai konsep yang mempunyai  lebih dari satu nilai , suatu keadaan dan suatu kondisi. Pembahasan tentang 
variable sangat  penting  untuk suatu keperluan pada penetapan system penelitian, menstrukturkannya ke dalam teori penelitian sebagai 
landasan pengembangan hipotesis.

Variabel sebagai suatu tanda pengenal/identifier yang digunakan untuk mewakili suatu nilai tertentu di dalam satu proses program. 
Sebuah nilai dari suatu variable bisa diubah sesuai kebutuha, Nama dari suatu variable  yang dapat ditentukan sendiri oleh program. 
Sebuah operator dalam Bahasa pemograman merupakan sebuah symbol yang dapat dikenakan atau dapat mempengaruhi nilai dari satu atau 
beberapa variable .

Dalam suatu program tidak hanya variable biasa yang meiliki nilai yang berubah-ubah , tetapi suatu bahasa pemograman juga menggunakan 
jenis variable konstanta.  Nilai variable konstanta yang sudah didefenisikan tidak dapat diubah selama proses program. Kemudian aturan 
penamaan konstanta pada dasarnya sama dengan penamaan variable biasa, hanya saja ada beberapa perbedaan dari sisi cara mengisi 
konstanta serta tidak menggunakan tanda $ di depan nama konstantaya.

Variable menyimpan data dan nilai nya dapat berubah-ubah, variable memiliki hubungan yang sangat erat dengan tipe data. Karena setiap 
adanya suatu data perlu di tentukan apa tipe datanya, yang berfungsi sebagai processor dalam mengolah data tersebut. Tipe data adalah 
suatu kelompok yang memiliki jenis-jenis tertentu, dengan kata lain jenis dari data tersebut. 

\subsection{Contoh program yang mengandung variable}
\begin{verbatim}
<?php
$gaji = 100000;
$pajak = 0.2;
$thp = $gaji – ($gaji*pajak;
echo “Gaji sebelum ada pajak = Rp. $gaji <br>”;
echo “Gaji yang dibawa pulang = Rp. $thp”;
?>
\end{verbatim}

\subsection{Variabel pada PHP}

Pada Bahasa pemograman PHP pada dasarnya tipe data variable tidak didefenisikan oleh sang programmer, melainkan tipe data akan secara 
otomatis ditentukan oleh interpreter PHP. Akan tetapi untuk beberapa kebutuhan, programmer dapet mendefenisikan tipe data variable. 
PHP mendukung 8 (delapan) buah tipe data yang memiliki sifat primitive dalam programnya. yaitu :
\begin{enumerate}
1.Boolean
2. Integer
3. Float
4.string 
5. Array
6. Object
7. resource
8. NULL
\end{enumerate}

\section{Aturan Penamaan Variabel}
Dalam bahasa pemrograman, penamaan variabel menjadi salahsatu hal yang sangat penting dan perlu diperhatikan. 
Ada beberapa bahasa pemrograman yang memiliki sifat case-sensitive sehingga penggunaan huruf besar dan kecil dibedakan seperti pada saat penamaan variabel. Variabel dengan nama PERUSAHAAN akan berbeda dengan perusahaan. 
Beberapa aturan yang  digunakan dalam penamaan variabel adalah sebagai berikut:
\begin{enumerate}
\item Harus unik, tidak boleh ada variabel dengan nama yang sama pada satu ruang lingkup yang sama.
\item Harus dimulai dengan huruf(alfabet).
\item Maksimum 255 karakter, tetapi hanya 40 karakter pertama yang dianggap sebagai nama variabel. Karakter sisanya akan diabaikan.
\item Tidak boleh ada spasi. Sebagai ganti spasi dapat menggunakan karakter underscore “_”. Misalnya nama_dosen.
\item Tidak boleh menggunakan karakter-karakter khusus yang digunakan unutk operator, seperti +,-,*/,<,>,;,=,#,koma, dan lain-lain.
\end{enumerate}

\section{Deklarasi Variabel dan Variabel Dinamis}
Variabel di PHP bisa digunakan meskipun belum dideklarasikan. Penamaan variabel harus diawali dengan tanda ($) dan diikuti oleh nama ringkas. Nama variabel tidak boleh diawali dengan angka, tetapi bisa berisi angka dan karakter underscore (_). 
Untuk menghindari kesalahan dalam menggunakan variabel, perlu diketahui, nama variabel bersifat case sensitive.
\begin{verbatim}
//deklarasi dan inisialisasi variabel
$result   = 1 + 5;	 // valid
$Result   = 2 + 5;   // valid, beda dengan $result
$_result  = 3 + 5;   // valid
//deklarasi dan inisialisasi variabel
$1   = 1 + 5;	// tidak valid
$1res  = 2 + 5;	// tidak valid
$#res  = 3 + 5;	// tidak valid
$my-res  = 4 + 5;	// tidak valid
\end{verbatim}
Tanda dollar berfungsi untuk mempermudah membantu anda dalam membedakan variabel dengan fungsi dan keyword PHP.
Tanda dollar merupakan bagian dari nama variabel atau bisa juga dikatakan sebagai suatu operator yang mengacu ke memori.
Pada  bahasa pemrograman PHP,  anda dapat menciptakan nama variabel secara dinamis yaitu dengan menggunakan sintaks variabel-variabel (double variabel).
Variabel-variabel akan mengambil nilai dari suatu variabel dan memperlakukannya seperti nama variabel pada umumnya. Contohnya sebagai berikut:
\begin{verbatim}
$a = ‘Hello’;
$$a = ‘World’;
\end{verbatim}
Tanda double dollar atau ($$) adalah sintaks yang digunakan untuk menuliskan suatu variabel-variabel. Variabel yang juga disebut variabel dinamis ini bisa dipanggil melalui dua cara.

\section{Jenis – Jenis Pengukuran Variabel}
Di dalam variable juga terdapat alat alat yang digunakan dalam pengukuran. Sehingga varibel dikelompokan menjadi empat menurut bentuk   
pengukuran, itu dilakukan supaya banyak cara yang dilakukan dalam menentukan variable yang diinginkan. Variabel Nominal, yaitu variabel
yang dapat ditetapkan berdasarkan penggolongan. Variabel ini juga bersifat diskrit (bijaksana) dan saling pilih memillih antara satu 
kategori dan kategori lain. Beberapa alat pengukuran:

\subsection{Variabel Ordinal}
Variabel Ordinal adalah variabel yang dapat dibangun dan dibentuk berdasarkan jenjang atau tingakatan atribut tertentu. Jenjang 
tertinggi dan terendah juga sesungguhnya dan terendah sesungguhnya ditetapkan menurut kesepakatan. Dan oleh karena itu, angka 1 atau 
juga kelompok 10 dapat berada pada tingkatan jenjang yang tinggi sampai paling rendah. Variabel ini sangat banyak digunakan oleh 
beberapa pihak karena kemudahan dan ke fleksibelannya.

\subsection{Variabel Interval}
Variabel Interval adalah variabel yang dapat dibentuk dan dibangun dari pengukuran. Dalam pengukuran tersebut kita dapat mengasumsikan 
bahwa terdapat satuan pengukuran yang sama. Misalnya,variabel pendapatan artis dalam setahun.Variabel ratio juga merupakan suatu 
variabel yang memiliki permulaan angka nol yang mutlak. Suatu contoh juga seperti variabel umur yaitu: ada yang berumur 0, 1, 2, 3, 
4tahun dan sebagainya. Itu adalah beberapa pengukuran yang ada sehingga lebih mudah melakukan pengukuran variabel.

\section{Bentuk dan Ragam Variabel}
Selain pengukuran variabel seperti penjelasan diatas, variabel juga dibedakan dalam ragamnnya yang berbentuk berbeda-beda seperti 
variabel bebas, variabel tergantung, dan variabel penyela. Variabel bebas adalah variabel yang menentukan arah dan perubahan tertentu 
pada variabel tergantung. Sementara itu variabel bebas berada dalam posisi yang lepas dari pengaruh variabel tergantung. Dengan begitu 
variabel tergantung merupakan variabel yang dipengaruhi oleh variabel bebas.

Namun dalam penelitiann banyak dapat dibuktikan, tidak selamanya variabel bebas dapat mempengaruhi variabel tergantung.
Dengan demikian perubahan pada variabel tergantung tidak semata-mata disebabkan oleh variabel bebas, akan tetapi karena ada factor lain. 
Factor ini juga disebut variabel penyela. Variabel penyela ini berada di antara variabel bebas dan variabel tergantung yang di dalam 
suatu hubungan terdapat sebab akibat. Variabel ini dapat mempengaruhu variabel tergantung  dan pengaruh dari variabel bebas.


\section{Hubungan variable pada hipotesis dapat kita bagi 3 yaitu}
1.Model kontingensi dapat kita nyatakan dalam bentuk table silang. Contohnya saja yaitu hubungan antara variable antar agama dan antara 
variable parpol pada pemilu pada tahun 1997. 
2.Model asosiatif yang mana model ini terdapat diantara 2 buah variable yang mana keduanya sama sama ordinal atau keduanya sama sama 
interval.variabel ini mempunyai pola monoton linier.
3.Hubungan fungsional yaitu merupakan antara suatu variable yang berfungsi di dalam sebuah variable lain.

\section{Macam macam variable}
Berdasarkan cara pengukuran maka variable(Ferdinand,2006:12) dapat dibagi atas:
1.Yang pertama variable laten , merupakan suatu variabel bentukan dari hasil bentukan melalui indikator-indikator yang diamati dalam 
kehidupan duni nyata. Nama lain dari variabel ini yaitu factor atau konstruk maupun unobserved variable
2.Yang kedua yaitu variabel terukut merupakan sebuah variabel yang mana datanya harus kita cari melalui penelitian langsung kelapangan 
contohnya saja yaitu seperti survei secara langsung. Nama lain dari variabel ini diantaranya observed variable, maupun indicator 
variable.

\section{variabel dalam penelitian}
variabel dalam sebuah penelitian dapat kita bedakan menjadi:
a.Variabel independen merupakan variabel yang menjelaskan atau dapat mempengaruhi variabel lainnya.
b.Variabel dependen merupakan variabel yang dijelaskan atau dipengaruhi oleh variabel independen.
c.Variabel moderation  merupakan sebuah variabel yang akan menjadikan kuat  atau melemahkan hubungan langsung antara variabel independen 
dan variabel dependen.
d.Variabel intervening merupakan variabel yang mempengaruhi hubungan antara variabel variabel dependen dan variabel variabel independen 
menjadi hubungan yang tidak langsung.

\subsection{Bentuk hubungan dasar antara variabel yaitu}
1.Hubungan antara variabel independen dengan variabel dependen dapat dikatakan yaitu hubungan korelasional dan juga hubungan antara 
sebab akibat. Hubungan antar variabel ini dapat berupa positif maupun neatif.
2.Hubungan antara variabel independen dan dependen yang dimoderasi oleh  variabel moderating. Dimana variabel moderating tersebut 
mempengaruhi hubungan langsung antara variabel dependen dan independen.
3.Hubungan antara variabel dependen dan independenyang dimediasi oleh variabel intervening. Dimana mempengaruhi hubungan antara kedua 
variabel tersebut menjadi tidak memiliki hubungan tidak langsung.

\section{Array}
Array adalah variabel yang mampu menampung koleksi data terurut dengan tipe sama. 
Nilai-nilai data dalam array disebut dengan elemen-elemen array. Letak urutan elemen array ditunjukkan oleh indeks elemen array. 
Ada dua jenis variabel array, yaitu array yang ukurannya tetap (Fixed Array) dan array yang ukurannya dapat berubah (Dynamic Array).

