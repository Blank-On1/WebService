\documentclass[12pt,a4paper]{article}
\usepackage[left=3.00cm, right=2.00cm, bottom=2.00cm, top=3.00cm]{geometry}
\linespread{1.5}
\begin{document}
\title{FUNGSI PYTHON}
\maketitle

\begin{itemize}
\item
NAMA KELOMPOK 4\\
Ajis Trigunawan			1164031\\
Alimu Dzul Ikroom		1164032\\
Muhammad Hanafi			1164092\\
Riki Karnovi			1164052\\
Yoga Sakti Hadi P		1164059\\
\end{itemize}

\section{Fungsi Phyton}
Python adalah bahasa pemrograman yang dibuat oleh Guido van Rossum dan popular sebagai bahasa skripting dan pemrograman Web. Merujuk pengertian dari wikipedia, Python adalah bahasa pemrograman interpretatif multiguna dengan filosofi perancangan yang berfokus pada tingkat keterbacaan kode. Python diketahui sebagai bahasa yang kemampuan dengan sintaksis kode yang sangat jelas. Salah satu fitur yang tersedia pada python adalah sebagai bahasa pemrograman dinamis yang dilengkapi dengan manajemen memori otomatis.

Python bisa digunakan dalam bermacam-macam pengembangan perangkat lunak dan juga bisa berjalan di banyak platform sistem operasi. Sebuah Komputer hanya bisa mengeksekusi program yang penulisannya dalam bahasa mesin atau bahasa tingkat rendah. Python adalah salah satu bahasa pemrograman tingkat tinggi, Sehingga agar bisa di eksekusi maka program harus diproses dulu sebelum dapat dijalankan. Keuntungan Python dengan bahasa tingkat tingginya yaitu lebih manusiawi.

Bahasa tingkat tinggi bisa dengan mudah dirubah portabel untuk disesuaikan dengan mesin yang menjalankannya. Hal ini beraneka ragam dengan bahasa mesin yang hanya dapat digunakan untuk mesin tersebut. Dengan berbagai macam kelebihan ini, maka tidak sedikit aplikasi ditulis menggunakan bahasa tinflrat tinggi. Proses mengubah dari bentuk bahasa tingkat tinggi ke tingkat lebih kecil dalam bahasa pemrograman ada rkra tipe.

Yakni interpreter dan compiler. interpreter membaca program berbahasa tingkat tinggi lau memproses program tersebut. Hal ini berarti interpreter melakukan perintah apa yang dikatakan dalam program tersebut. Dapat dikatakan. interpreter membaca per baris kemudian mengeksekusinya. 

Python merupakan Bahasa pemograman yang hampir tidak bisa dibedakan dengan C/C++, Setiap python mempunyai fungsi yang dapat mengembalikan sebuah Nilai, Tetapi di Python  bisa juga  tidak mengembalikan sebuah Nilai dan biasaya dikenal dengan nama subroutines yang terdapat pada pemograman VB. Python merupakan Bahasa yang tidak menggunakan compiler dan Bahasa ini juga bisa mengembangkan perangkat lunak, Membangun GUI desktop dan lain-lain.

Python dalam pengambangan web sering digunakan dibackend untuk pengelolaan database server, selain itu Python juga banyak digunakan untuk mengembangkan AI oleh google dan developer lainnya. Python juga di gunakan untuk pengembangan jaringan syaraf tiruan salah satunya project tensorflow. Python juga banyak digunakan dalam IOT karna python adalah salah satu bahasa mesin yang mudah dipelajari dan di pahami.

\end{document}
