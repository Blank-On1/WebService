%\documentclass[12pt,a4paper]{article}
%\usepackage[left=3.00cm, right=2.00cm, bottom=2.00cm, top=3.00cm]{geometry}
%\linespread{1.5}
%\begin{document}
%\title{FUNGSI PYTHON}
%\maketitle

%\begin{itemize}
%\item
%NAMA KELOMPOK 4\\
%Ajis Trigunawan			1164031\\
%Alimu Dzul Ikroom		1164032\\
%Muhammad Hanafi			1164092\\
%Riki Karnovi			1164052\\
%Yoga Sakti Hadi P		1164059\\
%\end{itemize}

\section{Fungsi Phyton}
Python adalah bahasa pemrograman yang dibuat oleh Guido van Rossum dan popular sebagai bahasa skripting dan pemrograman Web. Merujuk pengertian dari wikipedia, Python adalah bahasa pemrograman interpretatif multiguna dengan filosofi perancangan yang berfokus pada tingkat keterbacaan kode. Python diketahui sebagai bahasa yang kemampuan dengan sintaksis kode yang sangat jelas. Salah satu fitur yang tersedia pada python adalah sebagai bahasa pemrograman dinamis yang dilengkapi dengan manajemen memori otomatis.\\

Python bisa digunakan dalam bermacam-macam pengembangan perangkat lunak dan juga bisa berjalan di banyak platform sistem operasi. Sebuah Komputer hanya bisa mengeksekusi program yang penulisannya dalam bahasa mesin atau bahasa tingkat rendah. Python adalah salah satu bahasa pemrograman tingkat tinggi, Sehingga agar bisa di eksekusi maka program harus diproses dulu sebelum dapat dijalankan. Keuntungan Python dengan bahasa tingkat tingginya yaitu lebih manusiawi.\\

Bahasa tingkat tinggi bisa dengan mudah dirubah portabel untuk disesuaikan dengan mesin yang menjalankannya. Hal ini beraneka ragam dengan bahasa mesin yang hanya dapat digunakan untuk mesin tersebut. Dengan berbagai macam kelebihan ini, maka tidak sedikit aplikasi ditulis menggunakan bahasa tinflrat tinggi. Proses mengubah dari bentuk bahasa tingkat tinggi ke tingkat lebih kecil dalam bahasa pemrograman ada rkra tipe.Yakni interpreter dan compiler. interpreter membaca program berbahasa tingkat tinggi lau memproses program tersebut. Hal ini berarti interpreter melakukan perintah apa yang dikatakan dalam program tersebut. Dapat dikatakan. interpreter membaca per baris kemudian mengeksekusinya. \\

Python merupakan Bahasa pemograman yang hampir tidak bisa dibedakan dengan C/C++, Setiap python mempunyai fungsi yang dapat mengembalikan sebuah Nilai, Tetapi di Python  bisa juga  tidak mengembalikan sebuah Nilai dan biasaya dikenal dengan nama subroutines yang terdapat pada pemograman VB. Python merupakan Bahasa yang tidak menggunakan compiler dan Bahasa ini juga bisa mengembangkan perangkat lunak, Membangun GUI desktop dan lain-lain.\\

Python dalam pengambangan web sering digunakan dibackend untuk portal pengaksesan database server atau pembuatan API dengan mekanisme Client-Server, selain itu Python juga banyak digunakan untuk mengembangkan AI oleh google dan developer atau perusahaan lainnya. Python juga di gunakan untuk pengembangan jaringan syaraf tiruan salah satunya project tensorflow. Python juga banyak digunakan dalam IOT karna python adalah salah satu bahasa mesin yang mudah dipelajari dan di pahami. Dapat juga digunakan dalam pembuatan atau pengembangan robot pintar, membuat program untuk melakukan konfigurasi jaringan, melakukan Data Minning untuk mendapatkan data yang dibuthkan, dan dapat juga digunakan untuk membuat aplikasi desktop maupun command line.\\

Dalam dunia hacking, python dapat berfungsi untuk menjadi sebuah penghubung antara hacker dan korbannya, dimana script python akan membuka port pada server secara ilegal agar hacker dapat mengakses server tanpa sepengetahuan pemilik server itu sendiri, dengan begitu hacker dapat menjalankan script python lain untuk memanipulasi server atau membuat proxy server pada server tersebut untuk melakukan kejahatan agar identitas hacker tidak dapat ditemukan.\\

Python dapat berfungsi sebagai bahasa pemrograman untuk membuat game dimana dia memiliki lebih banyak pustaka pustaka yang dapat digunakan untuk mendukung logika dan algoritma dalam game, dimana python lebih mudah dan tidak memakan biaya banyak untuk perbaikan dan pengembangan lebih lanjutnya, apalagi dengan banyaknya buku buku yang membahas tentang python dan referensinya.

Di IOT Python biasanya berfungsi sebagai bahasa untuk program pengontrol seperti untuk mengontrol langkah2 yang akan dilakukan setelah menerima perintah dari web/aplikasi remotenya, dapat juga mengolah  masukan  dari  webcam  berupa  realtime  video  dan   mengirim secara serial. Kode r untuk belok kanan, l untuk berbelok ke kiri. Sedangkan sotware Arduino  IDE berfungsi untuk memprogram arduino agar dapat merespon komunikasi serial  dari python.\\

Terkadang ketika membuat suatu program yang kompleks adakalanya kumpulan instruksi dijadikan satu dalam satu berkas, terutama apabila sering menggunakan sekumpulan intruksi untuk melaksanakan satu tugas yang sifatnya rutin. Skrip python ditulis dengan akhiran .py. Setiap Skrip pada python dianggap sebagai modul. Intinya Modul adalah kode yang disimpan dalam sebuah berkas dalam media penyimpan eksternal. Selanjutnya, kode yang terdapat dalam modul dapat dipergunakan dalam suatu skrip dengan terlebih dahulu mengimpor (mengambil) modul tersebut.\\

Di dalam python, objek merupakan abstraksi terhadap data. Setiap data dinyatakan dalam objek. Sebuah objek memiliki nilai dan tipe, misalnya ketika diberikan perintah a=5, maka sebenarnya a adalah objek. Tipenya adalah bilangan bulat dan nilainya adalah 5. Dan yang perlu diketahui begitu diberikan nilai pada suatu objek maka identitas objek tersebut tidak berubah. Identasi (penulisan text yang menjorok ke kanan) memegang peranan penting dalam penulisan pernyataan pada python mengingat identasi digunakan sebagai blok kode pada pernyataan seperti if, while, dan for.\\

Sebuah program yang besar biasanya tersusun atas sejumlah fungsi. Sebuah fungsi berisi sejumlah pernyataan yang dikemas dengan sebuah nama. Selanjutnya nama ini dapat dipanggil beberapa kali dalam program dan tentu saja cara ini dapat mengurangi duplikasi kode. Alasan lain pembuatan fungsi adalah untuk membuat suatu program agar
dapat dipecah menjadi sejumlah bagian agar dapat dikelola dengan lebih mudah oleh pemrogram daripada kalau hanya berupa satu bagian kode yang besar.

\subsection{Bahasa Python}
Program adalah urutan-urutan instruksi untuk menjalankan suatu komputasi. Komputasi dapat berupa matematis, seperti mencari bilangan prima, persamaan kuadrat, atau yang lainnya.

Bentuk dasar bahasa Pyhton:
\begin{enumerate}
\item Informasi letak interpreter
\item Praprosessor
\item Fungsi
\item Program Utama
\end{enumerate}

Suatu fungsi yang terdiri dari sebuah instruksi atau blok instruksi, maka penulisannya harus menjorok satu spasi kedalam. \\


Python ialah bahasa pemrograman yang sangat mempunyai fungsi yang baik dan disukai oleh pengguna karena :
\begin{enumerate}

\item Sederhana
Python didirikan berdasarkan prinsip pengkodean yang dikembangkan oleh bahasa sebelumnya, namun prinsip-prinsip ini telah dieksploitasi untuk implementasi yang lebih sederhana dalam pemrograman Python.

\item High Level
Python ialah bahasa pemrograman yang sederhana. Ini bisa digunakan untuk pemrograman fungsi yang paling canggih sampai fungsi paling rendah.

\item Sangat sesuai untuk analisis
 Python banyak digunakan dalam matematika dan sains
 
\item Open Source
Python adalah Bahasa pemrograman yang bersifat open source atau tidak berbayar dapat didownload pada www.python.org secara gratis.\\

\item Bersifat OOP.

\end{enumerate}

Python sangatlah cocok untuk menulis program paralel tingkat tinggi. Python sekarang muncul sebagai alternatif kompetitif yang potensial untuk Matlab, Octave, dan lingkungan serupa lainnya. Keuntungan khusus Python adalah bahasanya sangat kaya dan kuat, terutama jika dibandingkan dengan Matlab, Fortran, dan C. Secara khusus, Python adalah bahasa berorientasi objek yang ditafsirkan yang mendukung operator overloading dan menawarkan antarmuka antar-platform ke operator fungsi sistem. Pemrogram C ++ Canggih dapat dengan mudah mencerminkan desain perangkat lunak mereka dengan Python dan bahkan memperoleh lebih banyak fleksibilitas dan keanggunan.\\


Python Memiliki beberapa fiktur contohnya adalah: 
\begin{itemize}

\item Memiliki kepustakaan yang besar. dalam pendistribusi python telah menyediakan langkah - langkah yang siap digunakan untuk berbagai kebetuhan. 

\item Memiliki tata Bahasa yang jernih dan mudah dipelajari. 

\item Memiliki peraturan layout kode sumber yang berguna untuk mempermudah pengecekan, pembacaan kembali, dan penulisan ulang kode sumber berorientasi objek. 

\item Memiliki sistem pengelohan memori otomatis (garbage collection seperti Java). 

\item Modular, mudah diperbarui dengan menciptakan modul-modul baru. Modul-modul tersebut dapat dibuat dengan pemograman Python maupun C/C++. 

\item Menyediakan tempat untuk pengumpulan sampah secara otomatis. Sama seperti pada Bahasa pemrograman Java, Python menyediakan pengaturan penggunaan ingatan computer sehingga para pemrogram tidak perlu melakukan pengaturan ingatan computer secara langsung.

\end{itemize}
Keuntungan lain dari Python adalah bahwa perangkat lunak warisan antarmuka yang ditulis dalam Fortran, C, dan Tidak ada tahapan kompilasi dan penyambungan (link) sehingga kecepatan perubahan pada masa pembuatan sistem aplikasi meningkat. Lalu C ++ jauh lebih sederhana daripada di sebagian besar lingkungan lainnya. Ini karena Python dirancang untuk dapat diperpanjang dengan kode yang dikompilasi untuk efisiensi, dan beberapa alat tersedia untuk memudahkan integrasi kode Python dan dikompilasi.


\subsection{Implementasi Fungsi Pyhton di Java}

Implementasi Python di Java terdiri dari tiga komponen-komponen kunci.Yang pertama adalah parser dan lexer untuk Python grammar. Ini adalah kode Java murni yang dihasilkan oleh generator parser JavaCC. tersedia secara bebas dari Sun. Mereka mirip dengan parser kode-C Python dan lexer. Ketika dijalankan pada file sumber Python parser menghasilkan koleksi objek Java yang mewakili pohon parser.\\

Komponen kedua dari sistem ini adalah kompilator ditulis seluruhnya dengan Python yang melintasi pohon parse ini untuk menghasilkan bytecode Java yang sebenarnya. Ini berbeda dengan implementasi Python saat ini, yang menghasilkan bytecode pribadinya sendiri format. Dengan memproduksi Java bytecodes, di C-world akan dicapai langsung menghasilkan kode mesin. Tapi ini portabel kode mesin yang akan berjalan di semua platform Java.

Komponen ketiga dari implementasi Python di Java adalah koleksi kelas dukungan berbasis Java. Kelas-kelas ini menyediakan implementasi untuk dasar Objek Python (daftar, bilangan bulat, instance, kelas, dan lain sebagainya sebagai serta implementasi fungsi berguna seperti "cetak" dan "impor". Fungsionalitas Python saat ini mesin virtual dibagi antara kelas-kelas pendukung ini dan mesin virtual Java sendiri.

\subsection{Organisasi Sistem}

Tujuannya adalah membangun sistem yang sangat modular yang dapat melakukan simulasi, analisis data, dan visualisasi sering melakukan semua tugas ini secara bersamaan. 

Tetapi, ada kecenderungan untuk melakukannya dengan membangun paket monolitik yang terintegrasi secara ketat (menggunakan hirarki kelas C ++ yang terstruktur dengan baik). Jadi membuat  terlalu formal dan restriktif.  Sehingga mendorong untuk mendukung modul yang mungkin hanya terkait secara longgar satu sama lain. Contoh, perpustakaan grafis tidak perlu bergantung pada struktur yang sama sebagai kode simulasi atau bahkan ditulis dalam bahasa yang sama.

Komponen utama sistem yang diterapkan  berupa perpustakaan C sebagai contoh yang diambil. Ketika Python digunakan, pustaka dikompilasi ke pustaka bersama dan dimuat secara dinamis ke dalam Python sebagai modul ekstensi. Fungsionalitas setiap perpustakaan terpapar berbentuk kumpulan "perintah" Python. Karena ini, lingkungan pemrograman C dan Python berhubungan erat. Dalam banyak kasus, dimungkinkan untuk menerapkan kode yang sama di C atau Python. 

Sementara itu C menyediakan banyak fungsi yang mendasarinya, kegunaan sebenarnya dari sistem adalah dalam bentuk modul dan skrip yang ditulis seluruhnya dengan python. pengguna menulis skrip untuk mengatur dan mengontrol semuanya. beberapa komponen utama seperti visualisasi dan sistem analisis data dapat membuat penggunaan python amat berat. memanfaatkan berbagai modul dalam pustaka phyton dan memiliki versi tkinter dan pustaka pencitraan python yang dapat dimuat secara dinamis.

\subsection{Penyematan Python dan Menyembunyikan Ketergantungan Sistem}
Salah satu masalah implementasi terbesar yang telah ditemui adalah fakta bahwa kode SPaSM adalah aplikasi paralel yang sangat bergantung pada implementasi yang tepat dari layanan sistem tingkat rendah seperti I / O dan manajemen proses. Saat ini, memungkinkan untuk menjalankan kode dalam dua konfigurasi yang berbeda, salah satu yang menggunakan pesan yang lewat melalui perpustakaan MPI dan yang lain menggunakan benang Solaris. Menggunakan Python dengan kedua versi kode membutuhkan tingkat kepedulian. Khususnya perushaan merasa perlu menyediakan Python dengan dukungan I / O yang ditingkatkan untuk berjalan dengan baik secara paralel. Pekerjaan ini telah dijelaskan di tempat lain.

\subparagraph{Elemen Dasar Pemrograman Python}

\begin{itemize}

\item Pengenal atau identifier adalah nama yang dipakai untuk fungsi, variable dan konstanta yang didefinisikan oleh pemrogram. Pengenal memiliki aturan dan penulisan seperti:

\begin{itemize}

\item Nama variable harus diawali dengan huruf atau karakter garis bawah , selanjutnya dapat diikuti dengan huruf maupun angka atau tanda garis bawah. Nama variable tidak boleh diawali dengan angka.
\item Nama variable tidak boleh menggunakan operator-operator aritmatika seperti dan karakter-karakter khusus. 
\item jika nama variable tediri dari dua kata atau lebih, maka antarkata tidak dibolehkan menggunakan spasi.
\item Nama variable tidak boleh menggunakan kata – kata yang telah memiliki sebuah arti yang khusus dalam Bahasa python.
\item Penggunaan huruf kecil dan huruf besar harus dibedakan.
\item Panjang maksimal suatu variable adalah 32 karakter sehingga jika mendeklarasikan suatu variable yang panjangnya lebih dari 32 karakter, secara otomatis system tetap akan mengenali sepanjang 32 karakter saja
\end{itemize}
\item Keyword Inti

Keyword inti dalam Bahasa pemograman python adalah kumpulun kata-kata yang dicadangkan. Kata-kata ini biasanya tidak digunakan sebagai identifier. Kata kunci sebagai yang dapat dicontohkan sebagai berikut :
And  assert  break  class
Continue  def  del  elif  else
Except
Exec  finally  for  from
Global
If  import  in  is
Lambda  not  or  pass  print
Raise  return  try  while

\item Tipe Data Dasar

Dalam Bahasa pemograman python memiliki tipe data yang dasar, Nilai dalam python adalah salah satu yang pokok, seperti huruf dan angka. Nilai sewaktu waktu dapat berupa hasil penjumlahan atau berupa String.

\subsection{Contoh Yang Lebih Rumit}
	Implementasi yang sebenarnya dari Python  ke Java bytecode translator yaitu melakukan terjemahan langsung dari Python ke Java bytecode. Meskipun demikian, hampir semuanya  bisa dilihat secara konseptual sebagai terjemahan Python ke Java. Karena cenderung lebih mudah membaca kode sumber Java daripada kode assembly Java VM, berikut ini akan disajikan beberapa contohnya:

\begin{itemize}

\item Baris 1 mendeklarasikan kelas Java baru. Setiap modul Python diimplementasikan sebagai satu kelas Java. Setiap modul Python juga harus mengimplementasikan antarmuka PyRunnable yang memungkinkan modul untuk diinisialisasi (dan membutuhkan modul untuk mengimplementasikan metode "run"). 
\item Baris 3-6 menentukan kolam konstan untuk modul Python ini sebagai medan statis di kelas Java. Masing-masing bidang statis ini menyimpan objek Java yang sesuai dengan beberapa konstanta Python primitif.

\end{itemize}

\end{itemize}

Baris 13 adalah pernyataan impor sederhana. Semua pekerjaan yang menarik disembunyikan di sini dalam implementasi Py.importModule (). Saya tidak ingin membahas detailnya di sini, saya hanya menggunakannya sebagai alat yang nyaman untuk memasukkan objek ke dalam ruang nama saya yang dapat saya dapatkan dan atur atributnya.

Baris 15 adalah contoh pengaturan atribut pada suatu objek. Mereka yang akrab dengan metode khusus Python harus mengenali setattr metode segera. Metode ini bekerja seperti versi Python kecuali bahwa ia beroperasi pada objek Java (beberapa di antaranya mungkin mewakili objek PyInstance). 

Akhirnya, garis 17 mencetak atribut yang baru saja diatur. Sekali lagi, etattr berfungsi seperti metode khusus Python yang sesuai.\\

\subsection {Integrasi Python dan Java}
Mengkompilasi Python ke Java bytecode hanya merupakan bagian dari gambar untuk membuat Python dan Java bekerja bersama. Yang juga dibutuhkan adalah mekanisme yang mulus untuk memungkinkan. Kode Python untuk menggunakan pustaka Java (lihat pekerjaan terkait oleh
Kevin Butler dan Douglas Cunningham dan kemungkinkan kode Python untuk dipanggil dari Java (saya pikir ini akan benar-benar sulit untuk dicapai dengan generalitas apa pun dua pendekatan lainnya untuk menanamkan Java dengan Python). Applet memberikan contoh yang bagus tentang tempat semacam ini integrasi bi-directional yang mulus diperlukan. Di meja 3 Saya menunjukkan kode sumber Python untuk wajangan sederhana Aplikasi "Hello World". Ini akan berjalan di browser web apa pun yang mendukung JDK1.1. Saat ini termasuk Browser HotJava SUN, dan versi Internet yang ditambal Explorer 3.1 dan Netscape Navigator 4.0. Sebelum akhir 1997 semua browser web utama harus mendukung JDK 1.1 tanpa tambalan. Contoh ini menunjukkan bagaimana kelas Python dapat membuat instance kelas Java, dan memohon metode pada ini contoh. Ini juga menunjukkan bagaimana kelas Python dapat menjadi subkelas dari kelas Java dan menimpa metode tertentu. Baris 1 mengimpor paket "java". Paket ini adalah akar hierarki kelas Java standar. Jika saya tidak ingin menggunakan nama yang memenuhi syarat, saya bisa menulis hal-hal seperti "dari Applet impor java.applet" sebagai gantinya. Dukungan baru untuk paket-paket dalam Python 1.5 membuatnya begitu bahwa sintaksis dan semantik ini untuk mengimpor paket Java hampir sama dengan yang digunakan untuk Python paket.\\

Baris 2 membuat kelas Python baru “HelloApplet” yang mana subclass kelas Java, "java.applet.Applet". Itu kemampuan kelas Python ke kelas Java subclass adalah a bagian kunci dari integrasi dua bahasa. Ini juga berarti bahwa Python dalam kode Java bias subkelas dari kelas bawaan termasuk daftar Python dan kamus (tidak ada lagi UserList.py).\\

Baris 3 mengimplementasikan metode “cat” untuk applet ini obyek. Ini mengesampingkan implementasi standar metode ini di superclass Java. Ia menerima satu argumen (selain dari dirinya) yang merupakan grafis Java obyek.\\

Baris 4-8 menggunakan berbagai metode pada objek grafik ini untuk menggambar magenta norak besar di hitam "Halo Dunia ”dalam browser tempat ia beroperasi. Semua metode yang disebut di sini adalah metode Java, diimplementasikan dalam paket java.awt. Pada baris 5, seseorang dipanggil dengan Objek java sesuai dengan warna hitam. Sejalan 6, metode "fill3DRect" dipanggil dengan lima bilangan bulat Python. Benda-benda Python ini dipaksa untuk yang sesuai Bilangan bulat primitif Java saat melakukan panggilan. Baris 6 menunjukkan pembuatan contoh baru dari kelas java.awt.Font. Sintaks untuk pembuatan contoh adalah persis seperti yang digunakan saat membuat instance baru dari Kelas Python. Objek Python yang merupakan argument secara tepat dipaksa ke objek Java atau primitif ketika kelas benar-benar dipakai.\\



