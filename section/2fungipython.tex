\documentclass[12pt,a4paper]{article}
\usepackage[left=3.00cm, right=2.00cm, bottom=2.00cm, top=3.00cm]{geometry}
\linespread{1.5}
\begin{document}
\title{FUNGSI PYTHON}
\maketitle

\begin{itemize}
\item
NAMA KELOMPOK 4\\
Ajis Trigunawan			1164031\\
Alimu Dzul Ikroom		1164032\\
Muhammad Hanafi			1164092\\
Riki Karnovi			1164052\\
Yoga Sakti Hadi P		1164059\\
\end{itemize}

\section{Fungsi Phyton}
Python adalah bahasa pemrograman yang dibuat oleh Guido van Rossum dan popular sebagai bahasa skripting dan pemrograman Web. Merujuk pengertian dari wikipedia, Python adalah bahasa pemrograman interpretatif multiguna dengan filosofi perancangan yang berfokus pada tingkat keterbacaan kode. Python diketahui sebagai bahasa yang kemampuan dengan sintaksis kode yang sangat jelas. Salah satu fitur yang tersedia pada python adalah sebagai bahasa pemrograman dinamis yang dilengkapi dengan manajemen memori otomatis.

Python bisa digunakan dalam bermacam-macam pengembangan perangkat lunak dan juga bisa berjalan di banyak platform sistem operasi. Sebuah Komputer hanya bisa mengeksekusi program yang penulisannya dalam bahasa mesin atau bahasa tingkat rendah. Python adalah salah satu bahasa pemrograman tingkat tinggi, Sehingga agar bisa di eksekusi maka program harus diproses dulu sebelum dapat dijalankan. Keuntungan Python dengan bahasa tingkat tingginya yaitu lebih manusiawi.
\end{document}