\documentclass[12pt,a4paper]{article}
\usepackage[left=3.00cm, right=2.00cm, bottom=2.00cm, top=3.00cm]{geometry}
\linespread{1.5}
\begin{document}
\title{FUNGSI PYTHON}
\maketitle

\begin{itemize}
\item
NAMA KELOMPOK 4\\
Ajis Trigunawan			1164031\\
Alimu Dzul Ikroom		1164032\\
Muhammad Hanafi			1164092\\
Riki Karnovi			1164052\\
Yoga Sakti Hadi P		1164059\\
\end{itemize}

\section{Fungsi Phyton}
Python adalah bahasa pemrograman yang dibuat oleh Guido van Rossum dan popular sebagai bahasa skripting dan pemrograman Web. Merujuk pengertian dari wikipedia, Python adalah bahasa pemrograman interpretatif multiguna dengan filosofi perancangan yang berfokus pada tingkat keterbacaan kode. Python diketahui sebagai bahasa yang kemampuan dengan sintaksis kode yang sangat jelas. Salah satu fitur yang tersedia pada python adalah sebagai bahasa pemrograman dinamis yang dilengkapi dengan manajemen memori otomatis.\\

Python bisa digunakan dalam bermacam-macam pengembangan perangkat lunak dan juga bisa berjalan di banyak platform sistem operasi. Sebuah Komputer hanya bisa mengeksekusi program yang penulisannya dalam bahasa mesin atau bahasa tingkat rendah. Python adalah salah satu bahasa pemrograman tingkat tinggi, Sehingga agar bisa di eksekusi maka program harus diproses dulu sebelum dapat dijalankan. Keuntungan Python dengan bahasa tingkat tingginya yaitu lebih manusiawi.\\

Bahasa tingkat tinggi bisa dengan mudah dirubah portabel untuk disesuaikan dengan mesin yang menjalankannya. Hal ini beraneka ragam dengan bahasa mesin yang hanya dapat digunakan untuk mesin tersebut. Dengan berbagai macam kelebihan ini, maka tidak sedikit aplikasi ditulis menggunakan bahasa tinflrat tinggi. Proses mengubah dari bentuk bahasa tingkat tinggi ke tingkat lebih kecil dalam bahasa pemrograman ada rkra tipe.Yakni interpreter dan compiler. interpreter membaca program berbahasa tingkat tinggi lau memproses program tersebut. Hal ini berarti interpreter melakukan perintah apa yang dikatakan dalam program tersebut. Dapat dikatakan. interpreter membaca per baris kemudian mengeksekusinya. \\

Python merupakan Bahasa pemograman yang hampir tidak bisa dibedakan dengan C/C++, Setiap python mempunyai fungsi yang dapat mengembalikan sebuah Nilai, Tetapi di Python  bisa juga  tidak mengembalikan sebuah Nilai dan biasaya dikenal dengan nama subroutines yang terdapat pada pemograman VB. Python merupakan Bahasa yang tidak menggunakan compiler dan Bahasa ini juga bisa mengembangkan perangkat lunak, Membangun GUI desktop dan lain-lain.\\

Python dalam pengambangan web sering digunakan dibackend untuk portal pengaksesan database server atau pembuatan API dengan mekanisme Client-Server, selain itu Python juga banyak digunakan untuk mengembangkan AI oleh google dan developer atau perusahaan lainnya. Python juga di gunakan untuk pengembangan jaringan syaraf tiruan salah satunya project tensorflow. Python juga banyak digunakan dalam IOT karna python adalah salah satu bahasa mesin yang mudah dipelajari dan di pahami. Dapat juga digunakan dalam pembuatan atau pengembangan robot pintar, membuat program untuk melakukan konfigurasi jaringan, melakukan Data Minning untuk mendapatkan data yang dibuthkan, dan dapat juga digunakan untuk membuat aplikasi desktop maupun command line.\\

Terkadang ketika membuat suatu program yang kompleks adakalanya kumpulan instruksi dijadikan satu dalam satu berkas, terutama apabila sering menggunakan sekumpulan intruksi untuk melaksanakan satu tugas yang sifatnya rutin. Skrip python ditulis dengan akhiran .py. Setiap Skrip pada python dianggap sebagai modul. Intinya Modul adalah kode yang disimpan dalam sebuah berkas dalam media penyimpan eksternal. Selanjutnya, kode yang terdapat dalam modul dapat dipergunakan dalam suatu skrip dengan terlebih dahulu mengimpor (mengambil) modul tersebut.\\

Di dalam python, objek merupakan abstraksi terhadap data. Setiap data dinyatakan dalam objek. Sebuah objek memiliki nilai dan tipe, misalnya ketika diberikan perintah a=5, maka sebenarnya a adalah objek. Tipenya adalah bilangan bulat dan nilainya adalah 5. Dan yang perlu diketahui begitu diberikan nilai pada suatu objek maka identitas objek tersebut tidak berubah. Identasi (penulisan text yang menjorok ke kanan) memegang peranan penting dalam penulisan pernyataan pada python mengingat identasi digunakan sebagai blok kode pada pernyataan seperti if, while, dan for.\\

Python ialah bahasa pemrograman yang sangat mempunyai fungsi yang baik dan disukai oleh pengguna karena :
\begin{enumerate}

\item Sederhana
Python didirikan berdasarkan prinsip pengkodean yang dikembangkan oleh bahasa sebelumnya, namun prinsip-prinsip ini telah dieksploitasi untuk implementasi yang lebih sederhana dalam pemrograman Python.

\item High Level
Python ialah bahasa pemrograman yang sederhana. Ini bisa digunakan untuk pemrograman fungsi yang paling canggih sampai fungsi paling rendah.

\item Sangat sesuai untuk analisis
 Python banyak digunakan dalam matematika dan sains
 
\item Open Source
Python adalah Bahasa pemrograman yang bersifat open source atau tidak berbayar dapat didownload pada www.python.org secara gratis.\\

\item Bersifat OOP.

Python sangatlah cocok untuk menulis program paralel tingkat tinggi. Python sekarang muncul sebagai alternatif kompetitif yang potensial untuk Matlab, Octave, dan lingkungan serupa lainnya. Keuntungan khusus Python adalah bahasanya sangat kaya dan kuat, terutama jika dibandingkan dengan Matlab, Fortran, dan C. Secara khusus, Python adalah bahasa berorientasi objek yang ditafsirkan yang mendukung operator overloading dan menawarkan antarmuka antar-platform ke operator fungsi sistem. Pemrogram C ++ Canggih dapat dengan mudah mencerminkan desain perangkat lunak mereka dengan Python dan bahkan memperoleh lebih banyak fleksibilitas dan keanggunan.


\end{enumerate}
\end{document}
