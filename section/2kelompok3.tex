\section{ArsitekturClientServer
Arsitektur Client Server merupakan  sebuah model yang membedakan kinerja koputer sebagai client dan server. 
Arsitektur akan menyesuaikan komputer sebagai server dan server akan melayani client yang terhubung kedalam sebuah jaringan.
Server dapat melayani berbagai file server, sebuah printer, bahkan jalur komunikasi.
Client tidak dapat berfungsi sebagai server akan tetapi server dapat berfungsi menjadi client yang dinamakan `server non-dedicated`.
Kerjanga Arsitektur ini sangatlah simpel yaitu server akan menunggu client membuat permintaan dan server akan memproses serta
akan memberikan hasilnya kepada client. Berbeda dengan client, tugas client akan mengirimkan sebuah permintaan kepada server, lalu
client akan menunggu respon yang diberikan oleh server.

\section{PenerapanXMLdiWebService}
Pada layanan web ialah suatu konsep yang baru dalam sistem yang terdistribusi melalui web yang menggunakan teknologi xml dengan menggunakan protokol yang standar SOAP dan HTTP. Dengan menggunakan layanan web sangat dapat mendukung sistem yang terdistribusi yang telah memiliki insfratuktur yang berbeda pula, karena kini layanan web telah menggunakan xml maka dari itu teknologi ini telah dapat mendukung dari berbagai platform yang ada pada sistem maupun aplikasi.

\section{ArsitekturServer}
Arsitektur clien/server menggunakan LAN untuk menjalankan personal komputer, Modul LAN dan DBMS mengendalikan, mengamankan secara bersamaan dan merupakan query untuk support akses dari beberapa pengguna dalam menyambungkan database.
Arsitektur client/memiliki tiga komponen antara :
- Presentation Logic, menangani memformat dan mempresenting data pada pengguna.
- Processing Logic, komponen ini menangani logika data pemrosesan. Proses data logic merupakan aktifitas memvalidasi data, mengindentifikasi proses error pada data.
- Storage Logic, menangani penyimpanan, perbaikan data dari alat penyimpanan yang bekerja dengan aplikasi tersebut.

\section{konsep Client-Server}
Clien-Server ialah komunikasi antara 2 atau lebih pada komputer yang melakukan pembagian tugas masing-masing komputer. Client memiliki beberapa tugas yaitu : input, update, delete, dan dapat menampilkan data sebuah database. Sedangkan Server bertugas untuk menyediakan pelayanan untuk melakukan managemen, yait : menyimpan & mengolah database. Aplkasi Client-server merupakan jawaban atas berkembangnya teknologi informasi, di mana suatu perusahaan memiliki banyak departemen dan harus terhubung satu sama lain dalam melakukan akses data.