\section{ArsitekturClientServer
Arsitektur Client Server merupakan  sebuah model yang membedakan kinerja koputer sebagai client dan server. 
Arsitektur akan menyesuaikan komputer sebagai server dan server akan melayani client yang terhubung kedalam sebuah jaringan.
Server dapat melayani berbagai file server, sebuah printer, bahkan jalur komunikasi.
Client tidak dapat berfungsi sebagai server akan tetapi server dapat berfungsi menjadi client yang dinamakan `server non-dedicated`.
Kerjanga Arsitektur ini sangatlah simpel yaitu server akan menunggu client membuat permintaan dan server akan memproses serta
akan memberikan hasilnya kepada client. Berbeda dengan client, tugas client akan mengirimkan sebuah permintaan kepada server, lalu
client akan menunggu respon yang diberikan oleh server.

\section{PenerapanXMLdiWebService}
Pada layanan web ialah suatu konsep yang baru dalam sistem yang terdistribusi melalui web yang menggunakan teknologi xml dengan menggunakan protokol yang standar SOAP dan HTTP. Dengan menggunakan layanan web sangat dapat mendukung sistem yang terdistribusi yang telah memiliki insfratuktur yang berbeda pula, karena kini layanan web telah menggunakan xml maka dari itu teknologi ini telah dapat mendukung dari berbagai platform yang ada pada sistem maupun aplikasi.

\section{ArsitekturServer}
Arsitektur clien/server menggunakan LAN untuk menjalankan personal komputer, Modul LAN dan DBMS mengendalikan, mengamankan secara bersamaan dan merupakan query untuk support akses dari beberapa pengguna dalam menyambungkan database.
Arsitektur client/memiliki tiga komponen antara :
- Presentation Logic, menangani memformat dan mempresenting data pada pengguna.
- Processing Logic, komponen ini menangani logika data pemrosesan. Proses data logic merupakan aktifitas memvalidasi data, mengindentifikasi proses error pada data.
- Storage Logic, menangani penyimpanan, perbaikan data dari alat penyimpanan yang bekerja dengan aplikasi tersebut.

\section{Komponen Client-Server}
Dari sisi server bertugas melayani client dalam hal memberikan data yang diminta oleh client.
Lalu, model 2-tier server meyediakan sebuah Stored procedure, Triggers, Query. 
Mengapa menggunakan MySQL dari pada MS SQL server yang lebih kompatibel dengan Visual Basic ? Karena, aplikasi MS SQL bersifat komersial tentu anda harus membelinya. 
Sebaliknya dengan MySQL server bersifat gratis yang mudah didapatkan serta banyak yang menggunakannya. Sedangkan dari sisi Client bertugas menyediakan interface untuk pengguna dalam mengoperasikan pada database. 
Interface dapat dibuat dengan menggunakan bahasa pemrograman yang sudah kita ketahui, contoh seperti bahasa pemrograman pada Visual Studio (VB, C# dan Visual C++), Java, Delpi dan lain-lain. 
Interface menyediakan tampilan untuk memudahkan pengguna dalam mengedit, mendelete serta menampilkan data yang ada di DBMS komputer server ke komputer client.

\section{konsep Client-Server}
Clien-Server ialah komunikasi antara 2 atau lebih pada komputer yang melakukan pembagian tugas masing-masing komputer. Client memiliki beberapa tugas yaitu : input, update, delete, dan dapat menampilkan data sebuah database. Sedangkan Server bertugas untuk menyediakan pelayanan untuk melakukan managemen, yait : menyimpan & mengolah database. Aplkasi Client-server merupakan jawaban atas berkembangnya teknologi informasi, di mana suatu perusahaan memiliki banyak departemen dan harus terhubung satu sama lain dalam melakukan akses data.

\section{Keuntungan Penerapan Client-Server Web Service}
Web Service bisa digunakan untuk alternatif dalam pengembangan Aplikasi n-tier, yang mana dapat dipisahkan
antara Database, aplikasi dan Klien. 

dalam penerapan n-tier ke web service, untuk logika aplikasi dapat diterapkan dengan web services
sehingga disisi klient tidak direpotkan dengan instalasi beberapa layer seperti halnya corba atau sejenisnya.
dengan menggunakan web service, method atau function yang developer buat dapat digunakan berulang - ulang bahkan
untuk keperluan aplikasi yang berbeda (penggunaan kembali function). penerapan yang lebih jauh dari web service adalah SOA dan SOAP

\section{Contoh penerapan Client-server Web Service}
\subsection{Implementasi Web Service Dalam Pencarian Objek Wisata Berbasis Android}
Pengimplementasian dari Web Service didalam sebuah perangkat android untuk memproses
dalam pencarian objek wisata menggunakan android, proses yang terjadi didalamnya adalah :

\begin{enumerate}
    \item Pilihan objek wisata berdasarkan kategori yang diinginkan.
    \item Fitur explore atau pencarian objek wisata berdasarkan kriteria yang diinginkan
    \item Review dan detail mengenai sebuah objek wisata.
    \item Fitur direksi untuk menunjukkan jalan menuju lokasi wisata yang dipilih dengan memanfaatkan google maupun
    \item Add Location atau fitur menambahkan lokasi objek wisata yang dikunjungi
\end{enumerate}

\section{ArsitekturDatabaseServer}
Dalam Arsitekture Client Server juga membutuhkan atau menggunakan sebuah Database, terutama di dalam server ada yang
dinamakan dengan Arsitekture Database Server yaitu dimana Client akan bertanggung jawab dalam pengelolaan antar muka pemakai
sedangkan Database Server akan bertanggung jawab dengan pada penyimpanan, pengaksesan, serta pemrosesan database.
Di dalam Database Server ini memiliki sebuah kemampuan dalam pemrosesan yang cukup tinggi sehingga beban dalam jaringan akan 
menjadi berkurang. Database Server ini termasuk kedalam two-tier architecture.

\section {Penggunaan Teknologi internet dalam Dunia Bisnis}
Pemasaran internet ada 2 metode yaitu, push dan pull marketing. Keuntungan yang dapat diperoleh dari internet ialah komunikasi global dan interaktif; yang menyediakan suatu informasi dan pelayanan sesuai dengan kebutuhan konsumen; meningkatkan kerjasama; kemudian memungkinkan bagi pengguna untuk membuka pasar,produk, atau pelayanan baru; serta mengintegrasikan aktivitas secara online. 
Pembayaran transaksi electronic commerce diatur oleh Secure Socket Layer yang dikembangkan menjadi Secure Electronic Transaction


    
