\documentclass[12pt,a4paper]{article}
\usepackage[left=3.00cm, right=2.00cm, bottom=2.00cm, top=3.00cm]{geometry}
\begin{document}
\title{Pemanggilan Modul}
\maketitle
\begin{enumerate}
\item Fransiscus Ivan Martongam      1164039 \\
\item Lalita Chandiany Adiputri      1164043\\
\item Eko Cahyono Putro              1164035\\
\item Lidwina Triniska Gulo          1164044\\
\item Sulpadianti Bunyamin           1164096\\
\end{enumerate}

\section{Modularitas dan Portabilitas}
Modularitas dan portabilitas merupakan faktor pentig karena termasuk atribut kualitas perangkat lunak. Modularitas berasal dari kata modul, modul adalah bagian perangkat lunak yang besar yang dipecah menjadi bagian yang lebih kecil dengan memberi nama. Pengalamatan memori berbeda beda kemudian diintergrasikan untuk membentuk perangkat lunak yang dapat memenuhi kebutuhan dari suatu permasalahan

\section{Paradigma Berorientasi Obyek}
Paradigma berorientasi obyek adalah suatu cara mengorganisasikan perangkat lunak sebagai kumpulan obyek- obyek yang memiliki sifat (struktur data) dan perilaku (fungsi) yang saling berinteraksi melalui pesan (message). Konsep yang menjadi pilar paradigma berorientasi obyek adalah : abstraction, encapsulation, inheritance, polymorphism. Abstraksi dipresentasikan sebagai suatu kelas yang digunakan untuk instansiasi objek.

\subsection{Obyek dan Class}
Obyek (object) merupakan representasi dari entitas sebagai sarana pembungkusan karakteristik struktural
yang dapat disebut atribut dan karakteristik perilaku yang disebut operasi (operation/method). Atribut 
mempresentasikan karakteristik entitas yang menentukan keadaan suatu obyek jika menerima pesan.
Operasi tersebut dapat berupa prosedur maupun fungsi. Kelas merupakan deskripsi dari suatu obyek pada 
saat implementasi (coding).


\subsection{Penurunan Sifat}
Penurunan sifat (inheriatance) adalah kemampuan suatu obyek mewarisi sifat dari obyek yang lain. Kemampuan ini menghasilkan program yang efesien karena ada mekanisme pemakaian kembali (resauble) kode program. Pemograman yang dibuat dapat menggunakan fungsi yang ada dalam file DLL dengan mengirim parameter dan menerima balikan dari fungsi dan selama dapat mengikuti kesepakatan dalam pemangglan fungsi atau prosedure tersebut.


\section{Dynamic link library}
System operasi windows dapat menggunakan proses linker konvensional (proses linker secara statis) dengan file berinteraksi LIB dan dapat secara dinamis menggunakan dynamic link library (DLL).
proses linking fungsi dari DLL secara fisik tidak disalin dan digabung kedalam executable file tetapi tetap terpisah dan dipanggil oleh executable file (“client”) pada saat runtime.

\end{document}
