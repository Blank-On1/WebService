\documentclass[12pt,a4paper]{article}
\usepackage[left=3.00cm, right=2.00cm, bottom=2.00cm, top=3.00cm]{geometry}
\begin{document}
\title{Pemanggilan Modul}
\maketitle
\begin{enumerate}
\item Fransiscus Ivan Martongam      1164039 \\
\item Lalita Chandiany Adiputri      1164043\\
\item Eko Cahyono Putro              1164035\\
\item Lidwina Triniska Gulo          1164044\\
\item Sulpadianti Bunyamin           1164096\\
\end{enumerate}

\section{Modularitas dan Portabilitas}
Modularitas dan portabilitas merupakan faktor pentig karena termasuk atribut kualitas perangkat lunak. Modularitas berasal dari kata modul, modul adalah bagian perangkat lunak yang besar yang dipecah menjadi bagian yang lebih kecil dengan memberi nama. Pengalamatan memori berbeda beda kemudian diintergrasikan untuk membentuk perangkat lunak yang dapat memenuhi kebutuhan dari suatu permasalahan

\section{Paradigma Berorientasi Obyek}
Paradigma berorientasi obyek adalah suatu cara mengorganisasikan perangkat lunak sebagai kumpulan obyek- obyek yang memiliki sifat (struktur data) dan perilaku (fungsi) yang saling berinteraksi melalui pesan (message). Konsep yang menjadi pilar paradigma berorientasi obyek adalah : abstraction, encapsulation, inheritance, polymorphism. Abstraksi dipresentasikan sebagai suatu kelas yang digunakan untuk instansiasi objek.





\end{document}
