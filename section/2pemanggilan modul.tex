\documentclass[12pt,a4paper]{article}
\usepackage[left=3.00cm, right=2.00cm, bottom=2.00cm, top=3.00cm]{geometry}
\begin{document}
\title{Pemanggilan Modul}
\maketitle
\begin{enumerate}
\item Fransiscus Ivan Martongam      1164039 \\
\item Lalita Chandiany Adiputri      1164043\\
\item Eko Cahyono Putro              1164035\\
\item Lidwina Triniska Gulo          1164044\\
\item Sulpadianti Bunyamin           1164096\\
\end{enumerate}

\section{Modularitas dan Portabilitas}
Modularitas dan portabilitas merupakan faktor pentig karena termasuk atribut kualitas perangkat lunak. Modularitas berasal dari kata modul, modul adalah bagian perangkat lunak yang besar yang dipecah menjadi bagian yang lebih kecil dengan memberi nama. Pengalamatan memori berbeda beda kemudian diintergrasikan untuk membentuk perangkat lunak yang dapat memenuhi kebutuhan dari suatu permasalahan

\section{Paradigma Berorientasi Obyek}
Paradigma berorientasi obyek adalah suatu cara mengorganisasikan perangkat lunak sebagai kumpulan obyek- obyek yang memiliki sifat (struktur data) dan perilaku (fungsi) yang saling berinteraksi melalui pesan (message). Konsep yang menjadi pilar paradigma berorientasi obyek adalah : abstraction, encapsulation, inheritance, polymorphism. Abstraksi dipresentasikan sebagai suatu kelas yang digunakan untuk instansiasi objek.

\subsection{Obyek dan Class}
Obyek (object) merupakan representasi dari entitas sebagai sarana pembungkusan karakteristik struktural
yang dapat disebut atribut dan karakteristik perilaku yang disebut operasi (operation/method). Atribut 
mempresentasikan karakteristik entitas yang menentukan keadaan suatu obyek jika menerima pesan.
Operasi tersebut dapat berupa prosedur maupun fungsi. Kelas merupakan deskripsi dari suatu obyek pada 
saat implementasi (coding).


\subsection{Penurunan Sifat}
Penurunan sifat (inheriatance) adalah kemampuan suatu obyek mewarisi sifat dari obyek yang lain. Kemampuan ini menghasilkan program yang efesien karena ada mekanisme pemakaian kembali (resauble) kode program. Pemograman yang dibuat dapat menggunakan fungsi yang ada dalam file DLL dengan mengirim parameter dan menerima balikan dari fungsi dan selama dapat mengikuti kesepakatan dalam pemangglan fungsi atau prosedure tersebut.


\section{Dynamic link library}
System operasi windows dapat menggunakan proses linker konvensional (proses linker secara statis) dengan file berinteraksi LIB dan dapat secara dinamis menggunakan dynamic link library (DLL).
proses linking fungsi dari DLL secara fisik tidak disalin dan digabung kedalam executable file tetapi tetap terpisah dan dipanggil oleh executable file (“client”) pada saat runtime.

\section{Perancangan Modul-modul Pengembangan}
\subsubsection{Perancangan Modul Pengiriman SMS}
Mekanisme pengiriman SMS yang digunakan adalah pengiriman melalui telepon selular. Telepon selular yang digunakan tersebut dihubungkan ke Komputer menggunakan kabel data. Ketika sinyal datang, program yang berada pada Komputer akan menginstruksikan telepon selular untuk mengirimkan pesan ke nomor telepon selular pemilik rumah.
mengenali instruksi ini sebagai instruksi AT Command untuk mengirim SMS. AT command merupakan instruksi-instruksi yang digunakan untuk mengendalikan telepon seluler atau modem GSM/GPRS yang dihubungkan ke Komputer.
\subsection{Perancangan Modul Dial-Up}
Modul dial merupakan modul didalam sebuah software isi ulang pulsa dengan metode dial atau call.Jika menggunakan handphone kita sering menggunakan metode dial ini ketika cek pulsa. Biasanya modul dial diawali dengan karakter bertanda bintang (*) 
dan diakhiri karakter tagar. Modul dial up membutuhkan modul mikrokontroler tambahan 
sebagai antarmuka antara mikrokontroler dan kabel telepon. Modul ini hanya dapat dihubungkan dengan 
sebuah mikrokontroler DT-51 MinSys.


\section{Pembuatan Modul-modul Pengembangan}
\subsection{Pembuatan Modul Pengiriman SMS}
Pengiriman SMS dilakukan dengan cara menghubungkan telepon selular ke PC menggunakan kabel data. 
Program yang digunakan untuk mengirimkan SMS menggunakan bahasa pemrograman Microsoft Visual Basic.NET 2003. Program ini membutuhkan library tambahan yang berisi class-class yang dapat digunakan untuk komunikasi ke telepon selular. Library yang digunakan adalah GSMComm yang dapat diunduh secara gratis dari internet. 
\subsection{Pembuatan Modul Dial-Up}
Mekanisme dial up dilakukan dengan menambahkan sebuah modul mikrokontroler ke dalam rangkaian mikrokontroler yang ada
Modul tambahan ini berfungsi sebagai antarmuka mikrokontroler ke kabel telepon atau ke pesawat telepon. Modul ini kompatibel penuh dengan mikrokontroler. 
Program untuk mengoperasikan rangkaian mikrokontroler ini dibangun dengan menggunakan bahasa assembler untuk mikrokontroler yang selanjutnya dikompilasi menjadi format Hexadesimal.


\section{Implementasi Modul}
\subsection{Implementasi Modul Pengiriman SMS}
Program yang dibuat berdasarkan algoritma pengiriman SMS dienkapsulasi menjadi sebuah modul. Penyisipan fungsi pemanggilan modul pengiriman SMS dilakukan sebelum program tersebut mengirimkan sinyal ke komputer server melalui jaringan nirkabel IEEE 802.11. Dengan demikian, diharapkan agar SMS diterima oleh pemilik rumah tidak lama setelah petugas keamanan mendapatkan sinyal. 
\subsection{Implementasi Modul Dial Up}
Modul dial up yang dibuat dengan bahasa assembler dienkapsulasi menjadi sebuah fungsi yang dapat digunakan dan dipanggil oleh program assembler lain. Pemanggilan fungsi dial up ini dilakukan setelah prosedur pengiriman data ke komputer melalui komunikasi serial.
\subsection{Menerima Pesan dalam Bentuk PDU}
Menerima pesan dalam bentuk PDU tidak hanya isi pesan saja, melainkan terdapat berbagai data di dalamnya seperti informasi mengenai pengirimnya (nomor telepon pengirim), SMSC, dan Waktu Pengiriman Pesan. Data yang masuk berupaHexa – Decimal Octets. SMSC yaitu menerangkan banyaknya informasi pengirim yang terdapat pada pesan yang digunakan oleh pengirim.
\subsection{Mengirim Pesan dalam Format PDU}
Pesan ditulis dalam format text akan di konversikan terlebih dahulu kedalam format PDU agar bisa di baca oleh HP. PDU, adalah proses yang akan memanggil modul konversi untuk merubah data dalam Format Text menjadi Format PDU.  Alur proses kirm SMS, masukan berupa Format Teks dikonversi ke bentuk Format PDU.

\section{Perencanaan Modul Sistem}
\subsection{Modul Terima SMS}
Modul ini berfungsi untuk menerima data PDU yang masuk dan menampung masing-masing bagian dari data tersebut.
input-an pada prosedur ini yaitu string data PDU yang belum dipisahkan sesuai dengan nama bagiannya, sedangkan pada output-nya
berupa string PDU yang sudah dipisahkan yang terdiri dari panjang nomor SMSC, tipe alamat
SMSC, nomor SMSC, Octet pertama dari pesan SMS-Deliver. Banyaknya suatu nomor berasal dari nomor pengirim, tipe alamat dari nomor pengirim, nomor telepon pengirim pesan, protokol identifier, data coding scheme, waktu pengiriman pesan, banyaknya
pesan yang dikirim dan isi pesan sesuai dengan nama bagiannya. 
\subsection{Modul Kirim SMS}
Modul kirim sms memiliki fungsi untuk menampung data yang akan dikirimkan yaitu dalam betuk format PDU, dimana dalam prosesnya akan memanggil modul konversi untuk merubah data dalam Format Text menjadi Format PDU. Alur proses kirm SMS, yakni dari masukan berupa Format Teks dikonversi ke bentuk Format PDU. 
\subsection{ Modul Konversi}
modul ini berfungsi untuk menerjemahkan data yang masuk dari modul terima sms  format PDU menjadi  text,  melakukan pengkonversian informasi dari srting hexa menjadi biner, biner menjadi decimal,decimal menjadi character. 
modul ini juga menerjemahkan data yang masuk  dari modul kirim sms format text menjadi PDU, mengembalikan posisi data PDU dari character sampai menjadi string hexa. 



\end{document}
