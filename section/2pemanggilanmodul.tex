

\section{Modularitas dan Portabilitas}
Modularitas dan portabilitas merupakan faktor pentig karena termasuk atribut kualitas perangkat lunak. Modularitas berasal dari kata modul, modul adalah bagian perangkat lunak yang besar yang dipecah menjadi bagian yang lebih kecil dengan memberi nama. Pengalamatan memori berbeda beda kemudian diintergrasikan untuk membentuk perangkat lunak yang dapat memenuhi kebutuhan dari suatu permasalahan

\section{Paradigma Berorientasi Obyek}
Paradigma berorientasi obyek adalah suatu cara mengorganisasikan perangkat lunak sebagai kumpulan obyek- obyek yang memiliki sifat (struktur data) dan perilaku (fungsi) yang saling berinteraksi melalui pesan (message). Konsep yang menjadi pilar paradigma berorientasi obyek adalah : abstraction, encapsulation, inheritance, polymorphism. Abstraksi dipresentasikan sebagai suatu kelas yang digunakan untuk instansiasi objek.

\subsection{Obyek dan Class}
Obyek (object) merupakan representasi dari entitas sebagai sarana pembungkusan karakteristik struktural
yang dapat disebut atribut dan karakteristik perilaku yang disebut operasi (operation/method). Atribut 
mempresentasikan karakteristik entitas yang menentukan keadaan suatu obyek jika menerima pesan.
Operasi tersebut dapat berupa prosedur maupun fungsi. Kelas merupakan deskripsi dari suatu obyek pada 
saat implementasi (coding).


\subsection{Penurunan Sifat}
Penurunan sifat (inheriatance) adalah kemampuan suatu obyek mewarisi sifat dari obyek yang lain. Kemampuan ini menghasilkan program yang efesien karena ada mekanisme pemakaian kembali (resauble) kode program. Pemograman yang dibuat dapat menggunakan fungsi yang ada dalam file DLL dengan mengirim parameter dan menerima balikan dari fungsi dan selama dapat mengikuti kesepakatan dalam pemangglan fungsi atau prosedure tersebut.


\section{Dynamic link library}
System operasi windows dapat menggunakan proses linker konvensional (proses linker secara statis) dengan file berinteraksi LIB dan dapat secara dinamis menggunakan dynamic link library (DLL).
proses linking fungsi dari DLL secara fisik tidak disalin dan digabung kedalam executable file tetapi tetap terpisah dan dipanggil oleh executable file (“client”) pada saat runtime.

\section{Perancangan Modul-modul Pengembangan}
\subsubsection{Perancangan Modul Pengiriman SMS}
Mekanisme pengiriman SMS yang digunakan adalah pengiriman melalui telepon selular. Telepon selular yang digunakan tersebut dihubungkan ke Komputer menggunakan kabel data. Ketika sinyal datang, program yang berada pada Komputer akan menginstruksikan telepon selular untuk mengirimkan pesan ke nomor telepon selular pemilik rumah.
mengenali instruksi ini sebagai instruksi AT Command untuk mengirim SMS. AT command merupakan instruksi-instruksi yang digunakan untuk mengendalikan telepon seluler atau modem GSM/GPRS yang dihubungkan ke Komputer.
\subsection{Perancangan Modul Dial-Up}
Modul dial merupakan modul didalam sebuah software isi ulang pulsa dengan metode dial atau call.Jika menggunakan handphone kita sering menggunakan metode dial ini ketika cek pulsa. Biasanya modul dial diawali dengan karakter bertanda bintang (*) 
dan diakhiri karakter tagar. Modul dial up membutuhkan modul mikrokontroler tambahan 
sebagai antarmuka antara mikrokontroler dan kabel telepon. Modul ini hanya dapat dihubungkan dengan 
sebuah mikrokontroler DT-51 MinSys.
\subsection{Dataset Modul Deteksi Plagiarisme}
Data yang digunakan pada modul ini berupa kode program mahasiswa dalam satu kelas. Kode program tersebut merupakan hasil pengerjaan tugas pemrograman yang diberikan oleh dosen Teknik Informatika ITS. Data yang diambil terdiri dari dua jenis kode program mahasiswa dalam satu kelas yang mengambil mata kuliah Pemrograman Berorientasi Objek (PBO). Dataset jenis pertama berjumlah 38 buah kode program dari 38 mahasiswa dalam satu kelas. Dataset jenis ke 2 berjumlah 21 buah kode program dari 21 mahasiswa dalam satu kelas. Pada dataset jenis kedua terdapat dua jenis file yaitu 21 file header (.h) dan 21 file yang berekstensi .cpp. Sehingga masing-masing file akan dihitung perfomanya pada tahap uji coba.
\subsection{Dataset Modul Student Feedback System}
Data-data yang digunakan pada modul ini berasal dari tugas mata kuliah Pemrograman Berorientasi Objek (PBO) Kelas C Tahun Ajaran 2013/2014 di Jurusan Teknik Informatika ITS. Dataset yang digunakan adalah tiga jenis dataset, yaitu dataset class Invoice, class Account, serta dataset kode sumber yang tidak mirip. Jumlah dataset pertama yaitu 31 buah kode program mahasiswa, jumlah dataset kedua adalah 31 buah mahasiswa, dan jumlah dataset ketiga adalah 9 mahasiswa.


\section{Pembuatan Modul-modul Pengembangan}
\subsection{Pembuatan Modul Pengiriman SMS}
Pengiriman SMS dilakukan dengan cara menghubungkan telepon selular ke PC menggunakan kabel data. 
Program yang digunakan untuk mengirimkan SMS menggunakan bahasa pemrograman Microsoft Visual Basic.NET 2003. Program ini membutuhkan library tambahan yang berisi class-class yang dapat digunakan untuk komunikasi ke telepon selular. Library yang digunakan adalah GSMComm yang dapat diunduh secara gratis dari internet. 
\subsection{Pembuatan Modul Dial-Up}
Mekanisme dial up dilakukan dengan menambahkan sebuah modul mikrokontroler ke dalam rangkaian mikrokontroler yang ada
Modul tambahan ini berfungsi sebagai antarmuka mikrokontroler ke kabel telepon atau ke pesawat telepon. Modul ini kompatibel penuh dengan mikrokontroler. 
Program untuk mengoperasikan rangkaian mikrokontroler ini dibangun dengan menggunakan bahasa assembler untuk mikrokontroler yang selanjutnya dikompilasi menjadi format Hexadesimal.


\section{Implementasi Modul}
\subsection{Implementasi Modul Pengiriman SMS}
Program yang dibuat berdasarkan algoritma pengiriman SMS dienkapsulasi menjadi sebuah modul. Penyisipan fungsi pemanggilan modul pengiriman SMS dilakukan sebelum program tersebut mengirimkan sinyal ke komputer server melalui jaringan nirkabel IEEE 802.11. Dengan demikian, diharapkan agar SMS diterima oleh pemilik rumah tidak lama setelah petugas keamanan mendapatkan sinyal. 
\subsection{Implementasi Modul Dial Up}
Modul dial up yang dibuat dengan bahasa assembler dienkapsulasi menjadi sebuah fungsi yang dapat digunakan dan dipanggil oleh program assembler lain. Pemanggilan fungsi dial up ini dilakukan setelah prosedur pengiriman data ke komputer melalui komunikasi serial.
\subsection{Menerima Pesan dalam Bentuk PDU}
Menerima pesan dalam bentuk PDU tidak hanya isi pesan saja, melainkan terdapat berbagai data di dalamnya seperti informasi mengenai pengirimnya (nomor telepon pengirim), SMSC, dan Waktu Pengiriman Pesan. Data yang masuk berupaHexa – Decimal Octets. SMSC yaitu menerangkan banyaknya informasi pengirim yang terdapat pada pesan yang digunakan oleh pengirim.
\subsection{Mengirim Pesan dalam Format PDU}
Pesan ditulis dalam format text akan di konversikan terlebih dahulu kedalam format PDU agar bisa di baca oleh HP. PDU, adalah proses yang akan memanggil modul konversi untuk merubah data dalam Format Text menjadi Format PDU.  Alur proses kirm SMS, masukan berupa Format Teks dikonversi ke bentuk Format PDU.
\subsection{Modul Form Pendaftaran Antrian}
Digunakan oleh konsumen untuk pendaftaran antrian. Berisi nomor antrian sebelumnya, waktu tunggu, pilihan untuk memilih jenis pelanggan (telpon/Flexi atau Speedy atau calon pelanggan), field untuk memasukkan nomor telpon/Flexi/selular, link menuju form Pendaftaran. Antrian 2 bagi calon pelanggan, pilihan jenis layanan berupa checkbox, tombol Daftar Antrian, tombol Batal.
\subsection{Modul Form Login Petugas}
Form login petugas digunakan untuk masuk ke dalam aplikasi untuk supervisor. Form login petugas ini berisi
field untuk mengisi username, password, dan nomor meja tempat bertugas. 
\subsection{Modul Form Input Pelayanan Konsumen}
Modul ini akan muncul setelah tombol Panggil Antrian Berikutnya dan tombol Panggil pada form CallNextCus ditekan. 
Dalam modul ini berisi data identitas konsumen, pilihan untuk masukan jenis produk berupa drop down list,
field untuk memasukkan data masalah dan solusi, tabel berisi data sejarah konsultasi konsumen sebelumnya, tombol Simpan, tombol
Batal, dan tombol Ubah Data Konsumen. 
\subsection{modul From Ubah Data Konsumen}
Modul ini akan muncul setelah tombol Ubah Data Konsumen ditekan. Dalam modul ini berisi data identitas konsumen yang ada dalam field-field dan bisa mengubahubah isinya, tombol Simpan akan mengubah jika ditekan menuju form Input Pelayanan Konsumen, tombol Batal, dan tombol Keluar akan mengubah jika ditekan menuju form Input Pelayanan Konsumen.
\subsection{Modul Form Display}
Modul form display ini berisi nomor antrian yang dipanggil, nomor meja petugas yang memanggil, dan daftar meja beserta petugas yang bertugas (jika petugas login maka namanya akan muncul dan jika logout namanya akan hilang. Nomor antrian dan nomor meja berubah setiap kali petugas menekan tombol Panggil Antrian Berikutnya pada form CallNextCus.
\subsection{Modul Form Data Konsumen}
Setelah proses login berhasil, secara default ditampilkan modul antrian. Digunakan untuk menampilkan daftar seluruh nama konsumen yang pernah datang ke Plasa. Modul ini menampilkan tabel yang berisi data data daftar nama konsumen berupa link menuju form Data Identitas Konsumen, alamat, dan nomor teleponnya. Di sebelah bawah, ditampilkan form  Pencarian data Konsumen.
\subsection{Pengujian Modul Student Feedback System}
Pengujian dilakukan dengan menilai akurasi pada penghitungan nilai kemiripan dua buah kode program, dilakukan sebanyak tiga kali menggunakan tiga buah dataset. Perhitungan kemiripan kode program dilakukan secara manual.
Pengujian kedua adalah penampilan rekomendasi kode program yang mirip. Sistem akan menampilkan kode program secara berdampingan, kode program yang sama akan diberi tanda khusus.
\subsection{Pengujian Modul Deteksi Plagiarisme}
Pengujian dilakukan dengan menilai akurasi pada penghitungan nilai yang memiliki dua buah kode program dan akurasi pengelompokan kode program dengan menggunakan jumlah cluster hasil penghitungan standar deviasi. Pengujian dilakukan menggunakan dua buah dataset yang telah dijelaskan sebelumnya. Pengujian pada penilaian akurasi pengelompokan kode program dilakukan dengan memberi label True dan False pada masing-masing anggota cluster. 

\section{Perencanaan Modul Sistem}
\subsection{Modul Terima SMS}
Modul ini berfungsi untuk menerima data PDU yang masuk dan menampung masing-masing bagian dari data tersebut.
input-an pada prosedur ini yaitu string data PDU yang belum dipisahkan sesuai dengan nama bagiannya, sedangkan pada output-nya
berupa string PDU yang sudah dipisahkan yang terdiri dari panjang nomor SMSC, tipe alamat
SMSC, nomor SMSC, Octet pertama dari pesan SMS-Deliver. Banyaknya suatu nomor berasal dari nomor pengirim, tipe alamat dari nomor pengirim, nomor telepon pengirim pesan, protokol identifier, data coding scheme, waktu pengiriman pesan, banyaknya
pesan yang dikirim dan isi pesan sesuai dengan nama bagiannya. 
\subsection{Modul Kirim SMS}
Modul kirim sms memiliki fungsi untuk menampung data yang akan dikirimkan yaitu dalam betuk format PDU, dimana dalam prosesnya akan memanggil modul konversi untuk merubah data dalam Format Text menjadi Format PDU. Alur proses kirm SMS, yakni dari masukan berupa Format Teks dikonversi ke bentuk Format PDU. 
\subsection{ Modul Konversi}
Modul ini berfungsi untuk menerjemahkan data yang masuk dari modul terima sms  format PDU menjadi  text,  melakukan pengkonversian informasi dari srting hexa menjadi biner, biner menjadi decimal,decimal menjadi character. Modul ini juga menerjemahkan data yang masuk  dari modul kirim sms format text menjadi PDU, mengembalikan posisi data PDU dari character sampai menjadi string hexa. 
\subsection{Modul Power Suply}
Modul ini berguna untuk memberikan daya yang diperlukan untuk menjalankan modul- modul tersebut. Pada power supply tersebut terdiri dari transformator, rectifier, filter dan voltage regulator. Pada perancangan power supply ini digunakan transformator dengan tipe step down atau penurun tegangan. Transformator ini digunakan untuk menurunkan tegangan dari PLN yang sebesar 220 V menjadi besar tegangan yang diinginkan, dimana pada perancangan ini tegangan yang diinginkan 5 V DC. Modul power supply berguna untuk memberi tegangan yang dibutuhkan oleh suatu alat untuk bekerja. Pada rancangan ini digunakan hanya satu buah power supply yang terletak pada modul simulasi.
\subsection{Mikrokontroler}
Mikrokontroler memiliki kemampuan untuk menyimpan dan menjalankan suatu program yang bertujuan sebagai kontrol.
Dalam sebuah IC mikrokontroler sudah terintegrasi ROM, RAM, EPROM, serial interface dan paralel
interface, timer, interrupt controller, konverter analog ke digital (ADC). Rangkaian tersebut terdapat dalam level chip atau biasa disebut single chip microcomputer.Mikrokontroler memiliki beberapa keunggulan yaitu Kehandalan tinggi dan kemudahan integrasi dengan komponen lain ,Mikrokontroler dapat mempermudah perbaikan maupun update system karena kontrolnya tidak berupa fisik,
melainkan software yang disimpan dalam Flash ROM ataupun EEPROM yang mudah diisi ulang dan Banyak kemampuan yang ditambahkan dalam sebuah chip tunggal, misalnya Timer, Serial interface, ADC,DAC, EEPROM, RTC, ISP.
\subsection{LIGHT EMITTING DIODE (LED)}
LED adalah sebuah dioda yang bisa menghasilkan cahaya apabila diberi tegangan foward bias. Ketika LED diberi tegangan foward bias, elektron bagian negatif pada dioda akan berpasangan dengan hole dari bagian positif. 
Dalam proses penyatuan tersebut, akan terjadi pemancaran energi dalam bentuk panas dan cahaya. Pada dioda Silikon dan Germanium, sebagian besar energi yang timbul dalam bentuk panas. Bahan yang biasanya digunakan untuk LED seperti Galium Arsen Phospida (Ga-AsP) dan Galium Phospida (GaP), energi yang dipancarkan dalam bentuk energi foton atau cahaya.
\subsection{Modul Interface RS-232}
Perancangan pengantrian meja pada restoran secara wireless  menggunakan  interface serial RS-232. Interface serial RS-232 berfungsi untuk mengirimkan data-data kedalam kode biner. Standar RS-232 ini dikembangkan oleh Electronics Industry Association and the Telecommunications Industry. Modul ini berguna sebagai penghubung antara modul mikrokontroler dengan komputer. Modul jaringan  dibutuhkan agar seluruh perintah komputer dapat diubah menjadi sinyal kontrol bagi mikrokontroler. 


\section{Hasil dan Evaluasi}
\subsection{Hasil dan Evaluasi Modul Deteksi Plagiarisme}
Skenario implementasi satu adalah penghitungan akurasi sistem pendeteksi plagiarisme kode program dengan menggunakan dataset jenis pertama yaitu dataset yang memiliki tingkat kemiripan antar kode program yang relatif kecil. Implementasi dilakukan dengan menghitung akurasi pada penghitungan nilai similarity antar kode program, pengelompokan kode program berdasarkan tingkat kemiripannya, dan penentuan jumlah cluster terbaik yang akan dipilih.Terdapat 38 kode program pada dataset jenis pertama yang memiliki total 703 nilai similarity.
\subsection{Hasil pengujian dan analisis}
Pengujian modul power supply bertujuan  untuk mengetahui hasil output dari power supply yang telah dirancang apakah telah sesuai dengan yang diinginkan dan untuk pengujian tersebut dilakukan dengan dua cara. Pertama yaitu pengujian hasil output dari power supply dengan tanpa beban dengan cara mengukur langsung pada output dari power supply  dan yang kedua yaitu pengujian hasil output dari power supply dengan beban dengan cara menyambung modul power supply tersebut dengan beberapa buah resistor dengan tahanan yang berbeda.

