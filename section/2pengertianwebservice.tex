\documentclass[12pt]{journal}

\begin{document}

\title{Pengertian Web Service}
\maketitle

\section{Definisi}

\subsection{Hartati Deviana}
	Web service adalah konsep baru dalam sistem terdistribusi melalui Web yang menggunakan teknologi XML, dengan standar protokol  HTTP dan SOAP. Konsep teknologi Web service muncul untuk mendukung sistem terdistribusi yang memiliki infrastruktur yang berbeda. Karena Web service menggunakan XML, maka teknologi ini dapat mendukung integrasi berbagai platform sistem dan aplikasi, baik infrastruktur intranet dan ekstranet. Dalam penelitian ini akan disusun oleh sebuah sistem informasi dengan menggunakan teknologi Web service menggunakan PHP dan NuSOAP yang diimplementasikan pada sistem pengelolaan distribusi barang di sebuah apotek yang memiliki beberapa cabang. Penelitian ini menghasilkan sistem informasi yang mampu mengintegrasikan aplikasi dan platform dari seluruh cabang.
	
	Web service merupakan suatu komponen software yang merupakan selfcontaining, aplikasi modular self-describing yang dapat dipublikasikan, dialokasikan, dan dilaksanakan pada web. Web service adalah teknologi yang mengubah kemampuan internet dengan menambahkan kemampuan transactional web, yaitu kemampuan web untuk saling berkomunikasi dengan pola program-to-program (P2P). Fokus web selama ini didominasi oleh komunikasi program-to-user dengan interaksi business-to-consumer (B2C), sedangkan transactional web akan didominasi oleh program-to-program dengan interaksi business-to-business\cite{deviana2013penerapan}.


\subsection{Richards Robert}

Web service merupakan salah satu implementasi dari teknologi XML (Extensible Markup Language) pada proses pertukaran antara (data exchange) platform yang berbeda sercara berbeda.

\textit{"A Web service is a software system designed to support interoperable machine-to-machine interaction over a network. It has an interface described in a machine-processable format(specifically WSDL).Other systems interact with the Web service in a manner prescribed by its description using SOAP messages, typically conveyed using HTTP with an XML seriali zation in conjunction with other Web-related standards"}.

Menurut Richards, web service dapat digunakan untuk berkomunikasi antara mesin satu dengan mesin yang lain melalui interface perantara yang umumnya berupa WSDL(Web Service Definition Language), layanan ini biasa bekerja pada protokol HTTP dengan bentuk response dan request berupa SOAP messange. SOAP (Simple Object Access Protocol) adalah standar untuk bertukar pesan-pesan berbasis XML melalui jaringan komputer atau sebuah jalan untuk program yang berjalan pada suatu sistem operasi (OS) untuk berkomunikasi dengan program pada OS yang sama maupun berbeda dengan menggunakan HTTP dan XML sebagai mekanisme untuk pertukaran data. Format SOAP message adalah mengikuti frame XML yang terstandarisasi\cite{ihya2011pembuatan}. 

\subsection{Chen, Xi dan Zheng, Zibin dan Yu, Qi dan Lyu, Michael R}

Web Service adalah komponen perangkat lunak yang terintegrasi untuk mendukung interaksi antar mesin dengan mesin yang lainya ( komputer ) antar jaringan , layanan web service telah banyak digunakan untuk membangun suatu aplikasi yang berorientasi dengan layanan industri dan akademisi dalam beberapa tahun trakhir , jumlah layanan web yang tersedia untuk umum terus meningkat di internet , Namun  ini menyulitkan pengguna untuk memilih layanan yang tepat di antara banyaknya layanan web services\cite{chen2014web}.

\subsection{Witono, Timotius and Susanto, Raphael}

Pengertian sederhana web service adalah aplikasi yang dibuat agar dapat dipanggil atau diakses oleh aplikasi lain melalui internet atau intranet dengan menggunakan XML sebagai format pengiriman pesan. Web service digunakan saat pengguna akan mentransformasi sebuah logik atau sebuah class dan objek yang terpisah dalam satu ruang lingkup yang menjadi satu, sehingga tingkat keamanan dapat ditangani dengan baik\cite{witono201511}.



\subsection{Kurniawan, Erick}

Web Service adalah layanan yang tersedia di Internet. Web Service menggunakan format standar XML untuk pengiriman pesannya. Web Services juga tidak terikat kepada bahasa pemrograman atau sistem operasi tertentu (Ethan Cerami, 2002). Web Services adalah antar muka yang mendeskripsikan koleksi yang dapat diakses dalam jaringan menggunakan format standar XML untuk pertukaran pesan. Web Services mengerjakan tugas yang spesifik. Web Services dideskripsikan menggunakan format standar notasi XML yang disebut services description (Gottschalk, 2002)\cite{chen2014web}.

\subsection{Sarbini, Riska Nurtantyo}

Web service merupakan satuan diskrit dari fungsionalitas programatis yang diekspos 
kepada client via protokol komunikasi, dan format data standar bernama HTTP dan 
XML. Protokol ini mengatasi masalah komunikasi lintas internet dan lintas 
firewall tanpa beralih ke solusi superior yang memerlukan port-port komunikasi 
tambahan yang harus dibuka untuk akses eksternal. Dikarenakan web service mamiliki fungsi untuk menformat dan menguraikan pesan XML\cite{sarbini2015pengembangan}. 

\subsection{M. Shalahuddin dan Rosa A.S.}

Web Service merupakan suatu sistem yang menyediakan pelayanan yang dibutuhkan oleh klien. Klien dari web service tidak hanya berupa aplikasi web, tetapi juga bisa sebuah aplikasi enterprise. Jadi web service tidak sama dengan web server, bahkan sebuah aplikasi web pada web server dapat menjadi klien dari web service\cite{inayah2014aplikasi}.

\subsection{Gottschalk (2002)}

Web Service adalah teknologi yang mengubah kemampuan internet dengan menambahkan kemampuan transactional web, yaitu kemampuan web untuk saling komunikasi dengan pola program to program (P2P). Fokus web selama ini didominasi oleh komunikasi program to user dengan interaksi business to costumer (B2C), sedangkan stransactional web akan didominasi oleh P2P dengan interaksi business to business\cite{fauziah2014aplikasi}.

\subsection{Jurnal Masyarakat Informatika}

web service adalah antarmuka yang mendeskripsikan sekumpulan operasi yang dapat diakses dalam sebuah jaringan melalui pesan XML yang telah distandartkan.xml iyalah bahasa markup yang sudah terintregrasi dengan web service. W3C mendefinisikan web service sebagai sebuah sistem perangkat lunak yang dirancang untuk mendukung inter operasi mesin ke mesin di sebuah jaringan.  Web service merupakan komponen perangkat lunak loosely coupled, dapat diguna ulang, membungkus fungsionalitas diskret, didistribusikan, dan diakses secara programatik melalui protokol internet standart . dan sangat di di perhatikan di bidang informatika \cite{saputra2integrasi}.

\section{Manfaat}

Layanan web memungkinkan penyedia layanan dan vendor untuk menjual layanan mereka dengan memublikasikannya
Yang di akses melalui World Wide Web.
Manfaat dari layanan web kita dapat berbagi data walaupun memiliki jarak yang jauh dan dapat mempermudah membagi suatu data dalam sebuah pekerjaan
interoperabilitas. Manfaat ini berasal dari antarmuka XML standar dan deskripsi akses
diberikan oleh WSDL (Web Services Description Language). Deskripsi WSDL sangat membantu dalam perusahaan
integrasi aplikasi, integrasi B2B (menyelesaikan tantangan antara bisnis dan bisnis partner, seperti customer, supplier, bank, dan jasa transportasi ) \cite{ferris2003web}.


\section{Arsitektur RESTful Web services}

Berikut merupakan langkah-langkah yang dilakukan dalam model dasar RESTful Web services (HostBridge, 2009):
1. Query Request Provider melalui HTTP dengan menggunakan URI (Uniform Resource Identifier). Request menggunakan methods (metode) HTTP untuk menentukan apakah request tersebut dimaksudkan untuk Create (menciptakan), Read (membaca), Update (memperbarui), atau Delete (menghapus) data.
2. HostBridge mengembalikan sebuah dokumen dalam bentuk XML untuk Requester (pemohon) dengan CICS data enclosed\cite{arsana2014rancang}.

\section{Pengertian Web Service}

Web Service dapat diartikan sebuah antar muka atau dalam bahas inggris yaitu interface  yang berarti menggambarkan sebuah sekumpulan operasi-operasi yang kemudian dapat diakses melalui jaringan, misalnya internet dalam bentuk pesan “Extensible Markup Language (XML)”. Web Service juga menyediakan standar komunikasi dalam berbagai software yang berbeda-beda, dan dapat berjalan di berbagai platform maupun framework\cite{hartono2013aplikasi}. 





\end{document}
