\documentclass[12pt,a4paper]{article}
\usepackage[left=3.00cm, right=2.00cm, bottom=2.00cm, top=3.00cm]{geometry}
\linespread{1.5}
\begin{document}
\title{definisi Dekorator, Contoh Kode dan Fungsi}
\maketitle

\begin{itemize}

\item
NAMA KELOMPOK 4\\
Ajis Trigunawan			1164031\\
Alimu Dzul Ikroom		1164032\\
Muhammad Hanafi			1164092\\
Riki Karnovi			1164052\\
Yoga Sakti Hadi P		1164059\\

\end{itemize}

\section{Definisi Dekorator, Kode dan Fungsi}

\subsection{Definisi Dekorator}
Python merupakan Bahasa pemrograman dengan fitur canggih dan ekspresif., salah satunya adalah dekorator. Dalam konteks desain, dekorator secara dinamis mengubah fungsi, metode atau pun kelas tanpa harus menggunakan subclass secara langsung. Ini merupakan hal yang ideal ketika kita perlu memngembangkan fungsi yang tidak ingin kita ubah. Kita dapat mengimplementasikan pola dekorator di mana saja dan di Python tentunya  dan Python memfasilitasi penerapannya dengan menyediakan lebih banyak fitur dan sintaksis yang ekspresif untuk itu.

Python menawarkan fitur dekorator sejak versi 2.4. Secara sederhana, dekorator adalah pabrik fungsi. Mereka memungkinkan kita untuk mengubah fungsi Python biasa menjadi fungsi yang berfungsi seperti MATLAB®. Dekorator Python menerima fungsi tepat sebelum dimuat ke dalam ruang kerja saat ini. Dekorator dapat memanipulasi fungsi dengan cara yang sewenang-wenang. Dekorator fungsi memodifikasi setiap fungsi yang diterjemahkan oleh OMPC. Kami menggunakan dekorator untuk meniru keberadaan variabel nargin / nargout, untuk memungkinkan penugasan ke variabel baru, dan untuk menerapkan pengembalian tersirat.
Decorators sesuai dengan namanya secara bahasa memiliki arti yaitu pendekorasi atau penghias. Jika di kaitkan dengan python maka decorators memiliki arti yaitu adalah sebuah method yang ‘mengambil’ method lain dan menambahkan beberapa fungsi kepada method tersebut tanpa harus melakukan modifikasi. Bahasa simplenya adalah method yang melakukan ‘pendekorasian/penghiasan’ kepada method lainnya.

Sebelumnya kita harus memahami bahwa semua yang ada di Python adalah objek (termasuk kelas). Nama pengenal, seperti variabel yang kita deklarasikan merujuk kepada objek tersebut. Begitu juga dengan fungsi. Fungsi adalah termasuk objek juga. Satu objek bisa memiliki banyak pengenal (identifier) yang merujuk kepadanya dengan contoh kodenya:

\begin{verbatim}
def first(msg):
    print(msg)

first("Hello")
second = first
second("Hello")
\end{verbatim}

Dekorator adalah fungsi yang mengambil fungsi lain dan memperluas perilaku fungsi yang terakhir tanpa secara eksplisit memodifikasinya, "Dekorator" yang kita dimaksud adalah kepedulian terhadap Python tidak persis sama dengan DecoratorPattern yang dijelaskan. Dekorator Python adalah perubahan spesifik pada sintaks Python yang memungkinkan kita mengubah fungsi dan metode dengan lebih mudah (dan mungkin kelas dalam versi yang akan datang). Ini mendukung aplikasi yang lebih mudah dibaca dari DecoratorPattern tetapi juga kegunaan lain juga.

fungsi dari decorators juga adalah menambahkan atau merubah beberapa fungsionalitas ke fungsi asli. Hal ini mirip dengan pengemasan bungkus kado. Decorators bertindak sebagai bungkusnya. Objek yang didekorasi atau isi kado tidak berubah. Akan tetapi, ketika dibungkus, akan terlihat lebih menarik (karena didekorasi). Umumnya, kita mendekorasi fungsi dan menyimpannya ke variable.

Cara paling mudah untuk menentukan rute dalam aplikasi flask adalah melalui decorator app.route oleh aplikasi instance. Dibawah ini adalah contoh kodenya:\\

@app.route(‘/’)\\
Def index():\\
      Return ‘ <h1> Hello World!</h1> ’\\

Catatan:

Dekorator adalah fitur standar dari bahasa python. Penggunaan umum dekorator adalah untuk mendaftarkan fungsi sebagai fungsi handler yang akan dipanggil ketika peristiwa tertentu terjadi

Panduan untuk dekorator fungsi Python
Python kaya dengan fitur canggih dan sintaksis ekspresif. Salah satu favorit saya adalah dekorator. Dalam konteks pola desain, dekorator mengubah fungsi suatu ke fungsi, metode atau kelas secara dinamis tanpa harus menggunakan subclass secara langsung. Ini sangat ideal ketika Anda perlu memperluas fungsi fungsi yang tidak ingin Anda ubah. Kita dapat mengimplementasikan pola dekorator di mana saja, tetapi Python memfasilitasi penerapannya dengan menyediakan lebih banyak fitur dan sintaksis yang ekspresif untuk itu.

\subsection{Jenis-Jenis Fungsi Dekorator}


Dekorator fungsi adalah sejenis deklarasi runtime tentang fungsi yang definisinya mengikuti. dekorator dikodekan pada baris tepat sebelum pernyataan def yang mendefinisikan fungsi atau metode, dan ini terdiri dari symbol @ yang di ikuti oleh referensi ke fungsi metafungsi (atau objek callable lain) yang mengelola fungsi lain.



@Decorator

       Class C: … C = \#Decorator(C\\
            X = C()\\
            Y = C() \#Overwrites x!\\
            
            

\end{document}
