\documentclass[12pt,a4paper]{article}
\usepackage[left=3.00cm, right=2.00cm, bottom=2.00cm, top=3.00cm]{geometry}
\linespread{1.5}
\begin{document}
\title{FUNGSI PYTHON}
\maketitle

\begin{itemize}
\item
NAMA KELOMPOK 4\\
Ajis Trigunawan			1164031\\
Alimu Dzul Ikroom		1164032\\
Muhammad Hanafi			1164092\\
Riki Karnovi			1164052\\
Yoga Sakti Hadi P		1164059\\
\end{itemize}

\section{Definisi Dekorator, Kode dan Fungsi}

\subsection{Definisi Dekorator}
Python merupakan Bahasa pemrograman dengan fitur canggih dan ekspresif., salah satunya adalah dekorator. Dalam konteks desain, dekorator secara dinamis mengubah fungsi, metode atau pun kelas tanpa harus menggunakan subclass secara langsung. Ini merupakan hal yang ideal ketika kita perlu memngembangkan fungsi yang tidak ingin kita ubah. Kita dapat mengimplementasikan pola dekorator di mana saja dan di Python tentunya  dan Python memfasilitasi penerapannya dengan menyediakan lebih banyak fitur dan sintaksis yang ekspresif untuk itu.


\end{document}