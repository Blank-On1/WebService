\begin{itemize}
\item Ahmad Syafrizal Huda (1164062)
\item Annisa Fathoroni (1164067)
\item Puad Hamdani (1164084)
\item Rahmi Roza (1164085)
\item Tasya Wiendhyra (1164086)
\end{itemize}

\section{Definisi Flask GET}
Flask merupakan microframework yang dibangun dengan menggunakan bahasa pemrograman Python. Flask digunakan untuk me-develop sebuah aplikasi web. Flask merupakan microframework yang artinya flask membuat sebuah pengerjaan aplikasi web menjadi mudah dan simple karena dapat menjalankan sebuah web hanya dengan menggunakan 1 file Python. Flask membuat susunan kerja yang ringan, dan mudah tetapi juga dapat dikembangkan dengan mudah. Setiap data memiliki akses dengan berbagai HTTP Method seperti GET (Menerima data dalam bentuk array atau object) \cite{gunawan2018aplikasi}.

\subsection{Metode Flask GET}
Metode penyediaan perintah operasional untuk program  yang dikirimkan menggunakan protokol HTTP termasuk menyimpan program  di server; menghasilkan data meta di server, di mana data meta mencakup tabel pemetaan yang menghubungkan rentang waktu untuk program  ke rentang byte untuk program  mentransmisikan data meta dan tabel pemetaan ke klien yang terkait dengan server, menghasilkan dan mentransmisikan perintah GET HTTP dari klien  ke server sebagai fungsi dari perintah operasional yang diinginkan; dan memilih I-frame yang tepat di server dan mengirimkan I-frame ke klien sebagai tanggapan atas perintah HTTP GET \cite{xu2006method}. 