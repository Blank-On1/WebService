\begin{itemize}
\item Ahmad Syafrizal Huda (1164062)
\item Annisa Fathoroni (1164067)
\item Puad Hamdani (1164084)
\item Rahmi Roza (1164085)
\item Tasya Wiendhyra (1164086)
\end{itemize}

\section{Definisi Flask GET}
Flask merupakan microframework yang dibangun dengan menggunakan bahasa pemrograman Python. Flask digunakan untuk me-develop sebuah aplikasi web. Flask merupakan microframework yang artinya flask membuat sebuah pengerjaan aplikasi web menjadi mudah dan simple karena dapat menjalankan sebuah web hanya dengan menggunakan 1 file Python. Flask membuat susunan kerja yang ringan, dan mudah tetapi juga dapat dikembangkan dengan mudah. Setiap data memiliki akses dengan berbagai HTTP Method seperti GET (Menerima data dalam bentuk array atau object) \cite{gunawan2018aplikasi}.

\subsection{Metode Flask GET}
Metode penyediaan perintah operasional untuk program  yang dikirimkan menggunakan protokol HTTP termasuk menyimpan program  di server; menghasilkan data meta di server, di mana data meta mencakup tabel pemetaan yang menghubungkan rentang waktu untuk program  ke rentang byte untuk program  mentransmisikan data meta dan tabel pemetaan ke klien yang terkait dengan server, menghasilkan dan mentransmisikan perintah GET HTTP dari klien  ke server sebagai fungsi dari perintah operasional yang diinginkan; dan memilih I-frame yang tepat di server dan mengirimkan I-frame ke klien sebagai tanggapan atas perintah HTTP GET \cite{xu2006method}.

\subsection{Sistem Terintegrasi Berbasis Ajax untuk Pengelolaan Data Bencana Alam di Indonesia}
Metode GET dan POST merupakan objek XHR yang sama bekerja sebagai standar HTTP request. Menggunakan salah satu, baik metode GET atau POST dapat digunakan untuk melakukan permintaan data dan menerima tanggapan dari server dengan format standar. Format standar yang dapat diterima dari server adalah XML, JSON (Javascript Object Notation), dan teks \cite{prasetyo2007sistem}.

\subsection{Sistem Terintegrasi Berbasis Ajax untuk Pengelolaan Data Bencana Alam di Indonesia}
Objek XHR (XMLHttpRequest) adalah inti dari Ajax angine. XHR merupakan objek yang memberikan kemampuan sebuah halaman untuk mendapatkan data (menggunakan metode GET) atau mengirim data (menggunakan metode POST) dari server yang prosesnya terjadi di belakang layar, itu berarti refresh browser tidak diperlukan sepanjang proses ini. Hal inilah yang menjadi factor kunci dalam memberikan kelebihan aplikasi kepada user. User tidak perlu mengetahui proses sehingga dapat focus dengan pekerjaan yang dilakukan \cite{prasetyo2007sistem}. 

\subsection{Contoh GET Request sederhana pada Flask Framework}
Contoh  GET request Sederhana

Berikut ini merupakan contoh skrip dari GET request :

@app.route('/a-get-request')
def get_request():
bar = request.args.get('foo', 'bar')
return 'A simple Flask request where foo is %s' % bar

Ini merupakan contoh sederhana dari apa GET request pada flask. Pada skrip ini hanya dilakukan pemeriksaan apakah query URL memiliki argumen yang disbut foo. apabila iya, maka akan menampilkan ini pada tanggapan, sedangkan jika tidak maka defaultnya adalah bar. \cite{aggarwal2014flask}.
