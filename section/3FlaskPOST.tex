%KELOMPOK 2 Pemasukan
%\begin{enumerate}
%\item Imron Sumadireja
%\item Jesron Marudut Hatuan
%\item Lusia Violita
%\item Mhd Zulfikar Akram Nst
%\end{enumerate}

\section{Pengertian Flask}
\emph{Flask} adalah \emph{framework} aplikasi web mikro yang ditulis dalam bahasa Python dan berbasiskan \emph{toolkit Wekzeug} dan \emph{template engine Jinja2} dan berlisensi BSD. Flask dikatakan framework mikro dikarenakan Flask tidak menganggap atau mengharuskan pengembang komponen menggunakan alat atau pustaka tertentu. Flask tidak memiliki lapisan abstrak basis data, validasi form, dan komponen-komponen lainnya yang sudah dimiliki oleh pustaka-pustaka sebelumnya. Flask mendukung ekstensi yang dapat menambah fitur-fitur seperti layaknya mereka diimplmenetasikan di dalam Flask itu sendiri. Terdapat ekstensi untuk objek-relational mappers, validasi form, upload handlint, dan berbagai teknologi otentikasi terbuka serta peralatan yang berhubungan dengan framework secara umum\cite{solihin2016implementasi}.

\section{Flask API}
Flask API merupakan implementasi dari web API yang dapat dijelajahi menggunakan kerangka yang sudah disediakan Django REST. Django REST merupakan web framework sumber terbuka berbasis Python. Aplikasi web yang dibuat dengan Flask disimpan dalam satu berkas .py. Flask merupakan web framework yang sederhana namun dapat diperluas dengan beragam pustaka tambahan yang sesuai dengan kebutuhan penggunanya. Flask API menyimpan perintah-perintah dari gphoto library dan piggyphoto library\cite{computingaplikasi}. 

\section{Fungsi Flask}
Flask berfungsi sebagai pengganti PHP POST dan GET yang dimana pengendali pertukaran data dari HTML ke database. Flask juga menggantikan fungsi Apache sebagai webserver dimana flask berjalan di http://localhost:5000/.
Flask dipilih karena berjalan diatas bahasa pemrograman Python sehingga lebih mudah diintegrasikan untuk mengontrol Virtual Data Center. Flask juga mengatur keluar masuknya data ke MySql untuk menyimpan inrformasi dan data dari user\cite{alauddin2017implementasi}.