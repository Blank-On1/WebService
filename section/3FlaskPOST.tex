%KELOMPOK 2 Pemasukan
%\begin{enumerate}
%\item Imron Sumadireja
%\item Jesron Marudut Hatuan
%\item Lusia Violita
%\item Mhd Zulfikar Akram Nst
%\end{enumerate}

\section{Pengertian Flask}
\emph{Flask} adalah \emph{framework} aplikasi web mikro yang ditulis dalam bahasa Python dan berbasiskan \emph{toolkit Wekzeug} dan \emph{template engine Jinja2} dan berlisensi BSD. Flask dikatakan framework mikro dikarenakan Flask tidak menganggap atau mengharuskan pengembang komponen menggunakan alat atau pustaka tertentu. Flask tidak memiliki lapisan abstrak basis data, validasi form, dan komponen-komponen lainnya yang sudah dimiliki oleh pustaka-pustaka sebelumnya. Flask mendukung ekstensi yang dapat menambah fitur-fitur seperti layaknya mereka diimplmenetasikan di dalam Flask itu sendiri. Terdapat ekstensi untuk objek-relational mappers, validasi form, upload handlint, dan berbagai teknologi otentikasi terbuka serta peralatan yang berhubungan dengan framework secara umum\cite{solihin2016implementasi}.

\section{Flask API}
Flask API merupakan implementasi dari web API yang dapat dijelajahi menggunakan kerangka yang sudah disediakan Django REST. Django REST merupakan web framework sumber terbuka berbasis Python. Aplikasi web yang dibuat dengan Flask disimpan dalam satu berkas .py. Flask merupakan web framework yang sederhana namun dapat diperluas dengan beragam pustaka tambahan yang sesuai dengan kebutuhan penggunanya. Flask API menyimpan perintah-perintah dari gphoto library dan piggyphoto library\cite{computingaplikasi}.

\section{Fungsi Flask}
Flask berfungsi sebagai pengganti PHP POST dan GET yang dimana pengendali pertukaran data dari HTML ke database. Flask juga menggantikan fungsi Apache sebagai webserver dimana flask berjalan di http://localhost:5000/.
Flask dipilih karena berjalan diatas bahasa pemrograman Python sehingga lebih mudah diintegrasikan untuk mengontrol Virtual Data Center. Flask juga mengatur keluar masuknya data ke MySql untuk menyimpan inrformasi dan data dari user\cite{alauddin2017implementasi}.

\subsection{Fungsi Flask}
Flask juga mengolah data untuk ditampilkan ke web untuk mempermudah user dalam memahami data tersebut. Flask sendiri merupakan kerangka kerja berbasis bahasa pemograman python yang sangat cocok untuk diintegrasikan bersama aplikasi berbasis cloud. Selain berfungsi sebagai penghubung antar Virtual Data Center dan web, Flask juga bertanggung jawab atas pengintegrasian database yang dimana disini dipilih MySql sebagai database\cite{alauddin2017implementasi}.


\section{Fitur Flask}
Flask mendukung ekstensi. Ekstensi yang tersedia untuk objek relasional seperti pembuatan peta,
validasi formulir, upload handling, teknologi otentikasi, dan lain sebagainya. Berikut ini adalah beberapa
fitur yang dimiliki oleh flask, diantaranya :
\begin{enumerate}
  \item Integrated supports for unit testing
  \item Uses Jinja2 templating
  \item Support for secure cookies
  \item Extensive documentation
  \item Google app engine compatibility
  \item Restful request dispatching
  \item Unicode based\cite{lokhande2015efficient}.
\end{enumerate}

\section{Aplikasi dan Merekuest Konteks}
Ketika flask menerima permintaan dari klien, maka perlu membuat beberapa objek tersedia untuk fungsi tampilan yang akan menangani flasknya. Contoh yang baik adalah pada objek permintaan, yang merangkum permintaan HTTP yang dikirim oleh klien. Cara yang jelas di mana Flask bisa memberikan fungsi tampilan akses ke objek permintaan adalah dengan mengirimkannya sebagai argumen, tapi itu akan membutuhkan setiap fungsi tampilan tunggal dalam aplikasi untuk memiliki argumen tambahan\cite{grinberg2018flask}.

\section{Implementasi Flask Post - Python}
Untuk demonstrasi, aplikasi demo posting blog menyajikan implementasi praktis API. Aplikasi ini menggunakan kerangka web Flask-Python. Selain itu, bagian ini memperkenalkan teknologi yang digunakan dan menyebutkan alasan mengapa mereka dipilih untuk proyek tersebut. Selain itu, ia menyediakan langkah demi langkah pendekatan untuk menunjukkan bagaimana sumber daya dibagi dari sumber data persisten. Oleh karena itu, API aplikasi mendasarkan prinsip REST\cite{alemu2014rest}.

\subsection{Tiga Alasan memilih Flask untuk Aplikasi Demo}
Ada tiga alasan untuk memilih Flask untuk aplikasi demo ini. Pertama salah satunya adalah kesederhanaannya. Dalam Flask, seseorang seharusnya tidak perlu tahu segalanya dari awal. Kerangka kerja bisa dipelajari sambil berkembang. Ini memiliki awal yang cepat dokumentasi yang memandu pengembang untuk membuka dan menjalankan aplikasi dengan dasar fungsionalitas. Alasan kedua adalah keterbukaannya. Seperti halnya Python, Flask perangkat lunak didistribusikan di bawah lisensi sumber terbuka permisif. Ini membuat lebih mudah diakses dan canggih. Akhirnya, itu adalah open source yang terdokumentasi dengan baik
kerangka. Ini memiliki tutorial rinci selain dokumentasi mulai cepat. Selain itu, ia menyediakan dokumentasi terbaru untuk setiap versi baru kerangka. Oleh karena itu, alasan-alasan ini membuat kerangka web Flask lebih disukai aplikasi demo layanan web blogging\cite{alemu2014rest}.
