%Resume GET (parameter GET, cara penggunaan dan kode) Kelompok 3 D4TI2B
%\begin{enumerate}
%\Fikri aldi nugraha                  1164038
%\Nur Arkhamia Batubara               1164049 
%\Miftahul Hasanah                    1164046 
%\Si Made Angga Dwitya P              1164053 
%\Widary Anggraini Mindo V Siahaan    1164057
%\end{enumerate}

\section{Pengenalan Method GET Pada HTTP}
HTTP mendefinisikan seperangkat metode permintaan untuk menunjukkan tindakan yang diinginkan yang akan dilakukan untuk sumber daya tertentu.
Meskipun mereka juga bisa menjadi kata benda, metode permintaan ini kadang-kadang disebut sebagai verba HTTP. Masing-masing menerapkan semantik yang berbeda, namun beberapa fitur umum digunakan bersama oleh mereka: mis. Metode permintaan dapat berupa safe, idempotent, atau cacheable. 
Salahsatu metode permintaan yang digunakan dalam Http adalah GET, dimana GET ini digunakan untuk meminta representasi sumber atau menampilkan data/nilai pada url yang nantinya akan ditampung oleh action.
