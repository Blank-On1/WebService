%Resume GET (parameter GET, cara penggunaan dan kode) Kelompok 3 D4TI2B
%\begin{enumerate}
%\Fikri aldi nugraha                  1164038
%\Nur Arkhamia Batubara               1164049 
%\Miftahul Hasanah                    1164046 
%\Si Made Angga Dwitya P              1164053 
%\Widary Anggraini Mindo V Siahaan    1164057
%\end{enumerate}

\section{Pengenalan Method GET Pada HTTP}
Http adalah protokol permintaan-jawaban (request-reply). Client akan mengawali koneksi dengan server dengan mengirimkan permintaan 
dengan nama dokumen yang client inginkan dan kemudian server mengirimkannya kembali dengan normal, termasuk dokumen yang diminta. http 
juga mengizinkan client untuk mengirimkan data yang diminta pengguna ke server. Http juga mengizinkan client untuk mengirimkan data 
yang diminta pengguna ke server. 

Method GET disebut juga method HTTP yang sangat sederhana dan paling banyak digunakan untuk meminta resource dari server. Method get 
memiliki banyak fungsi yaitu salah satunya digunakan untuk mengirim data diatas server, walaupun demikian hal itu memiliki batasan-
batasan. Jumlah dari seluruh karakter yang dienkapsulasi kepada permintaan method GET yaitu terbatas, jadi dalam situasu dimana banyak 
terdapat data perlu dikirimkan ke server tidak semua pesan bisa disampaikan. 

HTTP mendefinisikan seperangkat metode permintaan untuk menunjukkan tindakan yang diinginkan yang akan dilakukan untuk sumber daya 
tertentu.
Meskipun mereka juga bisa menjadi kata benda, metode permintaan ini kadang-kadang disebut sebagai verba HTTP. Masing-masing menerapkan 
semantik yang berbeda, namun beberapa fitur umum digunakan bersama oleh mereka: mis. Metode permintaan dapat berupa safe, idempotent, 
atau cacheable. Salah satu metode permintaan yang digunakan dalam Http adalah GET, dimana GET ini digunakan untuk meminta representasi 
sumber atau menampilkan data/nilai pada url yang nantinya akan ditampung oleh action.

REST menggunakan protokol HTTP yang bersifat stateless. Perintah HTTP yang bisa digunakan adalah fungsi GET, POST, PUT atau 
DELETE. Hasil yang dikirimkan dari server biasanya dalam bentuk format XML atau JSON sederhana tanpa ada protokol pemaketan  
data,sehingga informasi yang diterima dapat jauh lebih mudah dibaca dan diparsing disisi client.

\section{Pengertian Method Get}
GET adalah operasi read-only yang sering digunakan untuk meminta informasi yang spesifik pada sebuah server dalam bentuk query. 
Karakteristik dari operasi GET adalah idempotent dan safe. Idempontent dapat diartikan sebanyak-banyaknya apapun operasi ini dilakukan 
dan 
hasilnya akan tgetap sama sedangkan, safe berarti ketika operasi ini diinvokasi tetap tidak mengubah state di server.
Biasanya GET sering ditemukan di HTML,PHP dah diterapkan juga pada Web Service.

Web sevice merupakan layanan yang diidentifikasi dengan URI (Uniform Resource Identifier) yang mengekspos fiturnya dengan melalui 
internet dan memanfaatkan protocol, Bahasa standar internet  serta diimplementasikan menggunakan internet stantar contohnya XML. Bahasa 
pemograman php bisa digunakan dalam web service dengan memanfaatkan method seperti GET dan POST, kemudian PHP juga memiliki fungsi 
untuk mendukung pengiriman maupun pengolahan data dalam format JSON. Selain itu  agar PHP dapat mengirimkan data dalam format JSON, PHP 
menggunakan fungsi json_encode.  

Dalam Bahasa PHP melakukan HTTP request ke halaman kemudian secara default adalah get request, data yang didapatkan dari web service 
dikirimkan dalam bentuk format standar misalnya XML atau JSON atau Java Object Notation. Get memiliki fungsi yang sama seperti POST 
digunakan untuk mengirimkan nilai atau value variabel ke file yang telah diatur. Perbedaan method GET dan method POST sangat kecil 
tetapi sangat terlihat dengan jelas.

Bahasa pemograman PHP dapat kita gunakan untuk membuat web service yang  berbasis kepada RPC (Remote Procedure Call) , kemudian kita 
menggunakan SOAP untuk memanggil method atau fungsi yang berbeda pada computer lain dengan menggunakan internet. Supaya client bisa 
mengetahui method yang tersedia, port, format data input output, dan keterangan lain maka harus  dideklarasikan dengan standar WSDL.    
Metode umum yang dapat digunakan oleh HTTP untuk dapat membaca data kedalam sebuha browser adalah dengan menggunakan metode get. 
Metodeget ini dirancang hanya dapat kita gunakan untuk membaca sebuah data namun dapat disayangkan secara praktek dapat digunakan juga untuk melewatkan data dengan menambahkan informasi pada sebuah URL. Maka dengan cara tersebut makametode get akan melewatkan data yang dapat dilihat oleh pengguna melalui alamat URL yang juga dapat kita simpan pada bookmark.

Metode GET pada HTTP/1.0 mempunyai batas maksimum parameter data sepanjang batas URL dengan batas ukuran maksimum 2 kb untuk browser 
saat in. HTTP/1.1 tidak memberikan batas maksimum untuk Uniform Resource Identifier (URI). Penyebutan yang lebih umum untuk URL. Metode 
lain yang dapat kita gunakan agar kita dapat  mengirimkan data yaitu kita dapat menggunakan  metode POST. Dimana metode  POST ini 
memiliki banyak kelebihan dibandingkan dengan  metode GET ,kelebihan yang dimaksud tersebut yaitu metode POST  memliki panjang 
parameter dan tidak  terbatas,  metode POST tidak dapat terlihat oleh pengguna  seperti manusia dan  kita juga tidak bias menyimpannya 
ke dalam bookmark.

\subsection{fungsi Method GET}
Fungsi GET secara teknis diproses lebih sederhana karena permintaan dikirimkan melalui alamat halaman (URL) dengan sistem 
penulisan secara berpasangan yaitu nama varibel dan nilainya, dan pemisahan variabel menggunakan karakter dan &, misalnya :
http://localhost/obattradisional/bagian_tanaman.php. Metode GET digunakan sehingga HTTP Client bisa mengambil informasi dari 
server dengan mengirimkan data melalui URI.
dan juga GET dapat digunakan untuk mengambil data dari server dan meminta sebuah respon dari resource yang spesifik.  alur dari 
metode GET akan menampilkan data/nilai pada URL, kemudian akan ditampung oleh action.Sehingga Untuk melakukan proses read pada 
pin tertentu, HTTP request yang dikirimkan harus menggunakan method GET dengan pola Request- URI.

\subsection{Karateristik dari Method GET}
Metode get juga dapat diartikan sebagai metode pengiriman data dengan menggunakan query berupa string, sehingga seluruh nilai dari form 
diitampilkan pada baris url/Address bar. Get juga dapat difungsikan sebagai penamaan (link) dalam sebuah website. Adapun beberapa 
karakteristiknya yang dimiliki oleh Get sehingga method ini bisa memudahkan user untuk melakukan pencarian data ataupun penamaan pada 
url yaitu sebagai berikut :
\begin{enumerate}
\item Variabel dapat terlihat pada url.
\item Dibatasi pada panjang string yaitu 2047 karakter.
\item Dapat memungkinkan pengunjung dapat langsung memasukkan nilai pada form variabel proses.
\end{enumerate}

\subsection{Keuntungan Penggunaan Method GET}
Keuntungan dari penggunaan get tersebut adalah permintaan URl dari permintaan Get dapat juga disimpan oleh beberapa browser.
Hal itu berarti bahwa user dapat dengan mudah menyimpan permintaan dan dapat mengakses setiap saat melalui proses yang terjadi setiap 
waktunya. Hal tersebut dapat membahayakan jika penyimpanan yang dilakukan secara fungsional bukan merupakan sesuatu yang diinginkan oleh 
pengguna atau user.

Get itu sendiri dapat juga digunakan dalam pengiriman data ke server, meskipun begitu hal-hal itu mempunyai batasan. 
Jumlah dari total karakter yang dapat diubah ke dalam permintaan get adalah sangat terbatas. Sehingga untuk situasi tersebut, dimana 
banyak data-data yang perlu dikirim  ke server, tidak semua bisa pesan-pesan itu dapat disampaikan. Oleh sebab itu Get benar-benar 
sangat dibatasi dan tidak bebas.

\subsection{Kelemahan Penggunaan Method GET}
Kelemahan method GET adalah nilai dari form dapat dilihat langsung didalam URL yang dikirimkan. Jika membuat form untuk data-
data yang  sensitif seperti password. Maka form  dengan method GET bukan suatu pilihan yang bagus. Form dengan method GET 
disarankan untuk form yang berfungsi untuk menampilkan data  yaitu dimana hasil isian dari form hanya digunakan untuk  
menampilkan data. method GET sebaiknya digunakan untuk form yang mengambil data dari database.

\section{Parameter Get}
Dalam hal penggunaannya method GET ini memiliki suatu parameter. Jika klien menggunakan protokol HTTP pada server web untuk meminta 
sumber daya tertentu, maka klien akan mengirimkan parameter GET tertentu ke server melalui URL yang diminta. Parameter ini merupakan 
pasangan nama dan nilai yang harus memiliki kesesuaian, yang disebut pasangan nama-nilai.
Nama dan nilai ini ditambahkan ke URL dengan tanda \"?\" dan memberi tahu server sumber daya mana yang dimaksud. Nama dan nilai selalu dipisahkan menggunakan tanda \"=\". Tidak hanya satu tetapi juga beberapa parameter serta seluruh daftar dapat dikirim ke server. Di sini, berbagai parameter dipisahkan menggunakan tanda \"&\".
\begin{verbatim}
http://www.domain.com/index.html*?name1=value1
http://www.domain.com/index.html*?name1=value1&name2=value2
\end{verbatim}

Mengambil dan mengubah Teks pada Label
Walaupun jarang terjadi atau dapat kita sebut sangat jarang terjadi tapi terkaang kita juga harus mengambil isi teks suatu label atau 
kita dapat mengganti teks pada sebuah label saat program tersebut sedang berjalan.
Untuk mengambil tekas pada label kita dapat menggunakan method Get. Sedangkan untuk menentukan teks pada label kita dapat menggunakan 
method set_text. 
Method get akan dihilangkan pada GTK+2.0 dan kita dapat menggantinya dengan menggunakan get_teks.

\section{Parameter Get}
Method “get”
Pada saat penggunaan method “get” dimana membolehkan input yang kita masukkan dipaparkan sama ada pada halaman yang sama atau halaman 
 yang berbeda/lainnya. Brikut merupakan contoh dari penggunaan sintaks mehod “get” yaitu”
Simpanah sintaks berikut dalam bentuk shah.asp
\<html>
\<body>
\<form action=”shah.asp” method=”get”>
Nama
\<input type=”submit” value=”Submit”>
\</form>
\<%
Dim fname
Fname=Request.QueryString(“fname”)
If fname<>”” Then
Response.Write(“Hai” & fname &  “!<br> />”)
Response.Write(“Apa khabar hari ini ?”)
End if
%>
\</body>
\</html>

Dari contoh diatas menunjukkan contoh kemasukan data. Dimana kita hanya perlu memasukkan nama anda pada ruangan input. Setelah itu kita 
akan mengklik button submit.

