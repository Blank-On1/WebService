%Resume Hello word di Flask

%Kelompok 2 D4 TI / 2B

%Alwan Suryansah				1164033 
%Dinda Ayu Pratiwi				1164034
%Kurnia Sandi					1164042
%Teduh Sanubari					1164054
%Wildan Khaustara Wijaksana		1164058

\documentclass[12pt]{article}
\usepackage{graphicx}    

\begin{document}
\section{Apa itu Flask?}
\subsection{menurut Pennanen, Jussi and others}
Flink adalah kerangka kerja pada perangkat lunak yang dikembangkan bertujuan untuk pelaksanaan pada aplikasi Web dalam bahasa pemrograman Python. Fcalculator pembangunan dimulai tahun 2010 oleh pengembang utama yaitu Armin Ronacher. Mencakup kerangka, antara hal-hal lain, Jinja2-templateengine, dan server pengembangan yang memfasilitasi pengembangan. Flask berlisensi di bawah lisensi BSD\cite{pennanen2018sovellus}.

\subsection{Menurut Nandana Adya Samudera}
Flask adalah sebuah aplikasi microframework untuk bahasa Python yang dibuat dengan toolkit Werkzeug dan template Jinja2. Flask dibuat oleh Armin Ronacher. Pertama kali dirilis pada April 2010. Bila dibandingkan dengan Django, Flask jauh lebih ringan dan cepat karena Flask dibuat dengan ide menyederhanakan inti frameworknya seminimal mungkin, dengan tagline “web development, one drop at a time”, oleh karena itu Flask disebut microframework. Flask dapat membantu kita membuat situs dengan cepat meskipun dengan library yang sederhana. Contoh aplikasi yang menggunakan framework Flask adalah Pinterest, dan LinkdIn. Flask saat ini berada pada lisensi 0.10.1 dan dilisensikan dengan lisensi BSD\cite{samudera2015perancangan}.

\subsection{Menurut Gunawan, Andre and Palit, Henry Novianus and Handojo, Andreas}
Flask merupakan microframework yang dibangun dengan menggunakan bahasa pemrograman Python. Flask digunakan untuk me-develop sebuah aplikasi web. Flask merupakan microframework yang artinya flask membuat sebuah pengerjaan aplikasi web menjadi mudah dan simple karena dapat menjalankan sebuah web hanya dengan menggunakan 1 file Python. Flask membuat susunan kerja yang ringan, dan mudah tetapi juga dapat dikembangkan dengan mudah\cite{gunawan2018aplikasi}.

\subsection{Menurut jurnal Ash-shidiq, Usamah and Rumani, M and Saputra, Randy Erfa}
Flask adalah sebuah microframework yang berbasis python dan dipelopori oleh Armin Ronacher. Bila dibandingkan dengan Django, Flask jauh lebih ringan dan cepat karena Flask dibuat dengan ide menyederhanakan inti framework-nya seminimal mungkin dan seefesien mungkin. Flask dapat membantu dalam membuat situs dengan sangat cepat dan mudah meskipun dengan librari yang lumayan sederhana\cite{ash2017perancangan}.

\section{Aturan-Aturan Flask}

 
\section{Membuat "Hello World" di Flask}


\section{Error yang Muncul serta Solusinya}


\section{Kesimpulan}

	
\end{document}


