%Resume Hello word di Flask

%Kelompok 2 D4 TI / 2B

%Alwan Suryansah				1164033 
%Dinda Ayu Pratiwi				1164034
%Kurnia Sandi					1164042
%Teduh Sanubari					1164054
%Wildan Khaustara Wijaksana		1164058

\documentclass[12pt]{article}
\usepackage[pdftex]{graphicx}
\usepackage{epstopdf}
\usepackage{graphics}


\begin{document}
\section{Apa itu Flask?}
\subsection{menurut Pennanen, Jussi and others}
Flink adalah kerangka kerja pada perangkat lunak yang dikembangkan bertujuan untuk pelaksanaan pada aplikasi Web dalam bahasa pemrograman Python. Fcalculator pembangunan dimulai tahun 2010 oleh pengembang utama yaitu Armin Ronacher. Mencakup kerangka, antara hal-hal lain, Jinja2-templateengine, dan server pengembangan yang memfasilitasi pengembangan. Flask berlisensi di bawah lisensi BSD\cite{pennanen2018sovellus}.

\subsection{Menurut Nandana Adya Samudera}
Flask adalah sebuah aplikasi microframework untuk bahasa Python yang dibuat dengan toolkit Werkzeug dan template Jinja2. Flask dibuat oleh Armin Ronacher. Pertama kali dirilis pada April 2010. Bila dibandingkan dengan Django, Flask jauh lebih ringan dan cepat karena Flask dibuat dengan ide menyederhanakan inti frameworknya seminimal mungkin, dengan tagline “web development, one drop at a time”, oleh karena itu Flask disebut microframework. Flask dapat membantu kita membuat situs dengan cepat meskipun dengan library yang sederhana. Contoh aplikasi yang menggunakan framework Flask adalah Pinterest, dan LinkdIn. Flask saat ini berada pada lisensi 0.10.1 dan dilisensikan dengan lisensi BSD.

Aplikasi yang dibuat dengan framework Flask disimpan dalam satu berkas “.py”. Flask adalah framework yang dikolaborasikan dengan bahasa python yang sederhana namun dapat diperluas dengan beragam pustaka tambahan yang disesuaikan dengan kebutuhan penggunanya. Meskipun Flask belum menyampai versi 1.0 namun dokumentasi yang dmilikinya sangat lengkap yang dapat kita gunakan untuk membuat aplikasi atau situs dengan cepat\cite{samudera2015perancangan}.

\subsection{Menurut Gunawan, Andre and Palit, Henry Novianus and Handojo, Andreas}
Flask merupakan microframework yang dibangun dengan menggunakan bahasa pemrograman Python. Flask digunakan untuk me-develop sebuah aplikasi web. Flask merupakan microframework yang artinya flask membuat sebuah pengerjaan aplikasi web menjadi mudah dan simple karena dapat menjalankan sebuah web hanya dengan menggunakan 1 file Python. Flask membuat susunan kerja yang ringan, dan mudah tetapi juga dapat dikembangkan dengan mudah\cite{gunawan2018aplikasi}.

\subsection{Menurut jurnal Ash-shidiq, Usamah and Rumani, M and Saputra, Randy Erfa}
Flask adalah sebuah microframework yang berbasis python dan dipelopori oleh Armin Ronacher. Bila dibandingkan dengan Django, Flask jauh lebih ringan dan cepat karena Flask dibuat dengan ide menyederhanakan inti framework-nya seminimal mungkin dan seefesien mungkin. Flask dapat membantu dalam membuat situs dengan sangat cepat dan mudah meskipun dengan librari yang lumayan sederhana\cite{ash2017perancangan}.

\subsection{Flask menurut Reza, Robby and Jati, Agung Nugroho and Ahmad, Umar Ali}
Flask merupakan sebuah micro framework yang berbasis kan baha pemrograman python kemudian di pelopori oleh Armin Ronacher . Flask apabila  di bandingkan dengan Django, Flask akan lebih jauh ringan  cepat dan mudah  karena  Flask  dibuat   dengan  ide  untuk menyederhanakan  inti  framework-nya  sebaik  mungkin. Dengan sebuah tag line “web development, one drop at a time”, Flask dapat membantu kita membuat situs dengan sangat cepat meskipun dengan library yang sederhana\cite{reza2016perancangan}.

\subsection{menurut Grinberg, Miguel}
Ada sekali Bahasa pemrograman yang digunakan sebagai framework contohnya Bahasa pemrograman python, diantaranya Django, Flask, Pyramid, Tornado, Bottle, Diesel, Pecan, Falcon dan yang lainnya.Pada tulisan ini, penulis ingin membahas tentang penggunaan Flask sebagai framework untuk menunjang cloud server yang dibuat. Flask merupakan microframework berbasis Bahasa pemrograman python yang dipelopori oleh Armin Ronacher. Bila dibandingkan dengan Django, Flask ini memiliki keunggulan jauh lebih ringan dan lebih cepat \cite{grinberg2018flask}.

\subsection{Menurut Alauddin, Muhammad Fikri}
Flask adalah sebuah microframework untuk Python berbasis Werkzeug, Jinja 2 dan niat baik.Flask berfungsi sebagai pengganti PHP POST dan GET yang dimana pengendali pertukaran data dari HTML ke database. Flask juga menggantikan fungsi Apache sebagai webserver dimana flask berjalan di http://localhost:5000/ \cite{alauddin2017implementasi}.


\section{Aturan-Aturan Flask}

 
 
\section{Membuat "Hello World" di Flask}
\subsection{Membuat "Hello World" dengan Flask menurut Lokhande, PS and Aslam, Fankar and Hawa, Nabeel and Munir, Jumal and Gulamgaus, Murade}
Program Hello World di Flask adalah contoh dasar Flask di mana kita mengimpor kelas Flask menggunakan fungsi impor, kemudian kita mendefinisikan fungsi hello world dan kemudian mengembalikan 'Hello World!' \cite{lokhande2015efficient}.
\begin{verbatim}
from flask import Flask
app = Flask(__name__)
@app.route('/')
def hello_world():
return 'Hello World!'
if __name__ == '__main__':
app.run()
\end{verbatim}

\subsection{Membuat "Hello World" dengan Flask menurut Maia, Italo}
dibawah ini merupakan cara membuat hello wold pada bahasa pemrograman pyton yaitu menggunakan flask ,dimana kita akan mengimport dan membuatnya menjadi "hello world" perhatikan code dibawah ini :
\cite{maia2015building}.
\begin{verbatim}
# coding:utf-8
from flask import flask 
app = Flask(__name__)

@app.route('/')
def hello():
	return "Hello World!"
	
if __name__ == '__main__':
app.run()
\end{verbatim}

\subsection{Membuat "Hello World" dengan Flask menurut Aggarwal, Shalabh }
Flask bertujuan menjaga inti dari kerangka kecil tetapi sangat extensible. Ini membuat aplikasi atau ekstensi menulis sangat mudah dan fleksibel dan memberi pengembang kekuatan untuk memilih konfigurasi yang mereka inginkan untuk aplikasi mereka, tanpa memaksakan pembatasan pada pilihan basis data, mesin templating, dan seterusnya\cite{aggarwal2014flask}.

Menyiapkan aplikasi Hello World sederhana :
\begin{verbatim}
from flask import Flask
app = Flask(__name__)
@app.route('/')
def hello_world():
return 'Hello to the World of Flask!'
if __name__ == '__main__':
app.run()
\end{verbatim}

\subsection{Membuat Hello World dengan Flask menurut Grinberg, Miguel}
Versi kedua dari aplikasi ini yaitu menambahkan sebuah rute yang dinamis. Ketika Anda mengunjungi URL dinamis di browser Anda, Anda akan disajikan dengan ucapan pribadi yang menyertakan nama yang diberikan di URL. Contoh aplikasi Flask dengan rute dinamis.
\begin{verbatim}
From Flask import Flask
app = Flask(__name__)
@app.route(‘/’)
def index():
	return ‘<h1>Hello World !</h1>’
@app.route(‘/user/<name>’)
def user(name):
	return  ‘<h1> Hello, {}!</h1>’.format(name)

\end{verbatim}
Jika Anda telah mengkloning repositori git aplikasi di GitHub, Anda sekarang dapat menjalankan git untuk memeriksa versi aplikasi ini\cite{grinberg2018flask}.

\subsection{Hello World Dengan Flask Oleh Alessandro Zini}

Bahasa yang digunakan untuk implementasi logika sisi server dan Python 2.7: perlu untuk dilarifikasi bahwa, terlepas dari pilihan tersebut bahwa akan diambil beberapa pustaka yang digunakan tidak memperpanjang dukungan ke versi 3 dari Python.

Untuk memahami fungsi kerangka kerja Flask dan proyek Contoh klasik ilmu komputer diusulkan: Hello World.

\begin{verbatim}
from flask import Flask
app = Flask(__name__)
@app.route("/", methods=["GET"])
def hello_world():
return "Hello, World!"
\end{verbatim}

Hasil dari kode ini dan tampilan halaman web yang berisi string "Hello, World!". Yang pertama diimpor sebagai operasi pertama Kelas flask, instantiate suatu objek. Parameter yang ditentukan dan variabel khusus: "nama".  Variabel ini digunakan di lingkungan Python, dan masih banyak lagi yang lebih spesifik dari Flask, untuk mengidentifikasi apakah modul saat ini telah tersedia di aplikasi utama saya, atau bisa diimpor dari modul eksternal; di kasus pertama, kita akan memiliki nilai utama, keduanya akan berisi nama modul eksternal dari mana modul yang saat ini diimpor. Mekanisme ini diakali dan digunakan oleh Flask untuk mengatur parameter pencarian dengan benar\cite{ziniqr}.

\subsection{Hello World Flask by Django, Flask}
Routes merupakan sebuah kumpulan dari URL kemudian di implementasi oleh aplikasi. Dalam FLask, handler merupakan route aplikasi yang ditulis sebagai fungsi dari Python yang disebut dengan view function. View function akan memetakan satu atau lebih URL sehingga Flask akan tahu apa yang harus ia lakukan setiap kali klien memanggil sebuah URL. Berikut ini merupakan contoh Flask Route\cite{djangoweb}.
\begin{verbatim}
import flask

app = flask.Flask(__name__)

@app.route('/')

def hello():

return 'Hello’

app.run()

\end{verbatim}



\section{Error yang Muncul serta Solusinya}



\section{Implementasi dengan menggunakan Flask}
Implementasi web service menggunakan flask web framework, library pandas untuk membaca file csv dan membentuk fitur matrix X dan vektor target y. Penggunan library scikit learn agar dapat menggunakan modul naive bayes, serta untuk melakukan pembagian data training dan testing dengan fungsi train test split. Modul terakhir yang digunakan adalah pickle untuk menyimpan classifier yang telah dibuat ke dalam disk, agar tidak melakukan training berulang-ulang untuk setiap request yang dikirim.

Listing program 1 berfungsi untuk membuat naive bayes classifier, kemudian menyimpannya dengan nama nbp imadiebetspkl agar dapat dipanggil oleh web service. Listing program 2 menampilkan kode pembuatan web service yang diimplementasikan pada fungsi predict, sedangkan Listing program 3 menampilkan contoh kode aplikasi dalam bahasa python yang melakukan request ke web service dengan mengirimkan parameter 'pregn':6,'gluc':148,
'bp':72,'sk':35,'ins':0,
'bmi':33.6,'ped':0.627,'age':50, dimana akan menghasilkan variabel output dengan nilai {'results': [1]} yang berarti positif diabetes \cite{setyawan2017implementasi}.

\begin{verbatim}
Listing program 1. Pembuatan naive bayes classifier
url = 'pima-indians-diabetes.csv'
col_names = ['pregnant', 'glucose', 'bp', 'skin',
'insulin', 'bmi', 'pedigree', 'age', 'label']
pima = pd.read_csv(url, header=None, names=col_names)
feature_cols = ['pregnant', 'glucose', 'bp', 'skin',
'insulin', 'bmi', 'pedigree', 'age']
X = pima[feature_cols]
y = pima.label
X_train, X_test, y_train, y_test =
train_test_split(X, y, stratify=y, test_size=0.25, random_state = 0)
nb = GaussianNB()
nb.fit(X_train, y_train)
pickle.dump(nb, open("nb_pimadiabtes.pkl","wb"))

Listing program 2. Implementasi web service dengan flask
import numpy as np
from flask import Flask, request, abort, jsonify
import pickle
nbclassifier = pickle.load(open("nb_pimadiabtes.pkl","rb"))
app = Flask(__name__)
@app.route('/api', methods=['POST'])
def predict():
data = request.get_json(force = True)
predict_request = [data['pregn'],data['gluc'],data['bp']
data['sk'],data['ins'],data['bmi'],data['ped'],data['age']]
predict_request = np.array(predict_request)
y_result = nbclassifier.predict(predict_request)
output = [int(y_result[0])]
return jsonify(results=output)
if __name__ == '__main__':
app.run(debug = True)

Listing program 3. Contoh kode yang mengakses web service
import json
import requests
url = "http://127.0.0.1:5000/api"

Listing program 3. Lanjutan
data= json.dumps({'pregn':6,'gluc':148,'bp':72,'sk':35,
'ins':0,'bmi':33.6,
'ped':0.627,'age':50})
r = requests.post(url, data)
output = r.json()
\end{verbatim}






\section{Kesimpulan}
Dari berbagai pengertian mengenai bahasa pemograman flask kita dapat simpulkan bahwa bahasa pemograman flask merupakan bahasa pemrograman open source yang banyak digunakan untuk menangani beberapa jenis masalah dalam pemrograman salah satunya membuat suatu framework. flask banyak digunakan untuk meningkatkan kualitas perangkat lunak, produktivitas pengembang, portabilitas program, dan integrasi komponen.

kata lain Flask adalah framework aplikasi web mikro yang ditulis dalam bahasa Python dan berbasiskan toolkit Wekzeug dan template engine Jinja2 dan berlisensi BSD. Pada tahun 2015, versi paling stabil Flask adalah versi 0.10.1. Contoh aplikasi yang menggunakan framework Flask adalah Pinterest, LinkedIn, dan tentu saja halaman web Flask itu sendiri \cite{solihin2016implementasi}. 

maka dari dibuatnya resume ini kita dapat mengetahui: 
	
\begin{enumerate}
\item mengetahui apa itu flask.
\item wawasan lebih luas tentang flask
\item mengetahui bagaimana cara pembuatan "hello world di flask"
\item mengetahui cara memecahkan masalah pada suatu error code di flask
\end{enumerate}	
\end{document}




