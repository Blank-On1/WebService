\documentclass[12pt,a4paper]{article} 
\usepackage{graphicx}
\begin{document}

\section{Flask}
\subsection{Instlasi Flask}
\subsubsection{On Linux}
Untuk melakukan instalasi flask, ada beberapa tahap yang harus dilakukan yaitu :
\begin{enumerate}
\item Pastikan anda sudah menginstal pyhton baik pyhton versi 2.x maupun 3.x. Jika belum instal, instal terlebih dahulu.

\item Untuk melakukan instalasi flask, kita harus membuat sebuah virtual environment lalu instal flask di dalamnya. Berikut adalah cara instal virtual enviroment / virtualenv.

\$ sudo apt-get install python-virtualenv

\item Lalu, buat satu direktori di hard disk kita untuk tempat pengerjaan project. 

\$ mkdir myproject

\$ cd myproject

\item Setelah itu, kita masuk ke direktori yang sudah di buat sebelumnya dan jalankan terminal untuk menjalankan virtualenv untuk membuat virtualenv di dalam direktori tersebut. Berikut adalah cara menjalankan virtualenv.

\$ virtualenv venv

New python executable in venv/bin/python

Installing setuptools, pip............done.

\item Lalu, berikut adalah cara untuk mengaktifkan virtualenv.

\$ . venv/bin/activate 

dan jika ingin menonaktifkannya, berikut caranya 

\$ deactivate

\item Selanjutnya, kita melakukan penginstalan flask

\$ pip install Flask
\end{enumerate}
\subsubsection{On Mac OS}
1. Instal pip terlebih dahulu menggunakan perintah :\\
	python get-pip.py\\

2. Selanjutnya instal flask dengan menggunakan perintah berikut :\\
	sudo pip install Flask\\

3. Sekarang, jalankan Simple.py untuk memastikannya berfungsi dengan menggunakan perintah berikut :\\
	python simple.py\\

Lalu, jalankan di web browser dengan perintah :\\
http://127.0.0.1:5000/

ketika sudah selesai, hentikan server dengan menekan kontrol-C.


\end{document}
