\documentclass[12pt,a4paper]{article}
\usepackage[left=3.00cm, right=2.00cm, bottom=2.00cm, top=3.00cm]{geometry}
\begin{document}
\title{Postman dan Swagger}
\maketitle
\begin{enumerate}
\item Fransiscus Ivan Martongam      1164039 \\
\item Lalita Chandiany Adiputri      1164043\\
\item Eko Cahyono Putro              1164035\\
\item Lidwina Triniska Gulo          1164044\\
\item Sulpadianti Bunyamin           1164096\\
\end{enumerate}

\section{Pengertian Postman}
Postman merupakan sebuah software yang memuat fungsi lengkap pengembangan sistem dalam mengirimkan dan menerima respon server. Software ini mendukung pengembangan sistem REST API dengan mengklasifikasi request berdasarkan request method, URL dan parameter-parameter request. Postman juga adalah sebuah aplikasi (berupa plugin) untuk browser chrome, fungsinya adalah sebagai REST Client atau istilahnya adalah aplikasi yang digunakan untuk melakukan uji coba REST API yang telah kita buat.

\section{Implementasi Postman}
Implementasi tidak hanya aktivitas, melainkan suatu kegiatan yang terencana untuk mencapai
tujuan kegiatan. Tahap implementasi pada penelitian ini tidak dilakukan hanya dalam satu proses,
Tetapi dilakukan dalam beberapa sub proses yaitu membangun lingkungan pengembangan sistem, 
mendesain struktur table, function dan stored procedure pada database, mengembangkan sistem back-end (coding) 
dan menyesuaikan dengan database, dan mendesain struktur rewrite pada web server. 

\section{Pengujian}
Untuk memastikan bahwa sistem berjalan sesuai dengan rencana pengembangan sistem dan proses bisnis yang difasilitasi, diperlukan sebuah skenario pengujian sistem. Pada penelitian ini, pengujian akan dilakukan dengan menggunakan software Postman dan dibagi menjadi tiga skenario pengujian, yaitu: Pengujian otentikasi token untuk memastikan prinsip REST terpenuhi, pengujian dengan metode equivalent partitioning untuk memastikan nilai-nilai masukan sesuai dengan rencana pengembangan sistem, pengujian fungsional untuk memastikan setiap titik akses API.


\end{document}
