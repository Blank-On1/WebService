%Resume Membuat APlikasi Dengan FlaskGer

%Kelompok 2 D4 TI / 2B

%Alwan Suryansah				1164033 
%Dinda Ayu Pratiwi				1164034
%Kurnia Sandi					1164042
%Teduh Sanubari					1164054
%Wildan Khaustara Wijaksana		1164058

\documentclass[12pt]{article}
\usepackage{graphicx}    

\begin{document}

\section{Membuat Aplikasi dengan Flask}

\subsection{membuat aplikasi dengan flask dalam jurnal Web Service Dan  Analisis Kinerja Algoritma Klasifikasi Data Mining Untuk Memprediksi Diabetes Mellitus}
membuat aplikasi dengan menggunakan flask web framework, library pandas untuk membaca file csv dan membentuk fitur matrix X dan vektor target y. Penggunan library scikit-learn agar dapat menggunakan modul naive bayes, serta untuk melakukan pembagian data training dan testing dengan fungsi train test split. Modul terakhir yang digunakan adalah pickle untuk menyimpan classifier yang telah dibuat ke dalam disk, agar tidak melakukan training berulang-ulang untuk setiap request yang dikirim \cite{setyawan2017implementasi}. 

\subsection{Reza, Robby and Jati, Agung Nugroho and Ahmad, Umar Ali}
Aplikasi yang dibuat dengan Flask disimpan dalam satu berkas “.py”. Flask adalah framework yang sederhana namun dapat diperluas dengan beragam pustaka tambahan yang disesuaikan dengan kebutuhan penggunanya.membuat aplikasi menggunakan flask akan menjadi sangat cepat, Meskipun Flask belum menyampai versi 1.0 namun dokumentasi yang dmilikinya sangat lengkap\cite{reza2016perancangan}. 

\subsection{Membuat Aplikasi dengan flasgger}
Flasgger adalah salah satu ekstensi Flask yang di gunakan untuk membantu pembuatan API Flask dengan dokumentasi dan live playground yang didukung oleh SwaggerUI. Kemudian Anda juga dapat menentukan struktur API menggunakan file YAML dan Flasgger membuat semuanya sama untuk Anda dan dapat menggunakan skema yang sama untuk memvalidasi sebuah data, Flask dapat membuat susunan kerja yang ringan, dan mudah tetapi juga dapat dikembangkan dengan mudah\cite{gunawan2018aplikasi}.

\subsection{Membuat Aplikasi dengan Flask Menurut Jurnal Aplikasi Pengendali Kamera DSLR Nirkabel Tipe Low End Berbasis Android}
Untuk membuat aplikasi ini terdiri dari tahapan penelitian yang dilakukan. Tahapan tersebut terdiri dari gambaran umum sistem yang akan dijelaskan dalam blok diagram sistem, analisis dan perancangan sistem, analisis dan konfigurasi jaringan yang terdiri dari konfigurasi jaringan berupa konfigurasi interfaces jaringan, konfigurasi IP statis, serta konfigurasi dan pengaturan access point. Tahapan selanjutnya setelah konfigurasi jaringan yaitu tahapan konfigurasi gphoto2 library, konfigurasi ISO, konfigurasi aperture, konfigurasi shutter speed, konfigurasi preview.py, konfigurasi snap.py serta tahapan terakhir yaitu konfigurasi flask API \cite{computingaplikasi}.

\subsection{Membuat Aplikasi Flask}
Flask merupakan microframework yang dibangun dengan
menggunakan bahasa pemrograman Python. Flask digunakan
untuk me-develop sebuah aplikasi web. Flask merupakan
microframework yang artinya flask membuat sebuah pengerjaan
aplikasi web menjadi mudah dan simple karena dapat menjalankan
sebuah web hanya dengan menggunakan 1 file Python. Flask
membuat susunan kerja yang ringan, dan mudah tetapi juga dapat
dikembangkan dengan mudah\cite{gunawan2018aplikasi}.


\end{document}


