\documentclass[a4paper]{article}
\begin{document}
\section{JSON}
\subsection{Definisi JSON}
JSON atau JavaScript Object Notation adalah salah satu bentuk format yang ringkas untuk melakukan pertukaran sebuah data di dalam computer. JSON sendiri berbasis teks dan mudah untuk dibaca oleh manusia serta dapat digunakan untuk representasi struktur data sederhana JSON juga dapat digunakan untuk proses transmisi data terstruktur melalui media koneksi jaringan yang disebut Serialisasi.
\subsection{Definisi JSON}
JSON  adalah  bagian  dari  sebuah bahasa  pemrograman  JavaScript  JSON juga merupakan format teks yang sepenuhnya independen tetapi menggunakan konvensi yang  familiar  dengan  bahasa  pemrograman  dari  parent-C,  termasuk  bahasa C,  Java,  Java Script,  Perl, Python,  dan lain sebagainya.  Kelebihan  inilah  yang  membuat  JSON  menjadi  sebuah  bahasa yang disebut  data-interchange yang ideal.
\subsection{Struktur pada JSON}
JSON memiliki 2 struktur,yaitu:
1.	Kumpulan pasangan nama/nilai.
Dalam beberapa Bahasa pemrograman, ini biasanya sering disebut seperti objek, rekaman, struktur, kamus, tabel hash, daftar berkunci atau array assosiatif.
2.	Daftar nilai terurutkan.
Dalam kebanyakan Bahasa pemrograman, ini biasanya sering disebut seperti array, vektor, daftar, atau urutan.
Struktur data tersebut sering kali disebut sebagai struktur data universal. Semua bahasa pemrograman modern mendukung struktur data tersebut dalam bentuk yang sama maupun berbeda.
\subsection{Pengertian Lain Dari JSON}
JSON atau biasa dilafalkan dengan “Jason” merupakan singkatan dari JavaScript Object Notation adalah suatu format ringkas pertukaran data computer. Format Json berbasis teks dan mudah dibaca-manusia serta digunakan untuk merepresentasikan struktur data sederhana dan larik asosiatif. JSON juga seringkali digunakan untuk transmisi datayang  terstruktur melalui suatu koneksi jaringan pada suatu proses yang disebut serialisasi.
\end{document}
