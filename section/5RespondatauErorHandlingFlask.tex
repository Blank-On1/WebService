\documentclass[12pt,a4paper]{article}
\usepackage[left=3.00cm, right=2.00cm, bottom=2.00cm, top=3.00cm]{geometry}
\usepackage{graphics}
\begin{document}
\title{Postman dan Swagger}
\maketitle
\begin{enumerate}
\item Fransiscus Ivan Martongam      1164039 \\
\item Lalita Chandiany Adiputri      1164043\\
\item Eko Cahyono Putro              1164035\\
\item Lidwina Triniska Gulo          1164044\\
\item Sulpadianti Bunyamin           1164096\\
\end{enumerate}

\section{Pengertian Flask}
Flask merupakan sebuah Microweb framework yang dicantumkan ke dalam bahasa pemrograman Python berdasar dari Werkzeug toolkit dan template engine Jinja2. Berlisensi BSD. Flask disebut micro framework karena tidak membutuhkan alat-alat tertentu atau pustaka. Flask tidak memiliki database abstraction layer, validasi form, atau komponen lain karena ada pustaka pihak ketiga yang menyediakan fungsi umum.

\section{Error Handling di Flask}
Error handling dalam flask akan munculnya "Internal Server Erro" pesan yang muncul di sesi terminal saat aplikasi dijalankan. Kemudian akan menampilkan  stack trace dari kesalahan yang terjadi. Stack trace sangat berguna untuk mencari error karena memberikan urutan pemanggilan mana yang menyebabkan sebuah error tersebut terjadi. Stack trace akan memperlihatkan bagian mana yang menjadi penyebabnya. Error ini muncul dari SQLAlchemy yang mencoba menulis username baru ke database, tapi database menolak karena kolom username sebelumnya sudah diatur dengan unique=True.

\subsection{Debug Mode}
Halaman error yang muncul sebelumnya cocok dipakai oleh aplikasi yang sudah diunggah ke sebuah production server. Jika ada error, user akan diberitahu dengan sebuah halaman khusus (yang nanti akan kita perbagus), dengan pesan error yang lebih detail disimpan di file log server.
Tapi saat aplikasinya sedang dibuat, kita tentu menginginkan debug mode untuk diaktifkan. Jika Flask aktif dalam mode ini, kita akan mendapatkan pesan error yang sangat membantu yang akan ditampilkan di browser.

\begin{verbatim}
(venv) $ export FLASK_DEBUG=1
\end{verbatim}

Saat aplikasi sedang dibuat, di butuhkan debug mode untuk mengaktifkan. Apabila Flask aktif dalam mode ini,  pesan error akan muncu dan sangat membantu dan akan ditampilkan di browser. Untuk mengaktifkan debug mode, stop dulu aplikasi, lalu atur environment variable. 
Setelah mengatur FLASK_DEBUG, restart ulang server. Pesan yang ditampilkan saat memulai server menjadi agak berbeda dibanding sebelumnya:
\begin{verbatim}
(venv) microblog2 $ flask run
 * Serving Flask app "microblog"
 * Forcing debug mode on
 * Running on http://127.0.0.1:5000/ (Press CTRL+C to quit)
 * Restarting with stat
 * Debugger is active!
 * Debugger PIN: 177-562-960
 \end{verbatim}

\end{document}
