\documentclass[12pt,a4paper]{article}
\usepackage[left=3.00cm, right=2.00cm, bottom=2.00cm, top=3.00cm]{geometry}
\linespread{1.5}
\begin{document}
\title{PORT JARINGAN}
\date{}
\maketitle

\begin{itemize}
\item
NAMA KELOMPOK 5\\
Ajis Trigunawan			1164031\\
Alimu Dzul Ikroom		1164032\\
Muhammad Hanafi			1164092\\
Riki Karnovi			1164052\\
Yoga Sakti Hadi P		1164059\\
\end{itemize}

\section{PORT}
\subsection{Pengertian Port}
\hspace{1cm}
Port adalah tatacara yang memberi ijin sebuah komputer yang memberi dukungan untuk beberapa bagian koneksi dengan komputer yang lain. Port juga dapat mengenali sebuah aplikasi dan layanan yang sedang menggunakan koneksi di dalam sebuah jaringan TCP/IP. Sehingga port juga mengenali salah satu prosses yang di mana server memberikan layanan terhadap klien yang meminta layanan tersebut. Port adalah salah satu media penghubung untuk melewatkan atau mengirim data masuk atau keluar baik pada sebuah komputer maupun dalam penggunaan jaringan komunikasi. Port pada salah satu penggunaannya pada jaringan komunikasi merupakan nama yang diberikan pada titik akhir koneksi. Nama port yaitu berupa angka angka sebagai pembeda tiap  jenis port. Contoh dari penamaan port yang dipakai pada web sebagai transportasi data yaitu port 80.

\subsection{Jenis-jenis Port}
\hspace{1cm}
Klasikafikasi port yang ada saat ini tentulah sangat banyak, salah satu contohnya adalah port knocking, Port Knocking adalah port yang menjaga akses  dari user yang tidak diketahui, sebenarnya system kinerja port knocking bisa membuat user tertentu saja yang bisa mengakses yang telah ditetapkan dan kelebihan port knocking ini dapat menghubungkan semua port yang tidak terbuka meski tidak terbuka, service yang tersedia tetap berlangsung.

\hspace{1cm}
Sesuai dengan kegunaan sebuah Port yaitu media penghubung untuk melewatkan data masuk atau keluar pada jaringan komunikasi. Sebagai contoh jenis-jenis port yang digunakan pada beberapa protokol jaringan :\\
\begin{enumerate}
\item HTTP (Hypertext Transfer Protocol) menggunakan port 80.
\item HTTPS (HTTP Secure) menggunakan port 443.
\item FTP (File Transfer Protocol menggunakan port 21.
\item SSH (Secure Shell) menggunakan nama port 22.
\item Telnet menggunakan port 23.
\item IMAP (Internet Message Access Protocol) menggunakan port 143.
\end{enumerate}

\subsection{Fungsi Port}
\hspace{1cm}
Port mempunyai fungsi sangat vital dalam sebuah jaringan sebagai pintu masuk atau jalur yang digunakan untuk melakukan koneksi masuk dan keluar ke komputer lain pada jalur komunikasi. Setiap port mempunyai fungsi masing masing contohnya sebagai berikut:
\begin{enumerate}
\item Port 21 yaitu port FTP yang berfungsi untuk tukar menukar data dalam suatu jaringa.
\item Port 22 yaitu port SSH yang berfungsi untuk  melakukan koneksi amandalam suatu jaringan.
\item Port 80 yaitu port Http yang berfungsi untuk mentransfer dokumen dalam World Wide Web (WWW).
\end{enumerate}

\end{document}
