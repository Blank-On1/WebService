\documentclass[12pt,times new roman,a4paper]{article}
\usepackage[left=3.00cm, right=2.00cm, bottom=2.00cm, top=3.00cm]{geometry}
\linespread{1.5}
\begin{document}
\title{PORT JARINGAN}
\maketitle

\begin{itemize}
\item
NAMA KELOMPOK 5\\
Ajis Trigunawan			1164031\\
Alimu Dzul Ikroom		1164032\\
Muhammad Hanafi			1164092\\
Riki Karnovi			1164052\\
Yoga Sakti Hadi P		1164059\\
\end{itemize}

\section{Pengertian Port}
\hspace{1cm}
Port adalah tatacara yang memberi ijin sebuah komputer yang memberi dukungan untuk beberapa bagian koneksi dengan komputer yang lain. Port juga dapat mengenali sebuah aplikasi dan layanan yang sedang menggunakan koneksi di dalam sebuah jaringan TCP/IP. Sehingga port juga mengenali salah satu prosses yang di mana server memberikan layanan terhadap klien yang meminta layanan tersebut. Port adalah salah satu media penghubung untuk melewatkan atau mengirim data masuk atau keluar baik pada sebuah komputer maupun dalam penggunaan jaringan komunikasi. Port pada salah satu penggunaannya pada jaringan komunikasi merupakan nama yang diberikan pada titik akhir koneksi. Nama port yaitu berupa angka angka sebagai pembeda tiap  jenis port. Contoh dari penamaan port yang dipakai pada web sebagai transportasi data yaitu port 80.

\end{document}
