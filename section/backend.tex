\section{Backend}
Back-end atau server-side adalah merupakan bagaimana sebuah website berkerja, meng-update
dan berubah. back-end adalah sesuatu system yang dimana User tidak akan dapat melihatnya di dalam browser,
seperti Database dan Server. biasanya orang - orang yang bekerja di bagian back-end di panggil atau disebut sebagai
Programmers atau Developer. Back-end developers adalah orang yang paling khawatir tentang hal - hal yang menyangkut keamanan,
struktur sistem dan manajemen konten. sebenarnya para back-end develper juga mengetahui tentang front-end seperti HTML dan CSS.
namun itu bukanlah bidang mereka bekerja. 

\section{TugasBackend}
Back-end biasanya mengacu pada program dan skrip yang bekerja di dalam server, untuk membuat sebuah halaman web yang dinamis dan interatif. Back-end memiliki tugas-tugas yaitu seperti :

Desain Informasi pada web
Pemroswsan form
Pemrograman dalam database
Aplikasi Berbasis Web

Dari tugas - tugas tersebut Back-end memiliki tiga bagian diantaranya yaitu server,aplikasi, dan database.