%Resume protokol
%kelompok 3 D4 TI-2B 
%Fikri aldi nugraha                  1164038
%Nur Arkhamia Batubara               1164049 
%Miftahul Hasanah                    1164046 
%Si Made Angga Dwitya P              1164053 
%Widary Anggraini Mindo V Siahaan    1164057


\section{Pengenalan Protokol} 
 Suatu standar pertukaran informasi. Komputer dengan system operasi dan software berbeda dapat saling berkomunikasi melalui internet 
 karena pengadopsian protokol. Seperangkat aturan umum (atau bahasa) yang mengijinkan komputer-komputer untuk saling berkomunikasi. 
 Protokol yang standart adalah TCP/IP (Transmission Control Protocol/Internet Protocol). Web menggunakan server protokol HTTP agar 
 komputer dapat mengakses file , melakukan pencetakan, berkomunikasi, dan menyediakan layanan lain bagi user lain dalam jaringan.
 
 \section{Jenis-Jenis Protokol}
Protokol jaringan adalah berbagai protokol yang terdapat dari lapisan  teratas sampai terbawah yang ada dalam sederetan protocol.
Di pandang dari sudut komunikasi data,ada beberapa protokol yang banyak digunakan pada jaringan computer, di antaranya:

\subsection{TCP/IP}
TCP/IP merupakan protokol standar pada jaringan internet yang tidak tergantung pada jenis computer yang digunakan.
Barangkali perlu dicatat bahwa TCP/IP adalah perlengkapan standar pada sistem operasi Unix dan turunannya.
Saat ini mesin novell,SUN maupun Machintosh sudah dilengkapi protokol standar TCP/IP ini.

  \subsection{} 
   
    
  \subsection{} 
    
