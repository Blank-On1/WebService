\section{Definisi Frontend}
frontend bisa disebut tampilan utama dari sebuah website pada frontend biasanya ditampilkan beberapa konten-konten yang bisa diakses oleh pengguna atau user yang menggunakan website tersebut. frontend juga berfungsi untuk user interace dari setiap web site. Biasanya frontend hanya menampilkan fungsi fungsi dari kontent sebuah web site seperti fungsi sebuah tombol untuk mengirim berkas atau untuk menampilkan konten konten yang lainnya dalam website tersebut.

\section{Fungsi Front-end}
Bagian Front-end dari sebuah website adalah bagian yang bisa langsung dilihat oleh use r(pengguna). User (pengguna) juga bisa langsung berinteraksi pada bagian ini, bagian ini dibangun menggunakan HTML, CSS, dan JavaScript. Dengan hanya menggunakan HTML dan CSS itu sudah bisa membuat sebuah website .
- HTML (Hyper Text Markup Language) adalah bahasa yang digunakan untuk membuat sebuah halaman web. Semua website yang kamu kunjungi dibuat menggunakan HTML.
- CSS (Cascading Style Sheets) adalah bahasa pemrograman yang mengontrol tampilan HTML. CSS menentukan warna, font, gambar background.


