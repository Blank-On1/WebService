2C :
Tugas koding fork dan pull request https://github.com/peuyeum/Keju :
jalankan aplikasi flask dan cek http://127.0.0.1:5000/static/dist/index.html
gunakan library scholarly https://github.com/OrganicIrradiation/scholarly

1. /peneliti POST['nama peneliti'] : cari profile scholar peneliti kembalikan data
2. /peneliti/{idscholar} GET : cari profile peneliti dari id google scholar
3. /peneliti/surel/{surel} GET : cari profile scholar dengan email
4. /peneliti/afiliasi/{afiliasi} GET : cari profile scholar dengan affiliation


2B :
Tugas koding fork dan pull request https://github.com/peuyeum/Keju :
jalankan aplikasi flask dan cek http://127.0.0.1:5000/static/dist/index.html
Gunakan librari https://github.com/zhiyzuo/python-scopus
https://dev.elsevier.com/tips/AuthorSearchTips.htm

1. /scopus/FIRSTAUTH/{penulispertama} GET : mendapatkan daftar pencarian dengan input penulis pertama
2. /scopus/AFFIL/{afiliasi} GET : mendapatkan daftar pencarian dengan input afiliasi
3. /scopus/AUTHLASTNAME/{namabelakang} GET : mendapatkan daftar pencarian dari nama belakang
4. /scopus/AUTHFIRST/{namadepan} GET : mendapatkan daftar pencarian dari nama depan
5. /scopus/SUBJAREA/{subjekarea} GET : mendapatkan daftar pencarian dari subjek area



Parameter :
1. Orang 1 commit dalam sehari dikali 20
2. error -5